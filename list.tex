% #1 = list name
\def\@pytexList@new#1{%
	\expandafter\def\csname #1@size\endcsname{0}%
	\expandafter\def\csname #1@push\endcsname##1{\@pytexList@push{#1}{##1}}%
	\expandafter\def\csname #1@popd\endcsname{\@pytexList@popd{#1}}%
	\expandafter\def\csname #1@getim\endcsname##1{\@pytexList@getim{#1}{##1}}%
	\expandafter\def\csname #1@getendim\endcsname##1{\@pytexList@getendim{#1}{##1}}%
	\expandafter\def\csname #1@delete\endcsname{\@pytexList@delete{#1}}%
}

% #1 = list name, #2 = item
\def\@pytexList@push#1#2{%
	\expandafter\def\csname #1@\csname #1@size\endcsname\endcsname{#2}%
	\expandafter\edef\csname #1@size\endcsname{\the\numexpr\csname #1@size\endcsname+1}%
}

% #1 = list name
\def\@pytexList@popd#1{%
	\expandafter\edef\csname #1@size\endcsname{\the\numexpr\csname #1@size\endcsname-1}%
	\expandafter\let\csname#1@\csname #1@size\endcsname\endcsname\undefined%
}

% #1 = list name, #2 = index
\def\@pytexList@getim#1#2{\csname#1@#2\endcsname}

% #1 = list name, #2 = index from end (0 = last, 1 = before last)
\def\@pytexList@getendim#1#2{
	\csname#1@\the\numexpr\csname#1@size\endcsname-1-#2\relax\endcsname
}

% #1 = list name
\def\@pytexList@delete#1{%
	\ifnum\csname #1@size\endcsname=0 %
		\expandafter\let\csname #1@size\endcsname\undefined%
		\expandafter\let\csname #1@push\endcsname\undefined%
		\expandafter\let\csname #1@getim\endcsname\undefined%
		\expandafter\let\csname #1@delete\endcsname\undefined%
		\def\@pytexTMP{\relax}%
	\else%
		\@pytexList@popd{#1}%
		\def\@pytexTMP{\@pytexList@delete{#1}}%
	\fi%
	\@pytexTMP%
}
