
% #1 = stack name to create
\def\@pytexStack@new#1{%
	\e\def\csname #1@size\endcsname{0}%
}

% #1 = stack name, #2 = item
\def\@pytexStack@push#1#2{%
	\edef\@pytexTMP@stackElement{#1@\csname #1@size\endcsname}%
	\e\def\csname\@pytexTMP@stackElement\endcsname{#2}%
	\e\edef\csname #1@size\endcsname{\the\numexpr\csname #1@size\endcsname+1\relax}%
}

% #1 = stack name, #2 = macro to store result in
\def\@pytexStack@peek#1#2{%
	\edef\@pytexTMP@stackElement{#1@\the\numexpr\csname #1@size\endcsname-1\relax}%
	\e\e\e\let\e\e\e#2\e\csname\@pytexTMP@stackElement\endcsname%
}

% #1 = stack name, #2 = macro to store result in
\def\@pytexStack@pop#1#2{%
	\@pytexStack@peek{#1}{#2}%
	\e\edef\csname #1@size\endcsname{\the\numexpr\csname #1@size\endcsname-1\relax}%
}

% #1 = stack name
\def\@pytexStack@clear#1{% TODO: test
	\ifnum \csname #1@size\endcsname = 0\else%
		\e\edef\csname #1@size\endcsname{\the\numexpr\csname #1@size\endcsname-1\relax}%
		\e\let\csname #1@size\endcsname\undefined%
		\@pytexStack@clear{#1}%
	\fi%
}

% #1 = stack name
\def\@pytexStack@delete#1{% TODO: test
	\@pytexStack@clear{#1}%
	\e\let\csname #1@size\endcsname\undefined%
}
