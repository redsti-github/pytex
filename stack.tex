
% #1 = stack name to create
\def\@pytexStack@new#1{%
	\e\def\csname #1@size\endcsname{0}%
	\e\def\csname #1@push\endcsname##1{\@pytexStack@push{#1}{##1}}%
	\e\def\csname #1@pop\endcsname##1{\@pytexStack@pop{#1}{##1}}%
	\e\def\csname #1@peek\endcsname##1{\@pytexStack@peek{#1}{##1}}%
	\e\def\csname #1@popd\endcsname{\@pytexStack@popd{#1}}%
}

% #1 = stack name, #2 = item
\def\@pytexStack@push#1#2{%
	\e\def\csname#1@\csname #1@size\endcsname\endcsname{#2}%
	\e\edef\csname #1@size\endcsname{\the\numexpr\csname #1@size\endcsname+1\relax}%
}

% #1 = stack name, #2 = macro to store result in
\def\@pytexStack@peek#1#2{%
	\edef\@pytexTMP@stackElement{#1@\the\numexpr\csname #1@size\endcsname-1\relax}%
	\e\e\e\let\e\e\e#2\e\csname\@pytexTMP@stackElement\endcsname%
}

% #1 = stack name
\def\@pytexStack@popd#1{%
	\e\edef\csname #1@size\endcsname{\the\numexpr\csname #1@size\endcsname-1\relax}%
}

% #1 = stack name, #2 = macro to store result in
\def\@pytexStack@pop#1#2{%
	\@pytexStack@peek{#1}{#2}%
	\@pytexStack@popd{#1}%
}

% #1 = stack name
\def\@pytexStack@delete#1{%
	\ifnum\csname #1@size\endcsname=0 %
		\expandafter\let\csname #1@size\endcsname\undefined%
		\expandafter\let\csname #1@push\endcsname\undefined%
		\expandafter\let\csname #1@pop\endcsname\undefined%
		\expandafter\let\csname #1@popd\endcsname\undefined%
		\expandafter\let\csname #1@peek\endcsname\undefined%
		\def\@pytexTMP{\relax}%
	\else%
		\@pytexStack@popd{#1}%
		\def\@pytexTMP{\@pytexList@delete{#1}}%
	\fi%
	\@pytexTMP%
}
