\let\e\expandafter
\catcode`@=11
\catcode`_=11

\def\@pytexError#1{%
	\errmessage{#1}%
}


% #1 = stack name to create
\def\@pytexStack@new#1{%
	\e\def\csname #1@size\endcsname{0}%
	\e\def\csname #1@push\endcsname##1{\@pytexStack@push{#1}{##1}}%
	\e\def\csname #1@pop\endcsname##1{\@pytexStack@pop{#1}{##1}}%
	\e\def\csname #1@peek\endcsname##1{\@pytexStack@peek{#1}{##1}}%
	\e\def\csname #1@popd\endcsname{\@pytexStack@popd{#1}}%
}

% #1 = stack name, #2 = item
\def\@pytexStack@push#1#2{%
	\e\def\csname#1@\csname #1@size\endcsname\endcsname{#2}%
	\e\edef\csname #1@size\endcsname{\the\numexpr\csname #1@size\endcsname+1\relax}%
}

% #1 = stack name, #2 = macro to store result in
\def\@pytexStack@peek#1#2{%
	\edef\@pytexTMP@stackElement{#1@\the\numexpr\csname #1@size\endcsname-1\relax}%
	\e\e\e\let\e\e\e#2\e\csname\@pytexTMP@stackElement\endcsname%
}

% #1 = stack name
\def\@pytexStack@popd#1{%
	\e\edef\csname #1@size\endcsname{\the\numexpr\csname #1@size\endcsname-1\relax}%
}

% #1 = stack name, #2 = macro to store result in
\def\@pytexStack@pop#1#2{%
	\@pytexStack@peek{#1}{#2}%
	\@pytexStack@popd{#1}%
}

% #1 = stack name
\def\@pytexStack@clear#1{% TODO: test
	\ifnum \csname #1@size\endcsname = 0\else%
		\e\edef\csname #1@size\endcsname{\the\numexpr\csname #1@size\endcsname-1\relax}%
		\e\let\csname #1@size\endcsname\undefined%
		\@pytexStack@clear{#1}%
	\fi%
}

% #1 = stack name
\def\@pytexStack@delete#1{% TODO: test
	\ifnum\csname #1@size\endcsname=0 %
		\expandafter\let\csname #1@size\endcsname\undefined%
		\expandafter\let\csname #1@push\endcsname\undefined%
		\expandafter\let\csname #1@pop\endcsname\undefined%
		\expandafter\let\csname #1@popd\endcsname\undefined%
		\expandafter\let\csname #1@peek\endcsname\undefined%
		\def\@pytexTMP{\relax}%
	\else%
		\@pytexStack@popd{#1}%
		\def\@pytexTMP{\@pytexList@delete{#1}}%
	\fi%
	\@pytexTMP%
}



% #1 = queue name to create
\def\@pytexList@new#1{%
	\e\def\csname #1@size\endcsname{0}%
	\e\def\csname #1@push\endcsname##1{\@pytexList@push{#1}{##1}}%
	\e\def\csname #1@getim\endcsname##1{\@pytexList@getim{#1}{##1}}%
	\e\def\csname #1@delete\endcsname{\@pytexList@clear{#1}}%
}

% #1 = queue name, #2 = item
\def\@pytexList@push#1#2{%
	\edef\@pytexTMP{#1@\csname #1@size\endcsname}%
	\e\def\csname\@pytexTMP\endcsname{#2}%
	\e\edef\csname #1@size\endcsname{\the\numexpr\csname #1@size\endcsname+1}%
}

% #1 = queue name, #2 = index
\def\@pytexList@getim#1#2{\csname#1@#2\endcsname}

% #1 = queue name
\def\@pytexList@delete#1{% TODO: test
	\ifnum\csname #1@size\endcsname=0 %
		\e\let\csname #1@size\endcsname\undefined%
		\e\let\csname #1@push\endcsname\undefined%
		\e\let\csname #1@getim\endcsname\undefined%
		\e\let\csname #1@delete\endcsname\undefined%
	\else%
		\e\e\e\let\csname #1@\csname #1@size\endcsname\endscname\undefined%
		\e\edef\csname #1@size\endcsname{\the\numexpr\csname #1@size\endcsname-1}%
	\fi%
}


% DO NOT EDIT THIS FILE DIRECTLY
% This file was generated by 'gen_active.py'

\edef\@pytexTMP@prevTildaCatcode{\the\catcode126}
\edef\@pytexTMP@prevCaretCatcode{\the\catcode94}
\edef\@pytexTMP@prevMinusCatcode{\the\catcode45}
\edef\@pytexTMP@prevStarCatcode{\the\catcode42}
\catcode`~=13
\catcode`^=13
\catcode`*=13
\catcode`-=13
\let~\def
\let-\expandafter
\let^\catcode

\def*{\@pytexActive@a}
^97=13
-~*{a}
^97=11

\def*{\@pytexActive@b}
^98=13
-~*{b}
^98=11

\def*{\@pytexActive@c}
^99=13
-~*{c}
^99=11

\def*{\@pytexActive@d}
^100=13
-~*{d}
^100=11

\def*{\@pytexActive@e}
^101=13
-~*{e}
^101=11

\def*{\@pytexActive@f}
^102=13
-~*{f}
^102=11

\def*{\@pytexActive@g}
^103=13
-~*{g}
^103=11

\def*{\@pytexActive@h}
^104=13
-~*{h}
^104=11

\def*{\@pytexActive@i}
^105=13
-~*{i}
^105=11

\def*{\@pytexActive@j}
^106=13
-~*{j}
^106=11

\def*{\@pytexActive@k}
^107=13
-~*{k}
^107=11

\def*{\@pytexActive@l}
^108=13
-~*{l}
^108=11

\def*{\@pytexActive@m}
^109=13
-~*{m}
^109=11

\def*{\@pytexActive@n}
^110=13
-~*{n}
^110=11

\def*{\@pytexActive@o}
^111=13
-~*{o}
^111=11

\def*{\@pytexActive@p}
^112=13
-~*{p}
^112=11

\def*{\@pytexActive@q}
^113=13
-~*{q}
^113=11

\def*{\@pytexActive@r}
^114=13
-~*{r}
^114=11

\def*{\@pytexActive@s}
^115=13
-~*{s}
^115=11

\def*{\@pytexActive@t}
^116=13
-~*{t}
^116=11

\def*{\@pytexActive@u}
^117=13
-~*{u}
^117=11

\def*{\@pytexActive@v}
^118=13
-~*{v}
^118=11

\def*{\@pytexActive@w}
^119=13
-~*{w}
^119=11

\def*{\@pytexActive@x}
^120=13
-~*{x}
^120=11

\def*{\@pytexActive@y}
^121=13
-~*{y}
^121=11

\def*{\@pytexActive@z}
^122=13
-~*{z}
^122=11

\def*{\@pytexActive@A}
^65=13
-~*{A}
^65=11

\def*{\@pytexActive@B}
^66=13
-~*{B}
^66=11

\def*{\@pytexActive@C}
^67=13
-~*{C}
^67=11

\def*{\@pytexActive@D}
^68=13
-~*{D}
^68=11

\def*{\@pytexActive@E}
^69=13
-~*{E}
^69=11

\def*{\@pytexActive@F}
^70=13
-~*{F}
^70=11

\def*{\@pytexActive@G}
^71=13
-~*{G}
^71=11

\def*{\@pytexActive@H}
^72=13
-~*{H}
^72=11

\def*{\@pytexActive@I}
^73=13
-~*{I}
^73=11

\def*{\@pytexActive@J}
^74=13
-~*{J}
^74=11

\def*{\@pytexActive@K}
^75=13
-~*{K}
^75=11

\def*{\@pytexActive@L}
^76=13
-~*{L}
^76=11

\def*{\@pytexActive@M}
^77=13
-~*{M}
^77=11

\def*{\@pytexActive@N}
^78=13
-~*{N}
^78=11

\def*{\@pytexActive@O}
^79=13
-~*{O}
^79=11

\def*{\@pytexActive@P}
^80=13
-~*{P}
^80=11

\def*{\@pytexActive@Q}
^81=13
-~*{Q}
^81=11

\def*{\@pytexActive@R}
^82=13
-~*{R}
^82=11

\def*{\@pytexActive@S}
^83=13
-~*{S}
^83=11

\def*{\@pytexActive@T}
^84=13
-~*{T}
^84=11

\def*{\@pytexActive@U}
^85=13
-~*{U}
^85=11

\def*{\@pytexActive@V}
^86=13
-~*{V}
^86=11

\def*{\@pytexActive@W}
^87=13
-~*{W}
^87=11

\def*{\@pytexActive@X}
^88=13
-~*{X}
^88=11

\def*{\@pytexActive@Y}
^89=13
-~*{Y}
^89=11

\def*{\@pytexActive@Z}
^90=13
-~*{Z}
^90=11

\edef\@pytexTMP@prevcatcode{\the\catcode48}
\def\@pytexTMP@resetcatcode{\catcode48=\@pytexTMP@prevcatcode}
\catcode48=13
\def\@pytexActive@Zero{0}
\@pytexTMP@resetcatcode

\edef\@pytexTMP@prevcatcode{\the\catcode49}
\def\@pytexTMP@resetcatcode{\catcode49=\@pytexTMP@prevcatcode}
\catcode49=13
\def\@pytexActive@One{1}
\@pytexTMP@resetcatcode

\edef\@pytexTMP@prevcatcode{\the\catcode50}
\def\@pytexTMP@resetcatcode{\catcode50=\@pytexTMP@prevcatcode}
\catcode50=13
\def\@pytexActive@Two{2}
\@pytexTMP@resetcatcode

\edef\@pytexTMP@prevcatcode{\the\catcode51}
\def\@pytexTMP@resetcatcode{\catcode51=\@pytexTMP@prevcatcode}
\catcode51=13
\def\@pytexActive@Three{3}
\@pytexTMP@resetcatcode

\edef\@pytexTMP@prevcatcode{\the\catcode52}
\def\@pytexTMP@resetcatcode{\catcode52=\@pytexTMP@prevcatcode}
\catcode52=13
\def\@pytexActive@Four{4}
\@pytexTMP@resetcatcode

\edef\@pytexTMP@prevcatcode{\the\catcode53}
\def\@pytexTMP@resetcatcode{\catcode53=\@pytexTMP@prevcatcode}
\catcode53=13
\def\@pytexActive@Five{5}
\@pytexTMP@resetcatcode

\edef\@pytexTMP@prevcatcode{\the\catcode54}
\def\@pytexTMP@resetcatcode{\catcode54=\@pytexTMP@prevcatcode}
\catcode54=13
\def\@pytexActive@Six{6}
\@pytexTMP@resetcatcode

\edef\@pytexTMP@prevcatcode{\the\catcode55}
\def\@pytexTMP@resetcatcode{\catcode55=\@pytexTMP@prevcatcode}
\catcode55=13
\def\@pytexActive@Seven{7}
\@pytexTMP@resetcatcode

\edef\@pytexTMP@prevcatcode{\the\catcode56}
\def\@pytexTMP@resetcatcode{\catcode56=\@pytexTMP@prevcatcode}
\catcode56=13
\def\@pytexActive@Eight{8}
\@pytexTMP@resetcatcode

\edef\@pytexTMP@prevcatcode{\the\catcode57}
\def\@pytexTMP@resetcatcode{\catcode57=\@pytexTMP@prevcatcode}
\catcode57=13
\def\@pytexActive@Nine{9}
\@pytexTMP@resetcatcode

\edef\@pytexTMP@prevcatcode{\the\catcode32}
\def\@pytexTMP@resetcatcode{\catcode32=\@pytexTMP@prevcatcode}
\catcode32=13
\def\@pytexActive@Space{ }
\@pytexTMP@resetcatcode

\edef\@pytexTMP@prevcatcode{\the\catcode33}
\def\@pytexTMP@resetcatcode{\catcode33=\@pytexTMP@prevcatcode}
\catcode33=13
\def\@pytexActive@ExclamationMark{!}
\@pytexTMP@resetcatcode

\edef\@pytexTMP@prevcatcode{\the\catcode34}
\def\@pytexTMP@resetcatcode{\catcode34=\@pytexTMP@prevcatcode}
\catcode34=13
\def\@pytexActive@DoubleQuote{"}
\@pytexTMP@resetcatcode

\edef\@pytexTMP@prevcatcode{\the\catcode35}
\def\@pytexTMP@resetcatcode{\catcode35=\@pytexTMP@prevcatcode}
\catcode35=13
\def\@pytexActive@Hash{#}
\@pytexTMP@resetcatcode

\edef\@pytexTMP@prevcatcode{\the\catcode36}
\def\@pytexTMP@resetcatcode{\catcode36=\@pytexTMP@prevcatcode}
\catcode36=13
\def\@pytexActive@Dollar{$}
\@pytexTMP@resetcatcode

\edef\@pytexTMP@prevcatcode{\the\catcode37}
\def\@pytexTMP@resetcatcode{\catcode37=\@pytexTMP@prevcatcode}
\catcode37=13
\def\@pytexActive@Percent{%}
\@pytexTMP@resetcatcode

\edef\@pytexTMP@prevcatcode{\the\catcode38}
\def\@pytexTMP@resetcatcode{\catcode38=\@pytexTMP@prevcatcode}
\catcode38=13
\def\@pytexActive@Ampersand{&}
\@pytexTMP@resetcatcode

\edef\@pytexTMP@prevcatcode{\the\catcode39}
\def\@pytexTMP@resetcatcode{\catcode39=\@pytexTMP@prevcatcode}
\catcode39=13
\def\@pytexActive@SingleQuote{'}
\@pytexTMP@resetcatcode

\edef\@pytexTMP@prevcatcode{\the\catcode40}
\def\@pytexTMP@resetcatcode{\catcode40=\@pytexTMP@prevcatcode}
\catcode40=13
\def\@pytexActive@OpenParen{(}
\@pytexTMP@resetcatcode

\edef\@pytexTMP@prevcatcode{\the\catcode41}
\def\@pytexTMP@resetcatcode{\catcode41=\@pytexTMP@prevcatcode}
\catcode41=13
\def\@pytexActive@CloseParen{)}
\@pytexTMP@resetcatcode

\edef\@pytexTMP@prevcatcode{\the\catcode42}
\def\@pytexTMP@resetcatcode{\catcode42=\@pytexTMP@prevcatcode}
\catcode42=13
\def\@pytexActive@Asterisk{*}
\@pytexTMP@resetcatcode

\edef\@pytexTMP@prevcatcode{\the\catcode43}
\def\@pytexTMP@resetcatcode{\catcode43=\@pytexTMP@prevcatcode}
\catcode43=13
\def\@pytexActive@Plus{+}
\@pytexTMP@resetcatcode

\edef\@pytexTMP@prevcatcode{\the\catcode44}
\def\@pytexTMP@resetcatcode{\catcode44=\@pytexTMP@prevcatcode}
\catcode44=13
\def\@pytexActive@Comma{,}
\@pytexTMP@resetcatcode

\edef\@pytexTMP@prevcatcode{\the\catcode45}
\def\@pytexTMP@resetcatcode{\catcode45=\@pytexTMP@prevcatcode}
\catcode45=13
\def\@pytexActive@Minus{-}
\@pytexTMP@resetcatcode

\edef\@pytexTMP@prevcatcode{\the\catcode46}
\def\@pytexTMP@resetcatcode{\catcode46=\@pytexTMP@prevcatcode}
\catcode46=13
\def\@pytexActive@Dot{.}
\@pytexTMP@resetcatcode

\edef\@pytexTMP@prevcatcode{\the\catcode47}
\def\@pytexTMP@resetcatcode{\catcode47=\@pytexTMP@prevcatcode}
\catcode47=13
\def\@pytexActive@ForwardSlash{/}
\@pytexTMP@resetcatcode

\edef\@pytexTMP@prevcatcode{\the\catcode58}
\def\@pytexTMP@resetcatcode{\catcode58=\@pytexTMP@prevcatcode}
\catcode58=13
\def\@pytexActive@Colon{:}
\@pytexTMP@resetcatcode

\edef\@pytexTMP@prevcatcode{\the\catcode59}
\def\@pytexTMP@resetcatcode{\catcode59=\@pytexTMP@prevcatcode}
\catcode59=13
\def\@pytexActive@Semicolon{;}
\@pytexTMP@resetcatcode

\edef\@pytexTMP@prevcatcode{\the\catcode60}
\def\@pytexTMP@resetcatcode{\catcode60=\@pytexTMP@prevcatcode}
\catcode60=13
\def\@pytexActive@LessThan{<}
\@pytexTMP@resetcatcode

\edef\@pytexTMP@prevcatcode{\the\catcode61}
\def\@pytexTMP@resetcatcode{\catcode61=\@pytexTMP@prevcatcode}
\catcode61=13
\def\@pytexActive@Equals{=}
\@pytexTMP@resetcatcode

\edef\@pytexTMP@prevcatcode{\the\catcode62}
\def\@pytexTMP@resetcatcode{\catcode62=\@pytexTMP@prevcatcode}
\catcode62=13
\def\@pytexActive@GreaterThan{>}
\@pytexTMP@resetcatcode

\edef\@pytexTMP@prevcatcode{\the\catcode63}
\def\@pytexTMP@resetcatcode{\catcode63=\@pytexTMP@prevcatcode}
\catcode63=13
\def\@pytexActive@QuestionMark{?}
\@pytexTMP@resetcatcode

\edef\@pytexTMP@prevcatcode{\the\catcode91}
\def\@pytexTMP@resetcatcode{\catcode91=\@pytexTMP@prevcatcode}
\catcode91=13
\def\@pytexActive@OpenBracket{[}
\@pytexTMP@resetcatcode

\edef\@pytexTMP@prevcatcode{\the\catcode93}
\def\@pytexTMP@resetcatcode{\catcode93=\@pytexTMP@prevcatcode}
\catcode93=13
\def\@pytexActive@CloseBracket{]}
\@pytexTMP@resetcatcode

\edef\@pytexTMP@prevcatcode{\the\catcode94}
\def\@pytexTMP@resetcatcode{\catcode94=\@pytexTMP@prevcatcode}
\catcode94=13
\def\@pytexActive@Caret{^}
\@pytexTMP@resetcatcode

\edef\@pytexTMP@prevcatcode{\the\catcode95}
\def\@pytexTMP@resetcatcode{\catcode95=\@pytexTMP@prevcatcode}
\catcode95=13
\def\@pytexActive@Underscore{_}
\@pytexTMP@resetcatcode

\edef\@pytexTMP@prevcatcode{\the\catcode96}
\def\@pytexTMP@resetcatcode{\catcode96=\@pytexTMP@prevcatcode}
\catcode96=13
\def\@pytexActive@Backtick{`}
\@pytexTMP@resetcatcode

\edef\@pytexTMP@prevcatcode{\the\catcode124}
\def\@pytexTMP@resetcatcode{\catcode124=\@pytexTMP@prevcatcode}
\catcode124=13
\def\@pytexActive@Pipe{|}
\@pytexTMP@resetcatcode

\edef\@pytexTMP@prevcatcode{\the\catcode126}
\def\@pytexTMP@resetcatcode{\catcode126=\@pytexTMP@prevcatcode}
\catcode126=13
\def\@pytexActive@Tilde{~}
\@pytexTMP@resetcatcode

\edef\@pytexTMP@prevcatcode{\the\catcode9}
\def\@pytexTMP@resetcatcode{\catcode9=\@pytexTMP@prevcatcode}
\catcode9=13
\def\@pytexActive@Tab{	}
\@pytexTMP@resetcatcode

\catcode`~=1
\catcode`-=2
\edef\@pytexTMP@prevcatcode{\the\catcode123}
\def\@pytexTMP@resetcatcode{\catcode123=\@pytexTMP@prevcatcode}
\catcode123=13
\def\@pytexActive@OpenBrace~{-
\@pytexTMP@resetcatcode

\edef\@pytexTMP@prevcatcode{\the\catcode125}
\def\@pytexTMP@resetcatcode{\catcode125=\@pytexTMP@prevcatcode}
\catcode125=13
\def\@pytexActive@CloseBrace~}-
\@pytexTMP@resetcatcode

\catcode`~=13
\let~\let
\edef\@pytexTMP@prevcatcode{\the\catcode92}
\def\@pytexTMP@resetcatcode{\catcode92=\@pytexTMP@prevcatcode}
\catcode`~=0
\catcode92=13
~def~@pytexActive@Backslash{\}
~@pytexTMP@resetcatcode

\catcode126=13
\edef\@pytexTMP@prevcatcode{\the\catcode13}
\def\@pytexTMP@resetcatcode{\catcode13=\@pytexTMP@prevcatcode}%
\catcode13=13%
\def\@pytexActive@Newline{
}%
\catcode13=5%

\catcode`~=\@pytexTMP@prevTildaCatcode
\catcode`^=\@pytexTMP@prevCaretCatcode
\catcode`-=\@pytexTMP@prevMinusCatcode
\catcode`*=\@pytexTMP@prevStarCatcode

\let\@pytexTMP@macro\undefined
\let\@pytexTMP@prevcatcode\undefined
\let\@pytexTMP@prevTildaCatcode\undefined
\let\@pytexTMP@prevCaretCatcode\undefined
\let\@pytexTMP@prevMinusCatcode\undefined
\let\@pytexTMP@prevStarCatcode\undefined
\let\@pytexTMP@resetcatcode\undefined


\def\@pytexDefAll{%
	\expandafter\let\expandafter\@pytexPrevmacro@a\@pytexActive@a%
	\expandafter\def\@pytexActive@a{\@pytexChar@a}%
	\expandafter\let\expandafter\@pytexPrevmacro@b\@pytexActive@b%
	\expandafter\def\@pytexActive@b{\@pytexChar@b}%
	\expandafter\let\expandafter\@pytexPrevmacro@c\@pytexActive@c%
	\expandafter\def\@pytexActive@c{\@pytexChar@c}%
	\expandafter\let\expandafter\@pytexPrevmacro@d\@pytexActive@d%
	\expandafter\def\@pytexActive@d{\@pytexChar@d}%
	\expandafter\let\expandafter\@pytexPrevmacro@e\@pytexActive@e%
	\expandafter\def\@pytexActive@e{\@pytexChar@e}%
	\expandafter\let\expandafter\@pytexPrevmacro@f\@pytexActive@f%
	\expandafter\def\@pytexActive@f{\@pytexChar@f}%
	\expandafter\let\expandafter\@pytexPrevmacro@g\@pytexActive@g%
	\expandafter\def\@pytexActive@g{\@pytexChar@g}%
	\expandafter\let\expandafter\@pytexPrevmacro@h\@pytexActive@h%
	\expandafter\def\@pytexActive@h{\@pytexChar@h}%
	\expandafter\let\expandafter\@pytexPrevmacro@i\@pytexActive@i%
	\expandafter\def\@pytexActive@i{\@pytexChar@i}%
	\expandafter\let\expandafter\@pytexPrevmacro@j\@pytexActive@j%
	\expandafter\def\@pytexActive@j{\@pytexChar@j}%
	\expandafter\let\expandafter\@pytexPrevmacro@k\@pytexActive@k%
	\expandafter\def\@pytexActive@k{\@pytexChar@k}%
	\expandafter\let\expandafter\@pytexPrevmacro@l\@pytexActive@l%
	\expandafter\def\@pytexActive@l{\@pytexChar@l}%
	\expandafter\let\expandafter\@pytexPrevmacro@m\@pytexActive@m%
	\expandafter\def\@pytexActive@m{\@pytexChar@m}%
	\expandafter\let\expandafter\@pytexPrevmacro@n\@pytexActive@n%
	\expandafter\def\@pytexActive@n{\@pytexChar@n}%
	\expandafter\let\expandafter\@pytexPrevmacro@o\@pytexActive@o%
	\expandafter\def\@pytexActive@o{\@pytexChar@o}%
	\expandafter\let\expandafter\@pytexPrevmacro@p\@pytexActive@p%
	\expandafter\def\@pytexActive@p{\@pytexChar@p}%
	\expandafter\let\expandafter\@pytexPrevmacro@q\@pytexActive@q%
	\expandafter\def\@pytexActive@q{\@pytexChar@q}%
	\expandafter\let\expandafter\@pytexPrevmacro@r\@pytexActive@r%
	\expandafter\def\@pytexActive@r{\@pytexChar@r}%
	\expandafter\let\expandafter\@pytexPrevmacro@s\@pytexActive@s%
	\expandafter\def\@pytexActive@s{\@pytexChar@s}%
	\expandafter\let\expandafter\@pytexPrevmacro@t\@pytexActive@t%
	\expandafter\def\@pytexActive@t{\@pytexChar@t}%
	\expandafter\let\expandafter\@pytexPrevmacro@u\@pytexActive@u%
	\expandafter\def\@pytexActive@u{\@pytexChar@u}%
	\expandafter\let\expandafter\@pytexPrevmacro@v\@pytexActive@v%
	\expandafter\def\@pytexActive@v{\@pytexChar@v}%
	\expandafter\let\expandafter\@pytexPrevmacro@w\@pytexActive@w%
	\expandafter\def\@pytexActive@w{\@pytexChar@w}%
	\expandafter\let\expandafter\@pytexPrevmacro@x\@pytexActive@x%
	\expandafter\def\@pytexActive@x{\@pytexChar@x}%
	\expandafter\let\expandafter\@pytexPrevmacro@y\@pytexActive@y%
	\expandafter\def\@pytexActive@y{\@pytexChar@y}%
	\expandafter\let\expandafter\@pytexPrevmacro@z\@pytexActive@z%
	\expandafter\def\@pytexActive@z{\@pytexChar@z}%
	\expandafter\let\expandafter\@pytexPrevmacro@A\@pytexActive@A%
	\expandafter\def\@pytexActive@A{\@pytexChar@A}%
	\expandafter\let\expandafter\@pytexPrevmacro@B\@pytexActive@B%
	\expandafter\def\@pytexActive@B{\@pytexChar@B}%
	\expandafter\let\expandafter\@pytexPrevmacro@C\@pytexActive@C%
	\expandafter\def\@pytexActive@C{\@pytexChar@C}%
	\expandafter\let\expandafter\@pytexPrevmacro@D\@pytexActive@D%
	\expandafter\def\@pytexActive@D{\@pytexChar@D}%
	\expandafter\let\expandafter\@pytexPrevmacro@E\@pytexActive@E%
	\expandafter\def\@pytexActive@E{\@pytexChar@E}%
	\expandafter\let\expandafter\@pytexPrevmacro@F\@pytexActive@F%
	\expandafter\def\@pytexActive@F{\@pytexChar@F}%
	\expandafter\let\expandafter\@pytexPrevmacro@G\@pytexActive@G%
	\expandafter\def\@pytexActive@G{\@pytexChar@G}%
	\expandafter\let\expandafter\@pytexPrevmacro@H\@pytexActive@H%
	\expandafter\def\@pytexActive@H{\@pytexChar@H}%
	\expandafter\let\expandafter\@pytexPrevmacro@I\@pytexActive@I%
	\expandafter\def\@pytexActive@I{\@pytexChar@I}%
	\expandafter\let\expandafter\@pytexPrevmacro@J\@pytexActive@J%
	\expandafter\def\@pytexActive@J{\@pytexChar@J}%
	\expandafter\let\expandafter\@pytexPrevmacro@K\@pytexActive@K%
	\expandafter\def\@pytexActive@K{\@pytexChar@K}%
	\expandafter\let\expandafter\@pytexPrevmacro@L\@pytexActive@L%
	\expandafter\def\@pytexActive@L{\@pytexChar@L}%
	\expandafter\let\expandafter\@pytexPrevmacro@M\@pytexActive@M%
	\expandafter\def\@pytexActive@M{\@pytexChar@M}%
	\expandafter\let\expandafter\@pytexPrevmacro@N\@pytexActive@N%
	\expandafter\def\@pytexActive@N{\@pytexChar@N}%
	\expandafter\let\expandafter\@pytexPrevmacro@O\@pytexActive@O%
	\expandafter\def\@pytexActive@O{\@pytexChar@O}%
	\expandafter\let\expandafter\@pytexPrevmacro@P\@pytexActive@P%
	\expandafter\def\@pytexActive@P{\@pytexChar@P}%
	\expandafter\let\expandafter\@pytexPrevmacro@Q\@pytexActive@Q%
	\expandafter\def\@pytexActive@Q{\@pytexChar@Q}%
	\expandafter\let\expandafter\@pytexPrevmacro@R\@pytexActive@R%
	\expandafter\def\@pytexActive@R{\@pytexChar@R}%
	\expandafter\let\expandafter\@pytexPrevmacro@S\@pytexActive@S%
	\expandafter\def\@pytexActive@S{\@pytexChar@S}%
	\expandafter\let\expandafter\@pytexPrevmacro@T\@pytexActive@T%
	\expandafter\def\@pytexActive@T{\@pytexChar@T}%
	\expandafter\let\expandafter\@pytexPrevmacro@U\@pytexActive@U%
	\expandafter\def\@pytexActive@U{\@pytexChar@U}%
	\expandafter\let\expandafter\@pytexPrevmacro@V\@pytexActive@V%
	\expandafter\def\@pytexActive@V{\@pytexChar@V}%
	\expandafter\let\expandafter\@pytexPrevmacro@W\@pytexActive@W%
	\expandafter\def\@pytexActive@W{\@pytexChar@W}%
	\expandafter\let\expandafter\@pytexPrevmacro@X\@pytexActive@X%
	\expandafter\def\@pytexActive@X{\@pytexChar@X}%
	\expandafter\let\expandafter\@pytexPrevmacro@Y\@pytexActive@Y%
	\expandafter\def\@pytexActive@Y{\@pytexChar@Y}%
	\expandafter\let\expandafter\@pytexPrevmacro@Z\@pytexActive@Z%
	\expandafter\def\@pytexActive@Z{\@pytexChar@Z}%
	\expandafter\let\expandafter\@pytexPrevmacro@Zero\@pytexActive@Zero%
	\expandafter\def\@pytexActive@Zero{\@pytexChar@Zero}%
	\expandafter\let\expandafter\@pytexPrevmacro@One\@pytexActive@One%
	\expandafter\def\@pytexActive@One{\@pytexChar@One}%
	\expandafter\let\expandafter\@pytexPrevmacro@Two\@pytexActive@Two%
	\expandafter\def\@pytexActive@Two{\@pytexChar@Two}%
	\expandafter\let\expandafter\@pytexPrevmacro@Three\@pytexActive@Three%
	\expandafter\def\@pytexActive@Three{\@pytexChar@Three}%
	\expandafter\let\expandafter\@pytexPrevmacro@Four\@pytexActive@Four%
	\expandafter\def\@pytexActive@Four{\@pytexChar@Four}%
	\expandafter\let\expandafter\@pytexPrevmacro@Five\@pytexActive@Five%
	\expandafter\def\@pytexActive@Five{\@pytexChar@Five}%
	\expandafter\let\expandafter\@pytexPrevmacro@Six\@pytexActive@Six%
	\expandafter\def\@pytexActive@Six{\@pytexChar@Six}%
	\expandafter\let\expandafter\@pytexPrevmacro@Seven\@pytexActive@Seven%
	\expandafter\def\@pytexActive@Seven{\@pytexChar@Seven}%
	\expandafter\let\expandafter\@pytexPrevmacro@Eight\@pytexActive@Eight%
	\expandafter\def\@pytexActive@Eight{\@pytexChar@Eight}%
	\expandafter\let\expandafter\@pytexPrevmacro@Nine\@pytexActive@Nine%
	\expandafter\def\@pytexActive@Nine{\@pytexChar@Nine}%
	\expandafter\let\expandafter\@pytexPrevmacro@Space\@pytexActive@Space%
	\expandafter\def\@pytexActive@Space{\@pytexChar@Space}%
	\expandafter\let\expandafter\@pytexPrevmacro@ExclamationMark\@pytexActive@ExclamationMark%
	\expandafter\def\@pytexActive@ExclamationMark{\@pytexChar@ExclamationMark}%
	\expandafter\let\expandafter\@pytexPrevmacro@DoubleQuote\@pytexActive@DoubleQuote%
	\expandafter\def\@pytexActive@DoubleQuote{\@pytexChar@DoubleQuote}%
	\expandafter\let\expandafter\@pytexPrevmacro@Hash\@pytexActive@Hash%
	\expandafter\def\@pytexActive@Hash{\@pytexChar@Hash}%
	\expandafter\let\expandafter\@pytexPrevmacro@Dollar\@pytexActive@Dollar%
	\expandafter\def\@pytexActive@Dollar{\@pytexChar@Dollar}%
	\expandafter\let\expandafter\@pytexPrevmacro@Percent\@pytexActive@Percent%
	\expandafter\def\@pytexActive@Percent{\@pytexChar@Percent}%
	\expandafter\let\expandafter\@pytexPrevmacro@Ampersand\@pytexActive@Ampersand%
	\expandafter\def\@pytexActive@Ampersand{\@pytexChar@Ampersand}%
	\expandafter\let\expandafter\@pytexPrevmacro@SingleQuote\@pytexActive@SingleQuote%
	\expandafter\def\@pytexActive@SingleQuote{\@pytexChar@SingleQuote}%
	\expandafter\let\expandafter\@pytexPrevmacro@OpenParen\@pytexActive@OpenParen%
	\expandafter\def\@pytexActive@OpenParen{\@pytexChar@OpenParen}%
	\expandafter\let\expandafter\@pytexPrevmacro@CloseParen\@pytexActive@CloseParen%
	\expandafter\def\@pytexActive@CloseParen{\@pytexChar@CloseParen}%
	\expandafter\let\expandafter\@pytexPrevmacro@Asterisk\@pytexActive@Asterisk%
	\expandafter\def\@pytexActive@Asterisk{\@pytexChar@Asterisk}%
	\expandafter\let\expandafter\@pytexPrevmacro@Plus\@pytexActive@Plus%
	\expandafter\def\@pytexActive@Plus{\@pytexChar@Plus}%
	\expandafter\let\expandafter\@pytexPrevmacro@Comma\@pytexActive@Comma%
	\expandafter\def\@pytexActive@Comma{\@pytexChar@Comma}%
	\expandafter\let\expandafter\@pytexPrevmacro@Minus\@pytexActive@Minus%
	\expandafter\def\@pytexActive@Minus{\@pytexChar@Minus}%
	\expandafter\let\expandafter\@pytexPrevmacro@Dot\@pytexActive@Dot%
	\expandafter\def\@pytexActive@Dot{\@pytexChar@Dot}%
	\expandafter\let\expandafter\@pytexPrevmacro@ForwardSlash\@pytexActive@ForwardSlash%
	\expandafter\def\@pytexActive@ForwardSlash{\@pytexChar@ForwardSlash}%
	\expandafter\let\expandafter\@pytexPrevmacro@Colon\@pytexActive@Colon%
	\expandafter\def\@pytexActive@Colon{\@pytexChar@Colon}%
	\expandafter\let\expandafter\@pytexPrevmacro@Semicolon\@pytexActive@Semicolon%
	\expandafter\def\@pytexActive@Semicolon{\@pytexChar@Semicolon}%
	\expandafter\let\expandafter\@pytexPrevmacro@LessThan\@pytexActive@LessThan%
	\expandafter\def\@pytexActive@LessThan{\@pytexChar@LessThan}%
	\expandafter\let\expandafter\@pytexPrevmacro@Equals\@pytexActive@Equals%
	\expandafter\def\@pytexActive@Equals{\@pytexChar@Equals}%
	\expandafter\let\expandafter\@pytexPrevmacro@GreaterThan\@pytexActive@GreaterThan%
	\expandafter\def\@pytexActive@GreaterThan{\@pytexChar@GreaterThan}%
	\expandafter\let\expandafter\@pytexPrevmacro@QuestionMark\@pytexActive@QuestionMark%
	\expandafter\def\@pytexActive@QuestionMark{\@pytexChar@QuestionMark}%
	\expandafter\let\expandafter\@pytexPrevmacro@OpenBracket\@pytexActive@OpenBracket%
	\expandafter\def\@pytexActive@OpenBracket{\@pytexChar@OpenBracket}%
	\expandafter\let\expandafter\@pytexPrevmacro@CloseBracket\@pytexActive@CloseBracket%
	\expandafter\def\@pytexActive@CloseBracket{\@pytexChar@CloseBracket}%
	\expandafter\let\expandafter\@pytexPrevmacro@Caret\@pytexActive@Caret%
	\expandafter\def\@pytexActive@Caret{\@pytexChar@Caret}%
	\expandafter\let\expandafter\@pytexPrevmacro@Underscore\@pytexActive@Underscore%
	\expandafter\def\@pytexActive@Underscore{\@pytexChar@Underscore}%
	\expandafter\let\expandafter\@pytexPrevmacro@Backtick\@pytexActive@Backtick%
	\expandafter\def\@pytexActive@Backtick{\@pytexChar@Backtick}%
	\expandafter\let\expandafter\@pytexPrevmacro@OpenBrace\@pytexActive@OpenBrace%
	\expandafter\def\@pytexActive@OpenBrace{\@pytexChar@OpenBrace}%
	\expandafter\let\expandafter\@pytexPrevmacro@Pipe\@pytexActive@Pipe%
	\expandafter\def\@pytexActive@Pipe{\@pytexChar@Pipe}%
	\expandafter\let\expandafter\@pytexPrevmacro@CloseBrace\@pytexActive@CloseBrace%
	\expandafter\def\@pytexActive@CloseBrace{\@pytexChar@CloseBrace}%
	\expandafter\let\expandafter\@pytexPrevmacro@Tilde\@pytexActive@Tilde%
	\expandafter\def\@pytexActive@Tilde{\@pytexChar@Tilde}%
	\expandafter\let\expandafter\@pytexPrevmacro@Backslash\@pytexActive@Backslash%
	\expandafter\def\@pytexActive@Backslash{\@pytexChar@Backslash}%
	\expandafter\let\expandafter\@pytexPrevmacro@Tab\@pytexActive@Tab%
	\expandafter\def\@pytexActive@Tab{\@pytexChar@Tab}%
	\expandafter\let\expandafter\@pytexPrevmacro@Newline\@pytexActive@Newline%
	\expandafter\def\@pytexActive@Newline{\@pytexChar@Newline}%
}
\def\@pytexRelaxAll{%
	\expandafter\let\@pytexActive@a\relax%
	\expandafter\let\@pytexActive@b\relax%
	\expandafter\let\@pytexActive@c\relax%
	\expandafter\let\@pytexActive@d\relax%
	\expandafter\let\@pytexActive@e\relax%
	\expandafter\let\@pytexActive@f\relax%
	\expandafter\let\@pytexActive@g\relax%
	\expandafter\let\@pytexActive@h\relax%
	\expandafter\let\@pytexActive@i\relax%
	\expandafter\let\@pytexActive@j\relax%
	\expandafter\let\@pytexActive@k\relax%
	\expandafter\let\@pytexActive@l\relax%
	\expandafter\let\@pytexActive@m\relax%
	\expandafter\let\@pytexActive@n\relax%
	\expandafter\let\@pytexActive@o\relax%
	\expandafter\let\@pytexActive@p\relax%
	\expandafter\let\@pytexActive@q\relax%
	\expandafter\let\@pytexActive@r\relax%
	\expandafter\let\@pytexActive@s\relax%
	\expandafter\let\@pytexActive@t\relax%
	\expandafter\let\@pytexActive@u\relax%
	\expandafter\let\@pytexActive@v\relax%
	\expandafter\let\@pytexActive@w\relax%
	\expandafter\let\@pytexActive@x\relax%
	\expandafter\let\@pytexActive@y\relax%
	\expandafter\let\@pytexActive@z\relax%
	\expandafter\let\@pytexActive@A\relax%
	\expandafter\let\@pytexActive@B\relax%
	\expandafter\let\@pytexActive@C\relax%
	\expandafter\let\@pytexActive@D\relax%
	\expandafter\let\@pytexActive@E\relax%
	\expandafter\let\@pytexActive@F\relax%
	\expandafter\let\@pytexActive@G\relax%
	\expandafter\let\@pytexActive@H\relax%
	\expandafter\let\@pytexActive@I\relax%
	\expandafter\let\@pytexActive@J\relax%
	\expandafter\let\@pytexActive@K\relax%
	\expandafter\let\@pytexActive@L\relax%
	\expandafter\let\@pytexActive@M\relax%
	\expandafter\let\@pytexActive@N\relax%
	\expandafter\let\@pytexActive@O\relax%
	\expandafter\let\@pytexActive@P\relax%
	\expandafter\let\@pytexActive@Q\relax%
	\expandafter\let\@pytexActive@R\relax%
	\expandafter\let\@pytexActive@S\relax%
	\expandafter\let\@pytexActive@T\relax%
	\expandafter\let\@pytexActive@U\relax%
	\expandafter\let\@pytexActive@V\relax%
	\expandafter\let\@pytexActive@W\relax%
	\expandafter\let\@pytexActive@X\relax%
	\expandafter\let\@pytexActive@Y\relax%
	\expandafter\let\@pytexActive@Z\relax%
	\expandafter\let\@pytexActive@Zero\relax%
	\expandafter\let\@pytexActive@One\relax%
	\expandafter\let\@pytexActive@Two\relax%
	\expandafter\let\@pytexActive@Three\relax%
	\expandafter\let\@pytexActive@Four\relax%
	\expandafter\let\@pytexActive@Five\relax%
	\expandafter\let\@pytexActive@Six\relax%
	\expandafter\let\@pytexActive@Seven\relax%
	\expandafter\let\@pytexActive@Eight\relax%
	\expandafter\let\@pytexActive@Nine\relax%
	\expandafter\let\@pytexActive@Space\relax%
	\expandafter\let\@pytexActive@ExclamationMark\relax%
	\expandafter\let\@pytexActive@DoubleQuote\relax%
	\expandafter\let\@pytexActive@Hash\relax%
	\expandafter\let\@pytexActive@Dollar\relax%
	\expandafter\let\@pytexActive@Percent\relax%
	\expandafter\let\@pytexActive@Ampersand\relax%
	\expandafter\let\@pytexActive@SingleQuote\relax%
	\expandafter\let\@pytexActive@OpenParen\relax%
	\expandafter\let\@pytexActive@CloseParen\relax%
	\expandafter\let\@pytexActive@Asterisk\relax%
	\expandafter\let\@pytexActive@Plus\relax%
	\expandafter\let\@pytexActive@Comma\relax%
	\expandafter\let\@pytexActive@Minus\relax%
	\expandafter\let\@pytexActive@Dot\relax%
	\expandafter\let\@pytexActive@ForwardSlash\relax%
	\expandafter\let\@pytexActive@Colon\relax%
	\expandafter\let\@pytexActive@Semicolon\relax%
	\expandafter\let\@pytexActive@LessThan\relax%
	\expandafter\let\@pytexActive@Equals\relax%
	\expandafter\let\@pytexActive@GreaterThan\relax%
	\expandafter\let\@pytexActive@QuestionMark\relax%
	\expandafter\let\@pytexActive@OpenBracket\relax%
	\expandafter\let\@pytexActive@CloseBracket\relax%
	\expandafter\let\@pytexActive@Caret\relax%
	\expandafter\let\@pytexActive@Underscore\relax%
	\expandafter\let\@pytexActive@Backtick\relax%
	\expandafter\let\@pytexActive@OpenBrace\relax%
	\expandafter\let\@pytexActive@Pipe\relax%
	\expandafter\let\@pytexActive@CloseBrace\relax%
	\expandafter\let\@pytexActive@Tilde\relax%
	\expandafter\let\@pytexActive@Backslash\relax%
	\expandafter\let\@pytexActive@Tab\relax%
	\expandafter\let\@pytexActive@Newline\relax%
}
\def\@pytexResetAll{%
	\expandafter\let\@pytexActive@a\@pytexPrevmacro@a%
	\let\@pytexPrevmacro@a\undefined%
	\expandafter\let\@pytexActive@b\@pytexPrevmacro@b%
	\let\@pytexPrevmacro@b\undefined%
	\expandafter\let\@pytexActive@c\@pytexPrevmacro@c%
	\let\@pytexPrevmacro@c\undefined%
	\expandafter\let\@pytexActive@d\@pytexPrevmacro@d%
	\let\@pytexPrevmacro@d\undefined%
	\expandafter\let\@pytexActive@e\@pytexPrevmacro@e%
	\let\@pytexPrevmacro@e\undefined%
	\expandafter\let\@pytexActive@f\@pytexPrevmacro@f%
	\let\@pytexPrevmacro@f\undefined%
	\expandafter\let\@pytexActive@g\@pytexPrevmacro@g%
	\let\@pytexPrevmacro@g\undefined%
	\expandafter\let\@pytexActive@h\@pytexPrevmacro@h%
	\let\@pytexPrevmacro@h\undefined%
	\expandafter\let\@pytexActive@i\@pytexPrevmacro@i%
	\let\@pytexPrevmacro@i\undefined%
	\expandafter\let\@pytexActive@j\@pytexPrevmacro@j%
	\let\@pytexPrevmacro@j\undefined%
	\expandafter\let\@pytexActive@k\@pytexPrevmacro@k%
	\let\@pytexPrevmacro@k\undefined%
	\expandafter\let\@pytexActive@l\@pytexPrevmacro@l%
	\let\@pytexPrevmacro@l\undefined%
	\expandafter\let\@pytexActive@m\@pytexPrevmacro@m%
	\let\@pytexPrevmacro@m\undefined%
	\expandafter\let\@pytexActive@n\@pytexPrevmacro@n%
	\let\@pytexPrevmacro@n\undefined%
	\expandafter\let\@pytexActive@o\@pytexPrevmacro@o%
	\let\@pytexPrevmacro@o\undefined%
	\expandafter\let\@pytexActive@p\@pytexPrevmacro@p%
	\let\@pytexPrevmacro@p\undefined%
	\expandafter\let\@pytexActive@q\@pytexPrevmacro@q%
	\let\@pytexPrevmacro@q\undefined%
	\expandafter\let\@pytexActive@r\@pytexPrevmacro@r%
	\let\@pytexPrevmacro@r\undefined%
	\expandafter\let\@pytexActive@s\@pytexPrevmacro@s%
	\let\@pytexPrevmacro@s\undefined%
	\expandafter\let\@pytexActive@t\@pytexPrevmacro@t%
	\let\@pytexPrevmacro@t\undefined%
	\expandafter\let\@pytexActive@u\@pytexPrevmacro@u%
	\let\@pytexPrevmacro@u\undefined%
	\expandafter\let\@pytexActive@v\@pytexPrevmacro@v%
	\let\@pytexPrevmacro@v\undefined%
	\expandafter\let\@pytexActive@w\@pytexPrevmacro@w%
	\let\@pytexPrevmacro@w\undefined%
	\expandafter\let\@pytexActive@x\@pytexPrevmacro@x%
	\let\@pytexPrevmacro@x\undefined%
	\expandafter\let\@pytexActive@y\@pytexPrevmacro@y%
	\let\@pytexPrevmacro@y\undefined%
	\expandafter\let\@pytexActive@z\@pytexPrevmacro@z%
	\let\@pytexPrevmacro@z\undefined%
	\expandafter\let\@pytexActive@A\@pytexPrevmacro@A%
	\let\@pytexPrevmacro@A\undefined%
	\expandafter\let\@pytexActive@B\@pytexPrevmacro@B%
	\let\@pytexPrevmacro@B\undefined%
	\expandafter\let\@pytexActive@C\@pytexPrevmacro@C%
	\let\@pytexPrevmacro@C\undefined%
	\expandafter\let\@pytexActive@D\@pytexPrevmacro@D%
	\let\@pytexPrevmacro@D\undefined%
	\expandafter\let\@pytexActive@E\@pytexPrevmacro@E%
	\let\@pytexPrevmacro@E\undefined%
	\expandafter\let\@pytexActive@F\@pytexPrevmacro@F%
	\let\@pytexPrevmacro@F\undefined%
	\expandafter\let\@pytexActive@G\@pytexPrevmacro@G%
	\let\@pytexPrevmacro@G\undefined%
	\expandafter\let\@pytexActive@H\@pytexPrevmacro@H%
	\let\@pytexPrevmacro@H\undefined%
	\expandafter\let\@pytexActive@I\@pytexPrevmacro@I%
	\let\@pytexPrevmacro@I\undefined%
	\expandafter\let\@pytexActive@J\@pytexPrevmacro@J%
	\let\@pytexPrevmacro@J\undefined%
	\expandafter\let\@pytexActive@K\@pytexPrevmacro@K%
	\let\@pytexPrevmacro@K\undefined%
	\expandafter\let\@pytexActive@L\@pytexPrevmacro@L%
	\let\@pytexPrevmacro@L\undefined%
	\expandafter\let\@pytexActive@M\@pytexPrevmacro@M%
	\let\@pytexPrevmacro@M\undefined%
	\expandafter\let\@pytexActive@N\@pytexPrevmacro@N%
	\let\@pytexPrevmacro@N\undefined%
	\expandafter\let\@pytexActive@O\@pytexPrevmacro@O%
	\let\@pytexPrevmacro@O\undefined%
	\expandafter\let\@pytexActive@P\@pytexPrevmacro@P%
	\let\@pytexPrevmacro@P\undefined%
	\expandafter\let\@pytexActive@Q\@pytexPrevmacro@Q%
	\let\@pytexPrevmacro@Q\undefined%
	\expandafter\let\@pytexActive@R\@pytexPrevmacro@R%
	\let\@pytexPrevmacro@R\undefined%
	\expandafter\let\@pytexActive@S\@pytexPrevmacro@S%
	\let\@pytexPrevmacro@S\undefined%
	\expandafter\let\@pytexActive@T\@pytexPrevmacro@T%
	\let\@pytexPrevmacro@T\undefined%
	\expandafter\let\@pytexActive@U\@pytexPrevmacro@U%
	\let\@pytexPrevmacro@U\undefined%
	\expandafter\let\@pytexActive@V\@pytexPrevmacro@V%
	\let\@pytexPrevmacro@V\undefined%
	\expandafter\let\@pytexActive@W\@pytexPrevmacro@W%
	\let\@pytexPrevmacro@W\undefined%
	\expandafter\let\@pytexActive@X\@pytexPrevmacro@X%
	\let\@pytexPrevmacro@X\undefined%
	\expandafter\let\@pytexActive@Y\@pytexPrevmacro@Y%
	\let\@pytexPrevmacro@Y\undefined%
	\expandafter\let\@pytexActive@Z\@pytexPrevmacro@Z%
	\let\@pytexPrevmacro@Z\undefined%
	\expandafter\let\@pytexActive@Zero\@pytexPrevmacro@Zero%
	\let\@pytexPrevmacro@Zero\undefined%
	\expandafter\let\@pytexActive@One\@pytexPrevmacro@One%
	\let\@pytexPrevmacro@One\undefined%
	\expandafter\let\@pytexActive@Two\@pytexPrevmacro@Two%
	\let\@pytexPrevmacro@Two\undefined%
	\expandafter\let\@pytexActive@Three\@pytexPrevmacro@Three%
	\let\@pytexPrevmacro@Three\undefined%
	\expandafter\let\@pytexActive@Four\@pytexPrevmacro@Four%
	\let\@pytexPrevmacro@Four\undefined%
	\expandafter\let\@pytexActive@Five\@pytexPrevmacro@Five%
	\let\@pytexPrevmacro@Five\undefined%
	\expandafter\let\@pytexActive@Six\@pytexPrevmacro@Six%
	\let\@pytexPrevmacro@Six\undefined%
	\expandafter\let\@pytexActive@Seven\@pytexPrevmacro@Seven%
	\let\@pytexPrevmacro@Seven\undefined%
	\expandafter\let\@pytexActive@Eight\@pytexPrevmacro@Eight%
	\let\@pytexPrevmacro@Eight\undefined%
	\expandafter\let\@pytexActive@Nine\@pytexPrevmacro@Nine%
	\let\@pytexPrevmacro@Nine\undefined%
	\expandafter\let\@pytexActive@Space\@pytexPrevmacro@Space%
	\let\@pytexPrevmacro@Space\undefined%
	\expandafter\let\@pytexActive@ExclamationMark\@pytexPrevmacro@ExclamationMark%
	\let\@pytexPrevmacro@ExclamationMark\undefined%
	\expandafter\let\@pytexActive@DoubleQuote\@pytexPrevmacro@DoubleQuote%
	\let\@pytexPrevmacro@DoubleQuote\undefined%
	\expandafter\let\@pytexActive@Hash\@pytexPrevmacro@Hash%
	\let\@pytexPrevmacro@Hash\undefined%
	\expandafter\let\@pytexActive@Dollar\@pytexPrevmacro@Dollar%
	\let\@pytexPrevmacro@Dollar\undefined%
	\expandafter\let\@pytexActive@Percent\@pytexPrevmacro@Percent%
	\let\@pytexPrevmacro@Percent\undefined%
	\expandafter\let\@pytexActive@Ampersand\@pytexPrevmacro@Ampersand%
	\let\@pytexPrevmacro@Ampersand\undefined%
	\expandafter\let\@pytexActive@SingleQuote\@pytexPrevmacro@SingleQuote%
	\let\@pytexPrevmacro@SingleQuote\undefined%
	\expandafter\let\@pytexActive@OpenParen\@pytexPrevmacro@OpenParen%
	\let\@pytexPrevmacro@OpenParen\undefined%
	\expandafter\let\@pytexActive@CloseParen\@pytexPrevmacro@CloseParen%
	\let\@pytexPrevmacro@CloseParen\undefined%
	\expandafter\let\@pytexActive@Asterisk\@pytexPrevmacro@Asterisk%
	\let\@pytexPrevmacro@Asterisk\undefined%
	\expandafter\let\@pytexActive@Plus\@pytexPrevmacro@Plus%
	\let\@pytexPrevmacro@Plus\undefined%
	\expandafter\let\@pytexActive@Comma\@pytexPrevmacro@Comma%
	\let\@pytexPrevmacro@Comma\undefined%
	\expandafter\let\@pytexActive@Minus\@pytexPrevmacro@Minus%
	\let\@pytexPrevmacro@Minus\undefined%
	\expandafter\let\@pytexActive@Dot\@pytexPrevmacro@Dot%
	\let\@pytexPrevmacro@Dot\undefined%
	\expandafter\let\@pytexActive@ForwardSlash\@pytexPrevmacro@ForwardSlash%
	\let\@pytexPrevmacro@ForwardSlash\undefined%
	\expandafter\let\@pytexActive@Colon\@pytexPrevmacro@Colon%
	\let\@pytexPrevmacro@Colon\undefined%
	\expandafter\let\@pytexActive@Semicolon\@pytexPrevmacro@Semicolon%
	\let\@pytexPrevmacro@Semicolon\undefined%
	\expandafter\let\@pytexActive@LessThan\@pytexPrevmacro@LessThan%
	\let\@pytexPrevmacro@LessThan\undefined%
	\expandafter\let\@pytexActive@Equals\@pytexPrevmacro@Equals%
	\let\@pytexPrevmacro@Equals\undefined%
	\expandafter\let\@pytexActive@GreaterThan\@pytexPrevmacro@GreaterThan%
	\let\@pytexPrevmacro@GreaterThan\undefined%
	\expandafter\let\@pytexActive@QuestionMark\@pytexPrevmacro@QuestionMark%
	\let\@pytexPrevmacro@QuestionMark\undefined%
	\expandafter\let\@pytexActive@OpenBracket\@pytexPrevmacro@OpenBracket%
	\let\@pytexPrevmacro@OpenBracket\undefined%
	\expandafter\let\@pytexActive@CloseBracket\@pytexPrevmacro@CloseBracket%
	\let\@pytexPrevmacro@CloseBracket\undefined%
	\expandafter\let\@pytexActive@Caret\@pytexPrevmacro@Caret%
	\let\@pytexPrevmacro@Caret\undefined%
	\expandafter\let\@pytexActive@Underscore\@pytexPrevmacro@Underscore%
	\let\@pytexPrevmacro@Underscore\undefined%
	\expandafter\let\@pytexActive@Backtick\@pytexPrevmacro@Backtick%
	\let\@pytexPrevmacro@Backtick\undefined%
	\expandafter\let\@pytexActive@OpenBrace\@pytexPrevmacro@OpenBrace%
	\let\@pytexPrevmacro@OpenBrace\undefined%
	\expandafter\let\@pytexActive@Pipe\@pytexPrevmacro@Pipe%
	\let\@pytexPrevmacro@Pipe\undefined%
	\expandafter\let\@pytexActive@CloseBrace\@pytexPrevmacro@CloseBrace%
	\let\@pytexPrevmacro@CloseBrace\undefined%
	\expandafter\let\@pytexActive@Tilde\@pytexPrevmacro@Tilde%
	\let\@pytexPrevmacro@Tilde\undefined%
	\expandafter\let\@pytexActive@Backslash\@pytexPrevmacro@Backslash%
	\let\@pytexPrevmacro@Backslash\undefined%
	\expandafter\let\@pytexActive@Tab\@pytexPrevmacro@Tab%
	\let\@pytexPrevmacro@Tab\undefined%
	\expandafter\let\@pytexActive@Newline\@pytexPrevmacro@Newline%
	\let\@pytexPrevmacro@Newline\undefined%
}
\def\@pytexMakeAllActive{%
	\edef\@pytexPrevCatcode@A{\the\catcode0}%
	\edef\@pytexPrevCatcode@B{\the\catcode1}%
	\edef\@pytexPrevCatcode@C{\the\catcode2}%
	\edef\@pytexPrevCatcode@D{\the\catcode3}%
	\edef\@pytexPrevCatcode@E{\the\catcode4}%
	\edef\@pytexPrevCatcode@F{\the\catcode5}%
	\edef\@pytexPrevCatcode@G{\the\catcode6}%
	\edef\@pytexPrevCatcode@H{\the\catcode7}%
	\edef\@pytexPrevCatcode@I{\the\catcode8}%
	\edef\@pytexPrevCatcode@J{\the\catcode9}%
	\edef\@pytexPrevCatcode@BA{\the\catcode10}%
	\edef\@pytexPrevCatcode@BB{\the\catcode11}%
	\edef\@pytexPrevCatcode@BC{\the\catcode12}%
	\edef\@pytexPrevCatcode@BD{\the\catcode13}%
	\edef\@pytexPrevCatcode@BE{\the\catcode14}%
	\edef\@pytexPrevCatcode@BF{\the\catcode15}%
	\edef\@pytexPrevCatcode@BG{\the\catcode16}%
	\edef\@pytexPrevCatcode@BH{\the\catcode17}%
	\edef\@pytexPrevCatcode@BI{\the\catcode18}%
	\edef\@pytexPrevCatcode@BJ{\the\catcode19}%
	\edef\@pytexPrevCatcode@CA{\the\catcode20}%
	\edef\@pytexPrevCatcode@CB{\the\catcode21}%
	\edef\@pytexPrevCatcode@CC{\the\catcode22}%
	\edef\@pytexPrevCatcode@CD{\the\catcode23}%
	\edef\@pytexPrevCatcode@CE{\the\catcode24}%
	\edef\@pytexPrevCatcode@CF{\the\catcode25}%
	\edef\@pytexPrevCatcode@CG{\the\catcode26}%
	\edef\@pytexPrevCatcode@CH{\the\catcode27}%
	\edef\@pytexPrevCatcode@CI{\the\catcode28}%
	\edef\@pytexPrevCatcode@CJ{\the\catcode29}%
	\edef\@pytexPrevCatcode@DA{\the\catcode30}%
	\edef\@pytexPrevCatcode@DB{\the\catcode31}%
	\edef\@pytexPrevCatcode@DC{\the\catcode32}%
	\edef\@pytexPrevCatcode@DD{\the\catcode33}%
	\edef\@pytexPrevCatcode@DE{\the\catcode34}%
	\edef\@pytexPrevCatcode@DF{\the\catcode35}%
	\edef\@pytexPrevCatcode@DG{\the\catcode36}%
	\edef\@pytexPrevCatcode@DH{\the\catcode37}%
	\edef\@pytexPrevCatcode@DI{\the\catcode38}%
	\edef\@pytexPrevCatcode@DJ{\the\catcode39}%
	\edef\@pytexPrevCatcode@EA{\the\catcode40}%
	\edef\@pytexPrevCatcode@EB{\the\catcode41}%
	\edef\@pytexPrevCatcode@EC{\the\catcode42}%
	\edef\@pytexPrevCatcode@ED{\the\catcode43}%
	\edef\@pytexPrevCatcode@EE{\the\catcode44}%
	\edef\@pytexPrevCatcode@EF{\the\catcode45}%
	\edef\@pytexPrevCatcode@EG{\the\catcode46}%
	\edef\@pytexPrevCatcode@EH{\the\catcode47}%
	\edef\@pytexPrevCatcode@EI{\the\catcode48}%
	\edef\@pytexPrevCatcode@EJ{\the\catcode49}%
	\edef\@pytexPrevCatcode@FA{\the\catcode50}%
	\edef\@pytexPrevCatcode@FB{\the\catcode51}%
	\edef\@pytexPrevCatcode@FC{\the\catcode52}%
	\edef\@pytexPrevCatcode@FD{\the\catcode53}%
	\edef\@pytexPrevCatcode@FE{\the\catcode54}%
	\edef\@pytexPrevCatcode@FF{\the\catcode55}%
	\edef\@pytexPrevCatcode@FG{\the\catcode56}%
	\edef\@pytexPrevCatcode@FH{\the\catcode57}%
	\edef\@pytexPrevCatcode@FI{\the\catcode58}%
	\edef\@pytexPrevCatcode@FJ{\the\catcode59}%
	\edef\@pytexPrevCatcode@GA{\the\catcode60}%
	\edef\@pytexPrevCatcode@GB{\the\catcode61}%
	\edef\@pytexPrevCatcode@GC{\the\catcode62}%
	\edef\@pytexPrevCatcode@GD{\the\catcode63}%
	\edef\@pytexPrevCatcode@GE{\the\catcode64}%
	\edef\@pytexPrevCatcode@GF{\the\catcode65}%
	\edef\@pytexPrevCatcode@GG{\the\catcode66}%
	\edef\@pytexPrevCatcode@GH{\the\catcode67}%
	\edef\@pytexPrevCatcode@GI{\the\catcode68}%
	\edef\@pytexPrevCatcode@GJ{\the\catcode69}%
	\edef\@pytexPrevCatcode@HA{\the\catcode70}%
	\edef\@pytexPrevCatcode@HB{\the\catcode71}%
	\edef\@pytexPrevCatcode@HC{\the\catcode72}%
	\edef\@pytexPrevCatcode@HD{\the\catcode73}%
	\edef\@pytexPrevCatcode@HE{\the\catcode74}%
	\edef\@pytexPrevCatcode@HF{\the\catcode75}%
	\edef\@pytexPrevCatcode@HG{\the\catcode76}%
	\edef\@pytexPrevCatcode@HH{\the\catcode77}%
	\edef\@pytexPrevCatcode@HI{\the\catcode78}%
	\edef\@pytexPrevCatcode@HJ{\the\catcode79}%
	\edef\@pytexPrevCatcode@IA{\the\catcode80}%
	\edef\@pytexPrevCatcode@IB{\the\catcode81}%
	\edef\@pytexPrevCatcode@IC{\the\catcode82}%
	\edef\@pytexPrevCatcode@ID{\the\catcode83}%
	\edef\@pytexPrevCatcode@IE{\the\catcode84}%
	\edef\@pytexPrevCatcode@IF{\the\catcode85}%
	\edef\@pytexPrevCatcode@IG{\the\catcode86}%
	\edef\@pytexPrevCatcode@IH{\the\catcode87}%
	\edef\@pytexPrevCatcode@II{\the\catcode88}%
	\edef\@pytexPrevCatcode@IJ{\the\catcode89}%
	\edef\@pytexPrevCatcode@JA{\the\catcode90}%
	\edef\@pytexPrevCatcode@JB{\the\catcode91}%
	\edef\@pytexPrevCatcode@JC{\the\catcode92}%
	\edef\@pytexPrevCatcode@JD{\the\catcode93}%
	\edef\@pytexPrevCatcode@JE{\the\catcode94}%
	\edef\@pytexPrevCatcode@JF{\the\catcode95}%
	\edef\@pytexPrevCatcode@JG{\the\catcode96}%
	\edef\@pytexPrevCatcode@JH{\the\catcode97}%
	\edef\@pytexPrevCatcode@JI{\the\catcode98}%
	\edef\@pytexPrevCatcode@JJ{\the\catcode99}%
	\edef\@pytexPrevCatcode@BAA{\the\catcode100}%
	\edef\@pytexPrevCatcode@BAB{\the\catcode101}%
	\edef\@pytexPrevCatcode@BAC{\the\catcode102}%
	\edef\@pytexPrevCatcode@BAD{\the\catcode103}%
	\edef\@pytexPrevCatcode@BAE{\the\catcode104}%
	\edef\@pytexPrevCatcode@BAF{\the\catcode105}%
	\edef\@pytexPrevCatcode@BAG{\the\catcode106}%
	\edef\@pytexPrevCatcode@BAH{\the\catcode107}%
	\edef\@pytexPrevCatcode@BAI{\the\catcode108}%
	\edef\@pytexPrevCatcode@BAJ{\the\catcode109}%
	\edef\@pytexPrevCatcode@BBA{\the\catcode110}%
	\edef\@pytexPrevCatcode@BBB{\the\catcode111}%
	\edef\@pytexPrevCatcode@BBC{\the\catcode112}%
	\edef\@pytexPrevCatcode@BBD{\the\catcode113}%
	\edef\@pytexPrevCatcode@BBE{\the\catcode114}%
	\edef\@pytexPrevCatcode@BBF{\the\catcode115}%
	\edef\@pytexPrevCatcode@BBG{\the\catcode116}%
	\edef\@pytexPrevCatcode@BBH{\the\catcode117}%
	\edef\@pytexPrevCatcode@BBI{\the\catcode118}%
	\edef\@pytexPrevCatcode@BBJ{\the\catcode119}%
	\edef\@pytexPrevCatcode@BCA{\the\catcode120}%
	\edef\@pytexPrevCatcode@BCB{\the\catcode121}%
	\edef\@pytexPrevCatcode@BCC{\the\catcode122}%
	\edef\@pytexPrevCatcode@BCD{\the\catcode123}%
	\edef\@pytexPrevCatcode@BCE{\the\catcode124}%
	\edef\@pytexPrevCatcode@BCF{\the\catcode125}%
	\edef\@pytexPrevCatcode@BCG{\the\catcode126}%
	\edef\@pytexPrevCatcode@BCH{\the\catcode127}%
	\edef\@pytexPrevCatcode@BCI{\the\catcode128}%
	\edef\@pytexPrevCatcode@BCJ{\the\catcode129}%
	\edef\@pytexPrevCatcode@BDA{\the\catcode130}%
	\edef\@pytexPrevCatcode@BDB{\the\catcode131}%
	\edef\@pytexPrevCatcode@BDC{\the\catcode132}%
	\edef\@pytexPrevCatcode@BDD{\the\catcode133}%
	\edef\@pytexPrevCatcode@BDE{\the\catcode134}%
	\edef\@pytexPrevCatcode@BDF{\the\catcode135}%
	\edef\@pytexPrevCatcode@BDG{\the\catcode136}%
	\edef\@pytexPrevCatcode@BDH{\the\catcode137}%
	\edef\@pytexPrevCatcode@BDI{\the\catcode138}%
	\edef\@pytexPrevCatcode@BDJ{\the\catcode139}%
	\edef\@pytexPrevCatcode@BEA{\the\catcode140}%
	\edef\@pytexPrevCatcode@BEB{\the\catcode141}%
	\edef\@pytexPrevCatcode@BEC{\the\catcode142}%
	\edef\@pytexPrevCatcode@BED{\the\catcode143}%
	\edef\@pytexPrevCatcode@BEE{\the\catcode144}%
	\edef\@pytexPrevCatcode@BEF{\the\catcode145}%
	\edef\@pytexPrevCatcode@BEG{\the\catcode146}%
	\edef\@pytexPrevCatcode@BEH{\the\catcode147}%
	\edef\@pytexPrevCatcode@BEI{\the\catcode148}%
	\edef\@pytexPrevCatcode@BEJ{\the\catcode149}%
	\edef\@pytexPrevCatcode@BFA{\the\catcode150}%
	\edef\@pytexPrevCatcode@BFB{\the\catcode151}%
	\edef\@pytexPrevCatcode@BFC{\the\catcode152}%
	\edef\@pytexPrevCatcode@BFD{\the\catcode153}%
	\edef\@pytexPrevCatcode@BFE{\the\catcode154}%
	\edef\@pytexPrevCatcode@BFF{\the\catcode155}%
	\edef\@pytexPrevCatcode@BFG{\the\catcode156}%
	\edef\@pytexPrevCatcode@BFH{\the\catcode157}%
	\edef\@pytexPrevCatcode@BFI{\the\catcode158}%
	\edef\@pytexPrevCatcode@BFJ{\the\catcode159}%
	\edef\@pytexPrevCatcode@BGA{\the\catcode160}%
	\edef\@pytexPrevCatcode@BGB{\the\catcode161}%
	\edef\@pytexPrevCatcode@BGC{\the\catcode162}%
	\edef\@pytexPrevCatcode@BGD{\the\catcode163}%
	\edef\@pytexPrevCatcode@BGE{\the\catcode164}%
	\edef\@pytexPrevCatcode@BGF{\the\catcode165}%
	\edef\@pytexPrevCatcode@BGG{\the\catcode166}%
	\edef\@pytexPrevCatcode@BGH{\the\catcode167}%
	\edef\@pytexPrevCatcode@BGI{\the\catcode168}%
	\edef\@pytexPrevCatcode@BGJ{\the\catcode169}%
	\edef\@pytexPrevCatcode@BHA{\the\catcode170}%
	\edef\@pytexPrevCatcode@BHB{\the\catcode171}%
	\edef\@pytexPrevCatcode@BHC{\the\catcode172}%
	\edef\@pytexPrevCatcode@BHD{\the\catcode173}%
	\edef\@pytexPrevCatcode@BHE{\the\catcode174}%
	\edef\@pytexPrevCatcode@BHF{\the\catcode175}%
	\edef\@pytexPrevCatcode@BHG{\the\catcode176}%
	\edef\@pytexPrevCatcode@BHH{\the\catcode177}%
	\edef\@pytexPrevCatcode@BHI{\the\catcode178}%
	\edef\@pytexPrevCatcode@BHJ{\the\catcode179}%
	\edef\@pytexPrevCatcode@BIA{\the\catcode180}%
	\edef\@pytexPrevCatcode@BIB{\the\catcode181}%
	\edef\@pytexPrevCatcode@BIC{\the\catcode182}%
	\edef\@pytexPrevCatcode@BID{\the\catcode183}%
	\edef\@pytexPrevCatcode@BIE{\the\catcode184}%
	\edef\@pytexPrevCatcode@BIF{\the\catcode185}%
	\edef\@pytexPrevCatcode@BIG{\the\catcode186}%
	\edef\@pytexPrevCatcode@BIH{\the\catcode187}%
	\edef\@pytexPrevCatcode@BII{\the\catcode188}%
	\edef\@pytexPrevCatcode@BIJ{\the\catcode189}%
	\edef\@pytexPrevCatcode@BJA{\the\catcode190}%
	\edef\@pytexPrevCatcode@BJB{\the\catcode191}%
	\edef\@pytexPrevCatcode@BJC{\the\catcode192}%
	\edef\@pytexPrevCatcode@BJD{\the\catcode193}%
	\edef\@pytexPrevCatcode@BJE{\the\catcode194}%
	\edef\@pytexPrevCatcode@BJF{\the\catcode195}%
	\edef\@pytexPrevCatcode@BJG{\the\catcode196}%
	\edef\@pytexPrevCatcode@BJH{\the\catcode197}%
	\edef\@pytexPrevCatcode@BJI{\the\catcode198}%
	\edef\@pytexPrevCatcode@BJJ{\the\catcode199}%
	\edef\@pytexPrevCatcode@CAA{\the\catcode200}%
	\edef\@pytexPrevCatcode@CAB{\the\catcode201}%
	\edef\@pytexPrevCatcode@CAC{\the\catcode202}%
	\edef\@pytexPrevCatcode@CAD{\the\catcode203}%
	\edef\@pytexPrevCatcode@CAE{\the\catcode204}%
	\edef\@pytexPrevCatcode@CAF{\the\catcode205}%
	\edef\@pytexPrevCatcode@CAG{\the\catcode206}%
	\edef\@pytexPrevCatcode@CAH{\the\catcode207}%
	\edef\@pytexPrevCatcode@CAI{\the\catcode208}%
	\edef\@pytexPrevCatcode@CAJ{\the\catcode209}%
	\edef\@pytexPrevCatcode@CBA{\the\catcode210}%
	\edef\@pytexPrevCatcode@CBB{\the\catcode211}%
	\edef\@pytexPrevCatcode@CBC{\the\catcode212}%
	\edef\@pytexPrevCatcode@CBD{\the\catcode213}%
	\edef\@pytexPrevCatcode@CBE{\the\catcode214}%
	\edef\@pytexPrevCatcode@CBF{\the\catcode215}%
	\edef\@pytexPrevCatcode@CBG{\the\catcode216}%
	\edef\@pytexPrevCatcode@CBH{\the\catcode217}%
	\edef\@pytexPrevCatcode@CBI{\the\catcode218}%
	\edef\@pytexPrevCatcode@CBJ{\the\catcode219}%
	\edef\@pytexPrevCatcode@CCA{\the\catcode220}%
	\edef\@pytexPrevCatcode@CCB{\the\catcode221}%
	\edef\@pytexPrevCatcode@CCC{\the\catcode222}%
	\edef\@pytexPrevCatcode@CCD{\the\catcode223}%
	\edef\@pytexPrevCatcode@CCE{\the\catcode224}%
	\edef\@pytexPrevCatcode@CCF{\the\catcode225}%
	\edef\@pytexPrevCatcode@CCG{\the\catcode226}%
	\edef\@pytexPrevCatcode@CCH{\the\catcode227}%
	\edef\@pytexPrevCatcode@CCI{\the\catcode228}%
	\edef\@pytexPrevCatcode@CCJ{\the\catcode229}%
	\edef\@pytexPrevCatcode@CDA{\the\catcode230}%
	\edef\@pytexPrevCatcode@CDB{\the\catcode231}%
	\edef\@pytexPrevCatcode@CDC{\the\catcode232}%
	\edef\@pytexPrevCatcode@CDD{\the\catcode233}%
	\edef\@pytexPrevCatcode@CDE{\the\catcode234}%
	\edef\@pytexPrevCatcode@CDF{\the\catcode235}%
	\edef\@pytexPrevCatcode@CDG{\the\catcode236}%
	\edef\@pytexPrevCatcode@CDH{\the\catcode237}%
	\edef\@pytexPrevCatcode@CDI{\the\catcode238}%
	\edef\@pytexPrevCatcode@CDJ{\the\catcode239}%
	\edef\@pytexPrevCatcode@CEA{\the\catcode240}%
	\edef\@pytexPrevCatcode@CEB{\the\catcode241}%
	\edef\@pytexPrevCatcode@CEC{\the\catcode242}%
	\edef\@pytexPrevCatcode@CED{\the\catcode243}%
	\edef\@pytexPrevCatcode@CEE{\the\catcode244}%
	\edef\@pytexPrevCatcode@CEF{\the\catcode245}%
	\edef\@pytexPrevCatcode@CEG{\the\catcode246}%
	\edef\@pytexPrevCatcode@CEH{\the\catcode247}%
	\edef\@pytexPrevCatcode@CEI{\the\catcode248}%
	\edef\@pytexPrevCatcode@CEJ{\the\catcode249}%
	\edef\@pytexPrevCatcode@CFA{\the\catcode250}%
	\edef\@pytexPrevCatcode@CFB{\the\catcode251}%
	\edef\@pytexPrevCatcode@CFC{\the\catcode252}%
	\edef\@pytexPrevCatcode@CFD{\the\catcode253}%
	\edef\@pytexPrevCatcode@CFE{\the\catcode254}%
	\edef\@pytexPrevCatcode@CFF{\the\catcode255}%
	\catcode48=13%
	\catcode49=13%
	\catcode50=13%
	\catcode51=13%
	\catcode52=13%
	\catcode53=13%
	\catcode54=13%
	\catcode55=13%
	\catcode56=13%
	\catcode57=13%
	\catcode97=13%
	\catcode98=13%
	\catcode99=13%
	\catcode100=13%
	\catcode101=13%
	\catcode102=13%
	\catcode103=13%
	\catcode104=13%
	\catcode105=13%
	\catcode106=13%
	\catcode107=13%
	\catcode108=13%
	\catcode109=13%
	\catcode110=13%
	\catcode111=13%
	\catcode112=13%
	\catcode113=13%
	\catcode114=13%
	\catcode115=13%
	\catcode116=13%
	\catcode117=13%
	\catcode118=13%
	\catcode119=13%
	\catcode120=13%
	\catcode121=13%
	\catcode122=13%
	\catcode65=13%
	\catcode66=13%
	\catcode67=13%
	\catcode68=13%
	\catcode69=13%
	\catcode70=13%
	\catcode71=13%
	\catcode72=13%
	\catcode73=13%
	\catcode74=13%
	\catcode75=13%
	\catcode76=13%
	\catcode77=13%
	\catcode78=13%
	\catcode79=13%
	\catcode80=13%
	\catcode81=13%
	\catcode82=13%
	\catcode83=13%
	\catcode84=13%
	\catcode85=13%
	\catcode86=13%
	\catcode87=13%
	\catcode88=13%
	\catcode89=13%
	\catcode90=13%
	\catcode33=13%
	\catcode34=13%
	\catcode35=13%
	\catcode36=13%
	\catcode37=13%
	\catcode38=13%
	\catcode39=13%
	\catcode40=13%
	\catcode41=13%
	\catcode42=13%
	\catcode43=13%
	\catcode44=13%
	\catcode45=13%
	\catcode46=13%
	\catcode47=13%
	\catcode58=13%
	\catcode59=13%
	\catcode60=13%
	\catcode61=13%
	\catcode62=13%
	\catcode63=13%
	\catcode64=13%
	\catcode91=13%
	\catcode92=13%
	\catcode93=13%
	\catcode94=13%
	\catcode95=13%
	\catcode96=13%
	\catcode123=13%
	\catcode124=13%
	\catcode125=13%
	\catcode126=13%
	\catcode32=13%
	\catcode9=13%
	\catcode10=13%
	\catcode13=13%
	\catcode11=13%
	\catcode12=13%
}
\def\@pytexResetCatcodes{%
	\catcode0=\@pytexPrevCatcode@A%
	\let\@pytexPrevCatcode@A\undefined%
	\catcode1=\@pytexPrevCatcode@B%
	\let\@pytexPrevCatcode@B\undefined%
	\catcode2=\@pytexPrevCatcode@C%
	\let\@pytexPrevCatcode@C\undefined%
	\catcode3=\@pytexPrevCatcode@D%
	\let\@pytexPrevCatcode@D\undefined%
	\catcode4=\@pytexPrevCatcode@E%
	\let\@pytexPrevCatcode@E\undefined%
	\catcode5=\@pytexPrevCatcode@F%
	\let\@pytexPrevCatcode@F\undefined%
	\catcode6=\@pytexPrevCatcode@G%
	\let\@pytexPrevCatcode@G\undefined%
	\catcode7=\@pytexPrevCatcode@H%
	\let\@pytexPrevCatcode@H\undefined%
	\catcode8=\@pytexPrevCatcode@I%
	\let\@pytexPrevCatcode@I\undefined%
	\catcode9=\@pytexPrevCatcode@J%
	\let\@pytexPrevCatcode@J\undefined%
	\catcode10=\@pytexPrevCatcode@BA%
	\let\@pytexPrevCatcode@BA\undefined%
	\catcode11=\@pytexPrevCatcode@BB%
	\let\@pytexPrevCatcode@BB\undefined%
	\catcode12=\@pytexPrevCatcode@BC%
	\let\@pytexPrevCatcode@BC\undefined%
	\catcode13=\@pytexPrevCatcode@BD%
	\let\@pytexPrevCatcode@BD\undefined%
	\catcode14=\@pytexPrevCatcode@BE%
	\let\@pytexPrevCatcode@BE\undefined%
	\catcode15=\@pytexPrevCatcode@BF%
	\let\@pytexPrevCatcode@BF\undefined%
	\catcode16=\@pytexPrevCatcode@BG%
	\let\@pytexPrevCatcode@BG\undefined%
	\catcode17=\@pytexPrevCatcode@BH%
	\let\@pytexPrevCatcode@BH\undefined%
	\catcode18=\@pytexPrevCatcode@BI%
	\let\@pytexPrevCatcode@BI\undefined%
	\catcode19=\@pytexPrevCatcode@BJ%
	\let\@pytexPrevCatcode@BJ\undefined%
	\catcode20=\@pytexPrevCatcode@CA%
	\let\@pytexPrevCatcode@CA\undefined%
	\catcode21=\@pytexPrevCatcode@CB%
	\let\@pytexPrevCatcode@CB\undefined%
	\catcode22=\@pytexPrevCatcode@CC%
	\let\@pytexPrevCatcode@CC\undefined%
	\catcode23=\@pytexPrevCatcode@CD%
	\let\@pytexPrevCatcode@CD\undefined%
	\catcode24=\@pytexPrevCatcode@CE%
	\let\@pytexPrevCatcode@CE\undefined%
	\catcode25=\@pytexPrevCatcode@CF%
	\let\@pytexPrevCatcode@CF\undefined%
	\catcode26=\@pytexPrevCatcode@CG%
	\let\@pytexPrevCatcode@CG\undefined%
	\catcode27=\@pytexPrevCatcode@CH%
	\let\@pytexPrevCatcode@CH\undefined%
	\catcode28=\@pytexPrevCatcode@CI%
	\let\@pytexPrevCatcode@CI\undefined%
	\catcode29=\@pytexPrevCatcode@CJ%
	\let\@pytexPrevCatcode@CJ\undefined%
	\catcode30=\@pytexPrevCatcode@DA%
	\let\@pytexPrevCatcode@DA\undefined%
	\catcode31=\@pytexPrevCatcode@DB%
	\let\@pytexPrevCatcode@DB\undefined%
	\catcode32=\@pytexPrevCatcode@DC%
	\let\@pytexPrevCatcode@DC\undefined%
	\catcode33=\@pytexPrevCatcode@DD%
	\let\@pytexPrevCatcode@DD\undefined%
	\catcode34=\@pytexPrevCatcode@DE%
	\let\@pytexPrevCatcode@DE\undefined%
	\catcode35=\@pytexPrevCatcode@DF%
	\let\@pytexPrevCatcode@DF\undefined%
	\catcode36=\@pytexPrevCatcode@DG%
	\let\@pytexPrevCatcode@DG\undefined%
	\catcode37=\@pytexPrevCatcode@DH%
	\let\@pytexPrevCatcode@DH\undefined%
	\catcode38=\@pytexPrevCatcode@DI%
	\let\@pytexPrevCatcode@DI\undefined%
	\catcode39=\@pytexPrevCatcode@DJ%
	\let\@pytexPrevCatcode@DJ\undefined%
	\catcode40=\@pytexPrevCatcode@EA%
	\let\@pytexPrevCatcode@EA\undefined%
	\catcode41=\@pytexPrevCatcode@EB%
	\let\@pytexPrevCatcode@EB\undefined%
	\catcode42=\@pytexPrevCatcode@EC%
	\let\@pytexPrevCatcode@EC\undefined%
	\catcode43=\@pytexPrevCatcode@ED%
	\let\@pytexPrevCatcode@ED\undefined%
	\catcode44=\@pytexPrevCatcode@EE%
	\let\@pytexPrevCatcode@EE\undefined%
	\catcode45=\@pytexPrevCatcode@EF%
	\let\@pytexPrevCatcode@EF\undefined%
	\catcode46=\@pytexPrevCatcode@EG%
	\let\@pytexPrevCatcode@EG\undefined%
	\catcode47=\@pytexPrevCatcode@EH%
	\let\@pytexPrevCatcode@EH\undefined%
	\catcode48=\@pytexPrevCatcode@EI%
	\let\@pytexPrevCatcode@EI\undefined%
	\catcode49=\@pytexPrevCatcode@EJ%
	\let\@pytexPrevCatcode@EJ\undefined%
	\catcode50=\@pytexPrevCatcode@FA%
	\let\@pytexPrevCatcode@FA\undefined%
	\catcode51=\@pytexPrevCatcode@FB%
	\let\@pytexPrevCatcode@FB\undefined%
	\catcode52=\@pytexPrevCatcode@FC%
	\let\@pytexPrevCatcode@FC\undefined%
	\catcode53=\@pytexPrevCatcode@FD%
	\let\@pytexPrevCatcode@FD\undefined%
	\catcode54=\@pytexPrevCatcode@FE%
	\let\@pytexPrevCatcode@FE\undefined%
	\catcode55=\@pytexPrevCatcode@FF%
	\let\@pytexPrevCatcode@FF\undefined%
	\catcode56=\@pytexPrevCatcode@FG%
	\let\@pytexPrevCatcode@FG\undefined%
	\catcode57=\@pytexPrevCatcode@FH%
	\let\@pytexPrevCatcode@FH\undefined%
	\catcode58=\@pytexPrevCatcode@FI%
	\let\@pytexPrevCatcode@FI\undefined%
	\catcode59=\@pytexPrevCatcode@FJ%
	\let\@pytexPrevCatcode@FJ\undefined%
	\catcode60=\@pytexPrevCatcode@GA%
	\let\@pytexPrevCatcode@GA\undefined%
	\catcode61=\@pytexPrevCatcode@GB%
	\let\@pytexPrevCatcode@GB\undefined%
	\catcode62=\@pytexPrevCatcode@GC%
	\let\@pytexPrevCatcode@GC\undefined%
	\catcode63=\@pytexPrevCatcode@GD%
	\let\@pytexPrevCatcode@GD\undefined%
	\catcode64=\@pytexPrevCatcode@GE%
	\let\@pytexPrevCatcode@GE\undefined%
	\catcode65=\@pytexPrevCatcode@GF%
	\let\@pytexPrevCatcode@GF\undefined%
	\catcode66=\@pytexPrevCatcode@GG%
	\let\@pytexPrevCatcode@GG\undefined%
	\catcode67=\@pytexPrevCatcode@GH%
	\let\@pytexPrevCatcode@GH\undefined%
	\catcode68=\@pytexPrevCatcode@GI%
	\let\@pytexPrevCatcode@GI\undefined%
	\catcode69=\@pytexPrevCatcode@GJ%
	\let\@pytexPrevCatcode@GJ\undefined%
	\catcode70=\@pytexPrevCatcode@HA%
	\let\@pytexPrevCatcode@HA\undefined%
	\catcode71=\@pytexPrevCatcode@HB%
	\let\@pytexPrevCatcode@HB\undefined%
	\catcode72=\@pytexPrevCatcode@HC%
	\let\@pytexPrevCatcode@HC\undefined%
	\catcode73=\@pytexPrevCatcode@HD%
	\let\@pytexPrevCatcode@HD\undefined%
	\catcode74=\@pytexPrevCatcode@HE%
	\let\@pytexPrevCatcode@HE\undefined%
	\catcode75=\@pytexPrevCatcode@HF%
	\let\@pytexPrevCatcode@HF\undefined%
	\catcode76=\@pytexPrevCatcode@HG%
	\let\@pytexPrevCatcode@HG\undefined%
	\catcode77=\@pytexPrevCatcode@HH%
	\let\@pytexPrevCatcode@HH\undefined%
	\catcode78=\@pytexPrevCatcode@HI%
	\let\@pytexPrevCatcode@HI\undefined%
	\catcode79=\@pytexPrevCatcode@HJ%
	\let\@pytexPrevCatcode@HJ\undefined%
	\catcode80=\@pytexPrevCatcode@IA%
	\let\@pytexPrevCatcode@IA\undefined%
	\catcode81=\@pytexPrevCatcode@IB%
	\let\@pytexPrevCatcode@IB\undefined%
	\catcode82=\@pytexPrevCatcode@IC%
	\let\@pytexPrevCatcode@IC\undefined%
	\catcode83=\@pytexPrevCatcode@ID%
	\let\@pytexPrevCatcode@ID\undefined%
	\catcode84=\@pytexPrevCatcode@IE%
	\let\@pytexPrevCatcode@IE\undefined%
	\catcode85=\@pytexPrevCatcode@IF%
	\let\@pytexPrevCatcode@IF\undefined%
	\catcode86=\@pytexPrevCatcode@IG%
	\let\@pytexPrevCatcode@IG\undefined%
	\catcode87=\@pytexPrevCatcode@IH%
	\let\@pytexPrevCatcode@IH\undefined%
	\catcode88=\@pytexPrevCatcode@II%
	\let\@pytexPrevCatcode@II\undefined%
	\catcode89=\@pytexPrevCatcode@IJ%
	\let\@pytexPrevCatcode@IJ\undefined%
	\catcode90=\@pytexPrevCatcode@JA%
	\let\@pytexPrevCatcode@JA\undefined%
	\catcode91=\@pytexPrevCatcode@JB%
	\let\@pytexPrevCatcode@JB\undefined%
	\catcode92=\@pytexPrevCatcode@JC%
	\let\@pytexPrevCatcode@JC\undefined%
	\catcode93=\@pytexPrevCatcode@JD%
	\let\@pytexPrevCatcode@JD\undefined%
	\catcode94=\@pytexPrevCatcode@JE%
	\let\@pytexPrevCatcode@JE\undefined%
	\catcode95=\@pytexPrevCatcode@JF%
	\let\@pytexPrevCatcode@JF\undefined%
	\catcode96=\@pytexPrevCatcode@JG%
	\let\@pytexPrevCatcode@JG\undefined%
	\catcode97=\@pytexPrevCatcode@JH%
	\let\@pytexPrevCatcode@JH\undefined%
	\catcode98=\@pytexPrevCatcode@JI%
	\let\@pytexPrevCatcode@JI\undefined%
	\catcode99=\@pytexPrevCatcode@JJ%
	\let\@pytexPrevCatcode@JJ\undefined%
	\catcode100=\@pytexPrevCatcode@BAA%
	\let\@pytexPrevCatcode@BAA\undefined%
	\catcode101=\@pytexPrevCatcode@BAB%
	\let\@pytexPrevCatcode@BAB\undefined%
	\catcode102=\@pytexPrevCatcode@BAC%
	\let\@pytexPrevCatcode@BAC\undefined%
	\catcode103=\@pytexPrevCatcode@BAD%
	\let\@pytexPrevCatcode@BAD\undefined%
	\catcode104=\@pytexPrevCatcode@BAE%
	\let\@pytexPrevCatcode@BAE\undefined%
	\catcode105=\@pytexPrevCatcode@BAF%
	\let\@pytexPrevCatcode@BAF\undefined%
	\catcode106=\@pytexPrevCatcode@BAG%
	\let\@pytexPrevCatcode@BAG\undefined%
	\catcode107=\@pytexPrevCatcode@BAH%
	\let\@pytexPrevCatcode@BAH\undefined%
	\catcode108=\@pytexPrevCatcode@BAI%
	\let\@pytexPrevCatcode@BAI\undefined%
	\catcode109=\@pytexPrevCatcode@BAJ%
	\let\@pytexPrevCatcode@BAJ\undefined%
	\catcode110=\@pytexPrevCatcode@BBA%
	\let\@pytexPrevCatcode@BBA\undefined%
	\catcode111=\@pytexPrevCatcode@BBB%
	\let\@pytexPrevCatcode@BBB\undefined%
	\catcode112=\@pytexPrevCatcode@BBC%
	\let\@pytexPrevCatcode@BBC\undefined%
	\catcode113=\@pytexPrevCatcode@BBD%
	\let\@pytexPrevCatcode@BBD\undefined%
	\catcode114=\@pytexPrevCatcode@BBE%
	\let\@pytexPrevCatcode@BBE\undefined%
	\catcode115=\@pytexPrevCatcode@BBF%
	\let\@pytexPrevCatcode@BBF\undefined%
	\catcode116=\@pytexPrevCatcode@BBG%
	\let\@pytexPrevCatcode@BBG\undefined%
	\catcode117=\@pytexPrevCatcode@BBH%
	\let\@pytexPrevCatcode@BBH\undefined%
	\catcode118=\@pytexPrevCatcode@BBI%
	\let\@pytexPrevCatcode@BBI\undefined%
	\catcode119=\@pytexPrevCatcode@BBJ%
	\let\@pytexPrevCatcode@BBJ\undefined%
	\catcode120=\@pytexPrevCatcode@BCA%
	\let\@pytexPrevCatcode@BCA\undefined%
	\catcode121=\@pytexPrevCatcode@BCB%
	\let\@pytexPrevCatcode@BCB\undefined%
	\catcode122=\@pytexPrevCatcode@BCC%
	\let\@pytexPrevCatcode@BCC\undefined%
	\catcode123=\@pytexPrevCatcode@BCD%
	\let\@pytexPrevCatcode@BCD\undefined%
	\catcode124=\@pytexPrevCatcode@BCE%
	\let\@pytexPrevCatcode@BCE\undefined%
	\catcode125=\@pytexPrevCatcode@BCF%
	\let\@pytexPrevCatcode@BCF\undefined%
	\catcode126=\@pytexPrevCatcode@BCG%
	\let\@pytexPrevCatcode@BCG\undefined%
	\catcode127=\@pytexPrevCatcode@BCH%
	\let\@pytexPrevCatcode@BCH\undefined%
	\catcode128=\@pytexPrevCatcode@BCI%
	\let\@pytexPrevCatcode@BCI\undefined%
	\catcode129=\@pytexPrevCatcode@BCJ%
	\let\@pytexPrevCatcode@BCJ\undefined%
	\catcode130=\@pytexPrevCatcode@BDA%
	\let\@pytexPrevCatcode@BDA\undefined%
	\catcode131=\@pytexPrevCatcode@BDB%
	\let\@pytexPrevCatcode@BDB\undefined%
	\catcode132=\@pytexPrevCatcode@BDC%
	\let\@pytexPrevCatcode@BDC\undefined%
	\catcode133=\@pytexPrevCatcode@BDD%
	\let\@pytexPrevCatcode@BDD\undefined%
	\catcode134=\@pytexPrevCatcode@BDE%
	\let\@pytexPrevCatcode@BDE\undefined%
	\catcode135=\@pytexPrevCatcode@BDF%
	\let\@pytexPrevCatcode@BDF\undefined%
	\catcode136=\@pytexPrevCatcode@BDG%
	\let\@pytexPrevCatcode@BDG\undefined%
	\catcode137=\@pytexPrevCatcode@BDH%
	\let\@pytexPrevCatcode@BDH\undefined%
	\catcode138=\@pytexPrevCatcode@BDI%
	\let\@pytexPrevCatcode@BDI\undefined%
	\catcode139=\@pytexPrevCatcode@BDJ%
	\let\@pytexPrevCatcode@BDJ\undefined%
	\catcode140=\@pytexPrevCatcode@BEA%
	\let\@pytexPrevCatcode@BEA\undefined%
	\catcode141=\@pytexPrevCatcode@BEB%
	\let\@pytexPrevCatcode@BEB\undefined%
	\catcode142=\@pytexPrevCatcode@BEC%
	\let\@pytexPrevCatcode@BEC\undefined%
	\catcode143=\@pytexPrevCatcode@BED%
	\let\@pytexPrevCatcode@BED\undefined%
	\catcode144=\@pytexPrevCatcode@BEE%
	\let\@pytexPrevCatcode@BEE\undefined%
	\catcode145=\@pytexPrevCatcode@BEF%
	\let\@pytexPrevCatcode@BEF\undefined%
	\catcode146=\@pytexPrevCatcode@BEG%
	\let\@pytexPrevCatcode@BEG\undefined%
	\catcode147=\@pytexPrevCatcode@BEH%
	\let\@pytexPrevCatcode@BEH\undefined%
	\catcode148=\@pytexPrevCatcode@BEI%
	\let\@pytexPrevCatcode@BEI\undefined%
	\catcode149=\@pytexPrevCatcode@BEJ%
	\let\@pytexPrevCatcode@BEJ\undefined%
	\catcode150=\@pytexPrevCatcode@BFA%
	\let\@pytexPrevCatcode@BFA\undefined%
	\catcode151=\@pytexPrevCatcode@BFB%
	\let\@pytexPrevCatcode@BFB\undefined%
	\catcode152=\@pytexPrevCatcode@BFC%
	\let\@pytexPrevCatcode@BFC\undefined%
	\catcode153=\@pytexPrevCatcode@BFD%
	\let\@pytexPrevCatcode@BFD\undefined%
	\catcode154=\@pytexPrevCatcode@BFE%
	\let\@pytexPrevCatcode@BFE\undefined%
	\catcode155=\@pytexPrevCatcode@BFF%
	\let\@pytexPrevCatcode@BFF\undefined%
	\catcode156=\@pytexPrevCatcode@BFG%
	\let\@pytexPrevCatcode@BFG\undefined%
	\catcode157=\@pytexPrevCatcode@BFH%
	\let\@pytexPrevCatcode@BFH\undefined%
	\catcode158=\@pytexPrevCatcode@BFI%
	\let\@pytexPrevCatcode@BFI\undefined%
	\catcode159=\@pytexPrevCatcode@BFJ%
	\let\@pytexPrevCatcode@BFJ\undefined%
	\catcode160=\@pytexPrevCatcode@BGA%
	\let\@pytexPrevCatcode@BGA\undefined%
	\catcode161=\@pytexPrevCatcode@BGB%
	\let\@pytexPrevCatcode@BGB\undefined%
	\catcode162=\@pytexPrevCatcode@BGC%
	\let\@pytexPrevCatcode@BGC\undefined%
	\catcode163=\@pytexPrevCatcode@BGD%
	\let\@pytexPrevCatcode@BGD\undefined%
	\catcode164=\@pytexPrevCatcode@BGE%
	\let\@pytexPrevCatcode@BGE\undefined%
	\catcode165=\@pytexPrevCatcode@BGF%
	\let\@pytexPrevCatcode@BGF\undefined%
	\catcode166=\@pytexPrevCatcode@BGG%
	\let\@pytexPrevCatcode@BGG\undefined%
	\catcode167=\@pytexPrevCatcode@BGH%
	\let\@pytexPrevCatcode@BGH\undefined%
	\catcode168=\@pytexPrevCatcode@BGI%
	\let\@pytexPrevCatcode@BGI\undefined%
	\catcode169=\@pytexPrevCatcode@BGJ%
	\let\@pytexPrevCatcode@BGJ\undefined%
	\catcode170=\@pytexPrevCatcode@BHA%
	\let\@pytexPrevCatcode@BHA\undefined%
	\catcode171=\@pytexPrevCatcode@BHB%
	\let\@pytexPrevCatcode@BHB\undefined%
	\catcode172=\@pytexPrevCatcode@BHC%
	\let\@pytexPrevCatcode@BHC\undefined%
	\catcode173=\@pytexPrevCatcode@BHD%
	\let\@pytexPrevCatcode@BHD\undefined%
	\catcode174=\@pytexPrevCatcode@BHE%
	\let\@pytexPrevCatcode@BHE\undefined%
	\catcode175=\@pytexPrevCatcode@BHF%
	\let\@pytexPrevCatcode@BHF\undefined%
	\catcode176=\@pytexPrevCatcode@BHG%
	\let\@pytexPrevCatcode@BHG\undefined%
	\catcode177=\@pytexPrevCatcode@BHH%
	\let\@pytexPrevCatcode@BHH\undefined%
	\catcode178=\@pytexPrevCatcode@BHI%
	\let\@pytexPrevCatcode@BHI\undefined%
	\catcode179=\@pytexPrevCatcode@BHJ%
	\let\@pytexPrevCatcode@BHJ\undefined%
	\catcode180=\@pytexPrevCatcode@BIA%
	\let\@pytexPrevCatcode@BIA\undefined%
	\catcode181=\@pytexPrevCatcode@BIB%
	\let\@pytexPrevCatcode@BIB\undefined%
	\catcode182=\@pytexPrevCatcode@BIC%
	\let\@pytexPrevCatcode@BIC\undefined%
	\catcode183=\@pytexPrevCatcode@BID%
	\let\@pytexPrevCatcode@BID\undefined%
	\catcode184=\@pytexPrevCatcode@BIE%
	\let\@pytexPrevCatcode@BIE\undefined%
	\catcode185=\@pytexPrevCatcode@BIF%
	\let\@pytexPrevCatcode@BIF\undefined%
	\catcode186=\@pytexPrevCatcode@BIG%
	\let\@pytexPrevCatcode@BIG\undefined%
	\catcode187=\@pytexPrevCatcode@BIH%
	\let\@pytexPrevCatcode@BIH\undefined%
	\catcode188=\@pytexPrevCatcode@BII%
	\let\@pytexPrevCatcode@BII\undefined%
	\catcode189=\@pytexPrevCatcode@BIJ%
	\let\@pytexPrevCatcode@BIJ\undefined%
	\catcode190=\@pytexPrevCatcode@BJA%
	\let\@pytexPrevCatcode@BJA\undefined%
	\catcode191=\@pytexPrevCatcode@BJB%
	\let\@pytexPrevCatcode@BJB\undefined%
	\catcode192=\@pytexPrevCatcode@BJC%
	\let\@pytexPrevCatcode@BJC\undefined%
	\catcode193=\@pytexPrevCatcode@BJD%
	\let\@pytexPrevCatcode@BJD\undefined%
	\catcode194=\@pytexPrevCatcode@BJE%
	\let\@pytexPrevCatcode@BJE\undefined%
	\catcode195=\@pytexPrevCatcode@BJF%
	\let\@pytexPrevCatcode@BJF\undefined%
	\catcode196=\@pytexPrevCatcode@BJG%
	\let\@pytexPrevCatcode@BJG\undefined%
	\catcode197=\@pytexPrevCatcode@BJH%
	\let\@pytexPrevCatcode@BJH\undefined%
	\catcode198=\@pytexPrevCatcode@BJI%
	\let\@pytexPrevCatcode@BJI\undefined%
	\catcode199=\@pytexPrevCatcode@BJJ%
	\let\@pytexPrevCatcode@BJJ\undefined%
	\catcode200=\@pytexPrevCatcode@CAA%
	\let\@pytexPrevCatcode@CAA\undefined%
	\catcode201=\@pytexPrevCatcode@CAB%
	\let\@pytexPrevCatcode@CAB\undefined%
	\catcode202=\@pytexPrevCatcode@CAC%
	\let\@pytexPrevCatcode@CAC\undefined%
	\catcode203=\@pytexPrevCatcode@CAD%
	\let\@pytexPrevCatcode@CAD\undefined%
	\catcode204=\@pytexPrevCatcode@CAE%
	\let\@pytexPrevCatcode@CAE\undefined%
	\catcode205=\@pytexPrevCatcode@CAF%
	\let\@pytexPrevCatcode@CAF\undefined%
	\catcode206=\@pytexPrevCatcode@CAG%
	\let\@pytexPrevCatcode@CAG\undefined%
	\catcode207=\@pytexPrevCatcode@CAH%
	\let\@pytexPrevCatcode@CAH\undefined%
	\catcode208=\@pytexPrevCatcode@CAI%
	\let\@pytexPrevCatcode@CAI\undefined%
	\catcode209=\@pytexPrevCatcode@CAJ%
	\let\@pytexPrevCatcode@CAJ\undefined%
	\catcode210=\@pytexPrevCatcode@CBA%
	\let\@pytexPrevCatcode@CBA\undefined%
	\catcode211=\@pytexPrevCatcode@CBB%
	\let\@pytexPrevCatcode@CBB\undefined%
	\catcode212=\@pytexPrevCatcode@CBC%
	\let\@pytexPrevCatcode@CBC\undefined%
	\catcode213=\@pytexPrevCatcode@CBD%
	\let\@pytexPrevCatcode@CBD\undefined%
	\catcode214=\@pytexPrevCatcode@CBE%
	\let\@pytexPrevCatcode@CBE\undefined%
	\catcode215=\@pytexPrevCatcode@CBF%
	\let\@pytexPrevCatcode@CBF\undefined%
	\catcode216=\@pytexPrevCatcode@CBG%
	\let\@pytexPrevCatcode@CBG\undefined%
	\catcode217=\@pytexPrevCatcode@CBH%
	\let\@pytexPrevCatcode@CBH\undefined%
	\catcode218=\@pytexPrevCatcode@CBI%
	\let\@pytexPrevCatcode@CBI\undefined%
	\catcode219=\@pytexPrevCatcode@CBJ%
	\let\@pytexPrevCatcode@CBJ\undefined%
	\catcode220=\@pytexPrevCatcode@CCA%
	\let\@pytexPrevCatcode@CCA\undefined%
	\catcode221=\@pytexPrevCatcode@CCB%
	\let\@pytexPrevCatcode@CCB\undefined%
	\catcode222=\@pytexPrevCatcode@CCC%
	\let\@pytexPrevCatcode@CCC\undefined%
	\catcode223=\@pytexPrevCatcode@CCD%
	\let\@pytexPrevCatcode@CCD\undefined%
	\catcode224=\@pytexPrevCatcode@CCE%
	\let\@pytexPrevCatcode@CCE\undefined%
	\catcode225=\@pytexPrevCatcode@CCF%
	\let\@pytexPrevCatcode@CCF\undefined%
	\catcode226=\@pytexPrevCatcode@CCG%
	\let\@pytexPrevCatcode@CCG\undefined%
	\catcode227=\@pytexPrevCatcode@CCH%
	\let\@pytexPrevCatcode@CCH\undefined%
	\catcode228=\@pytexPrevCatcode@CCI%
	\let\@pytexPrevCatcode@CCI\undefined%
	\catcode229=\@pytexPrevCatcode@CCJ%
	\let\@pytexPrevCatcode@CCJ\undefined%
	\catcode230=\@pytexPrevCatcode@CDA%
	\let\@pytexPrevCatcode@CDA\undefined%
	\catcode231=\@pytexPrevCatcode@CDB%
	\let\@pytexPrevCatcode@CDB\undefined%
	\catcode232=\@pytexPrevCatcode@CDC%
	\let\@pytexPrevCatcode@CDC\undefined%
	\catcode233=\@pytexPrevCatcode@CDD%
	\let\@pytexPrevCatcode@CDD\undefined%
	\catcode234=\@pytexPrevCatcode@CDE%
	\let\@pytexPrevCatcode@CDE\undefined%
	\catcode235=\@pytexPrevCatcode@CDF%
	\let\@pytexPrevCatcode@CDF\undefined%
	\catcode236=\@pytexPrevCatcode@CDG%
	\let\@pytexPrevCatcode@CDG\undefined%
	\catcode237=\@pytexPrevCatcode@CDH%
	\let\@pytexPrevCatcode@CDH\undefined%
	\catcode238=\@pytexPrevCatcode@CDI%
	\let\@pytexPrevCatcode@CDI\undefined%
	\catcode239=\@pytexPrevCatcode@CDJ%
	\let\@pytexPrevCatcode@CDJ\undefined%
	\catcode240=\@pytexPrevCatcode@CEA%
	\let\@pytexPrevCatcode@CEA\undefined%
	\catcode241=\@pytexPrevCatcode@CEB%
	\let\@pytexPrevCatcode@CEB\undefined%
	\catcode242=\@pytexPrevCatcode@CEC%
	\let\@pytexPrevCatcode@CEC\undefined%
	\catcode243=\@pytexPrevCatcode@CED%
	\let\@pytexPrevCatcode@CED\undefined%
	\catcode244=\@pytexPrevCatcode@CEE%
	\let\@pytexPrevCatcode@CEE\undefined%
	\catcode245=\@pytexPrevCatcode@CEF%
	\let\@pytexPrevCatcode@CEF\undefined%
	\catcode246=\@pytexPrevCatcode@CEG%
	\let\@pytexPrevCatcode@CEG\undefined%
	\catcode247=\@pytexPrevCatcode@CEH%
	\let\@pytexPrevCatcode@CEH\undefined%
	\catcode248=\@pytexPrevCatcode@CEI%
	\let\@pytexPrevCatcode@CEI\undefined%
	\catcode249=\@pytexPrevCatcode@CEJ%
	\let\@pytexPrevCatcode@CEJ\undefined%
	\catcode250=\@pytexPrevCatcode@CFA%
	\let\@pytexPrevCatcode@CFA\undefined%
	\catcode251=\@pytexPrevCatcode@CFB%
	\let\@pytexPrevCatcode@CFB\undefined%
	\catcode252=\@pytexPrevCatcode@CFC%
	\let\@pytexPrevCatcode@CFC\undefined%
	\catcode253=\@pytexPrevCatcode@CFD%
	\let\@pytexPrevCatcode@CFD\undefined%
	\catcode254=\@pytexPrevCatcode@CFE%
	\let\@pytexPrevCatcode@CFE\undefined%
	\catcode255=\@pytexPrevCatcode@CFF%
	\let\@pytexPrevCatcode@CFF\undefined%
}


\def\swap#1#2{#2{#1}}
\def\ExpandAfter#1#2{%
	\e\swap\e{#2}{#1}%
}

% define a new local variable name
\def\@pytexLocal@new#1{%
	\@pytexStack@new{#1@stack}%
}

\def\@pytexLocal@delete#1{%
	\@pytexStack@delete{#1@stack}%
}

% allocate a new variable
\def\@pytexLocal@begin#1{%
	\e\let\e\@pytexTMP@localvar\csname#1\endcsname%
	\ExpandAfter{\@pytexStack@push{#1@stack}}{\@pytexTMP@localvar}%
}

% release allocated variable
\def\@pytexLocal@end#1{%
	\@pytexStack@pop{#1@stack}{\@pytexTMP@localvar}%
	\e\let\csname#1\endcsname\@pytexTMP@localvar%
}

% DO NOT EDIT THIS FILE DIRECTLY
% This file was generated by 'gen_token.py'

% --- BEGIN tokeniser.tex copy ---

\newcount\state
\state=5

\def\@pytexTokeniser@buffer{}
\@pytexList@new{@pytexTokenList}

\newcount\currentindent
\currentindent=0
\@pytexStack@new{indentStack}
\indentStack@push{0}


\def\STR_python{python}
\def\STR_CloseBrace{CloseBrace}
\def\STR_end{end}

\def\@pytexTokeniser@checkEnd{%
	\@pytexTokenList@getendim{0}%
	\ifnum\tokentype=\tokentypeIDENTIFIER% python
		\ifx\tokenvalue\STR_python%
		
			\@pytexTokenList@getendim{1}%
			\ifnum\tokentype=\tokentypeLBRACE% {
			
				\@pytexTokenList@getendim{2}%
				\ifnum\tokentype=\tokentypeIDENTIFIER% end
					\ifx\tokenvalue\STR_end%
					
						\@pytexTokenList@getendim{3}%
						\ifnum\tokentype=\tokentypeBACKSLASH% \
						
							\ifx\@pytexTokeniser@buffer\STR_CloseBrace% }
								\@pytexTokeniser@endsymboltoken%
								\@pytexTokenList@popd% }
								\@pytexTokenList@popd% python
								\@pytexTokenList@popd% {
								\@pytexTokenList@popd% end
								\@pytexTokenList@popd% \

								\end{python}%
							\fi%
						\fi%
					\fi%
				\fi%
			\fi%
		\fi%
	\fi%
}


\def\@pytexTokeniser@endnumbertoken{
	\ifnum\state=2
		\e\@pytexTokenList@push\e{\e\@pytexToken@Number\e{\@pytexTokeniser@buffer}}
		\gdef\@pytexTokeniser@buffer{}
		\state=0
	\fi
}

\def\@pytexTokeniser@dedent{
	\ifnum\currentindent<\lastindent
		\@pytexTokenList@push{\@pytexToken@Dedent}
		\indentStack@popd
		\indentStack@peek{\lastindent}
		\ifnum\currentindent>\lastindent
			\@pytexError{IndentationError: unindent does not match any outer indentation level}
		\fi
		\@pytexTokeniser@dedent
	\fi
}

\def\@pytexTokeniser@endindent{
	\ifnum\state=5
		\indentStack@peek{\lastindent}
		\ifnum\currentindent>\lastindent
			\e\indentStack@push\e{\the\currentindent}
			\edef\lastindent{\the\currentindent}
			\@pytexTokenList@push{\@pytexToken@Indent}
		\fi
		\@pytexTokeniser@dedent
		\currentindent=0
		\state=0
	\fi
}



\def\@pytexTokeniser@letter#1{
	\@pytexTokeniser@endindent
	\@pytexTokeniser@endsymboltoken
	\ifnum\state=4\else	\ifnum\state=0\else\ifnum\state=1\else \@pytexError{Internal error: starting to tokenise name when not in 0 state.}\fi\fi
		\xdef\@pytexTokeniser@buffer{\@pytexTokeniser@buffer #1}
		\state=1
	\fi
}

\def\@pytexTokeniser@number#1{
	\@pytexTokeniser@endindent
	\@pytexTokeniser@endsymboltoken
	\ifnum\state=4\else	\ifnum\state=1
		\@pytexTokeniser@letter{#1}
	\else
		\ifnum\state=0\else\ifnum\state=2\else \@pytexError{Internal error: starting to tokenise number when not in 0 state.}\fi\fi
		\xdef\@pytexTokeniser@buffer{\@pytexTokeniser@buffer #1}
		\state=2
	\fi
\fi}

\def\@pytexTokeniser@symbol#1{
	\@pytexTokeniser@endnametoken
	\@pytexTokeniser@endnumbertoken
	\@pytexTokeniser@endindent
	\ifnum\state=4\else	\ifnum\state=0\else\ifnum\state=3\else \@pytexError{Internal error: starting to symbols name when not in 0 state.}\fi\fi
	\xdef\@pytexTokeniser@buffer{\@pytexTokeniser@buffer #1}
	\state=3
	\fi
}




\def\@pytexChar@Newline{
    \@pytexTokeniser@endnametoken
    \@pytexTokeniser@endnumbertoken
    \@pytexTokeniser@endindent
    \@pytexTokeniser@endsymboltoken
    \ifnum\state=0\else\ifnum\state=4\else \@pytexError{Internal error: starting to tokenise comment when not in 0 state.}\fi\fi
    \state=5
    \@pytexTokenList@push{\@pytexToken@Newline}
}

\def\@pytexChar@Space{
    \@pytexTokeniser@endnametoken
    \@pytexTokeniser@endnumbertoken
    \@pytexTokeniser@endsymboltoken
    \ifnum\state=5
        \advance\currentindent by 1
    \fi
}

\def\@pytexChar@Tab{
    \@pytexTokeniser@endnametoken
    \@pytexTokeniser@endnumbertoken
    \@pytexTokeniser@endsymboltoken
    \ifnum\state=5
        \advance\currentindent by 1
	\fi
}

\def\@pytexChar@Hash{
\ifnum\state=4\else
    \@pytexTokeniser@endnametoken
    \@pytexTokeniser@endnumbertoken
    \@pytexTokeniser@endsymboltoken
    \@pytexTokeniser@endindent
    \ifnum\state=0\else \@pytexError{Internal error: starting comment when not in 0 state.}\fi
    \state=4
\fi
}
% --- END tokeniser.tex copy ---

\def\@pytexTokeniser@endnametoken{
		\ifnum\state=1
		\def\@pytexTMP@keyword{False}
		\ifx\@pytexTokeniser@buffer\@pytexTMP@keyword
			\@pytexTokenList@push{\@pytexToken@False}
		\else
		\def\@pytexTMP@keyword{None}
		\ifx\@pytexTokeniser@buffer\@pytexTMP@keyword
			\@pytexTokenList@push{\@pytexToken@None}
		\else
		\def\@pytexTMP@keyword{True}
		\ifx\@pytexTokeniser@buffer\@pytexTMP@keyword
			\@pytexTokenList@push{\@pytexToken@True}
		\else
		\def\@pytexTMP@keyword{and}
		\ifx\@pytexTokeniser@buffer\@pytexTMP@keyword
			\@pytexTokenList@push{\@pytexToken@and}
		\else
		\def\@pytexTMP@keyword{as}
		\ifx\@pytexTokeniser@buffer\@pytexTMP@keyword
			\@pytexTokenList@push{\@pytexToken@as}
		\else
		\def\@pytexTMP@keyword{assert}
		\ifx\@pytexTokeniser@buffer\@pytexTMP@keyword
			\@pytexTokenList@push{\@pytexToken@assert}
		\else
		\def\@pytexTMP@keyword{async}
		\ifx\@pytexTokeniser@buffer\@pytexTMP@keyword
			\@pytexTokenList@push{\@pytexToken@async}
		\else
		\def\@pytexTMP@keyword{await}
		\ifx\@pytexTokeniser@buffer\@pytexTMP@keyword
			\@pytexTokenList@push{\@pytexToken@await}
		\else
		\def\@pytexTMP@keyword{break}
		\ifx\@pytexTokeniser@buffer\@pytexTMP@keyword
			\@pytexTokenList@push{\@pytexToken@break}
		\else
		\def\@pytexTMP@keyword{class}
		\ifx\@pytexTokeniser@buffer\@pytexTMP@keyword
			\@pytexTokenList@push{\@pytexToken@class}
		\else
		\def\@pytexTMP@keyword{continue}
		\ifx\@pytexTokeniser@buffer\@pytexTMP@keyword
			\@pytexTokenList@push{\@pytexToken@continue}
		\else
		\def\@pytexTMP@keyword{def}
		\ifx\@pytexTokeniser@buffer\@pytexTMP@keyword
			\@pytexTokenList@push{\@pytexToken@def}
		\else
		\def\@pytexTMP@keyword{del}
		\ifx\@pytexTokeniser@buffer\@pytexTMP@keyword
			\@pytexTokenList@push{\@pytexToken@del}
		\else
		\def\@pytexTMP@keyword{elif}
		\ifx\@pytexTokeniser@buffer\@pytexTMP@keyword
			\@pytexTokenList@push{\@pytexToken@elif}
		\else
		\def\@pytexTMP@keyword{else}
		\ifx\@pytexTokeniser@buffer\@pytexTMP@keyword
			\@pytexTokenList@push{\@pytexToken@else}
		\else
		\def\@pytexTMP@keyword{except}
		\ifx\@pytexTokeniser@buffer\@pytexTMP@keyword
			\@pytexTokenList@push{\@pytexToken@except}
		\else
		\def\@pytexTMP@keyword{finally}
		\ifx\@pytexTokeniser@buffer\@pytexTMP@keyword
			\@pytexTokenList@push{\@pytexToken@finally}
		\else
		\def\@pytexTMP@keyword{for}
		\ifx\@pytexTokeniser@buffer\@pytexTMP@keyword
			\@pytexTokenList@push{\@pytexToken@for}
		\else
		\def\@pytexTMP@keyword{from}
		\ifx\@pytexTokeniser@buffer\@pytexTMP@keyword
			\@pytexTokenList@push{\@pytexToken@from}
		\else
		\def\@pytexTMP@keyword{global}
		\ifx\@pytexTokeniser@buffer\@pytexTMP@keyword
			\@pytexTokenList@push{\@pytexToken@global}
		\else
		\def\@pytexTMP@keyword{if}
		\ifx\@pytexTokeniser@buffer\@pytexTMP@keyword
			\@pytexTokenList@push{\@pytexToken@if}
		\else
		\def\@pytexTMP@keyword{import}
		\ifx\@pytexTokeniser@buffer\@pytexTMP@keyword
			\@pytexTokenList@push{\@pytexToken@import}
		\else
		\def\@pytexTMP@keyword{in}
		\ifx\@pytexTokeniser@buffer\@pytexTMP@keyword
			\@pytexTokenList@push{\@pytexToken@in}
		\else
		\def\@pytexTMP@keyword{is}
		\ifx\@pytexTokeniser@buffer\@pytexTMP@keyword
			\@pytexTokenList@push{\@pytexToken@is}
		\else
		\def\@pytexTMP@keyword{lambda}
		\ifx\@pytexTokeniser@buffer\@pytexTMP@keyword
			\@pytexTokenList@push{\@pytexToken@lambda}
		\else
		\def\@pytexTMP@keyword{nonlocal}
		\ifx\@pytexTokeniser@buffer\@pytexTMP@keyword
			\@pytexTokenList@push{\@pytexToken@nonlocal}
		\else
		\def\@pytexTMP@keyword{not}
		\ifx\@pytexTokeniser@buffer\@pytexTMP@keyword
			\@pytexTokenList@push{\@pytexToken@not}
		\else
		\def\@pytexTMP@keyword{or}
		\ifx\@pytexTokeniser@buffer\@pytexTMP@keyword
			\@pytexTokenList@push{\@pytexToken@or}
		\else
		\def\@pytexTMP@keyword{pass}
		\ifx\@pytexTokeniser@buffer\@pytexTMP@keyword
			\@pytexTokenList@push{\@pytexToken@pass}
		\else
		\def\@pytexTMP@keyword{raise}
		\ifx\@pytexTokeniser@buffer\@pytexTMP@keyword
			\@pytexTokenList@push{\@pytexToken@raise}
		\else
		\def\@pytexTMP@keyword{return}
		\ifx\@pytexTokeniser@buffer\@pytexTMP@keyword
			\@pytexTokenList@push{\@pytexToken@return}
		\else
		\def\@pytexTMP@keyword{try}
		\ifx\@pytexTokeniser@buffer\@pytexTMP@keyword
			\@pytexTokenList@push{\@pytexToken@try}
		\else
		\def\@pytexTMP@keyword{while}
		\ifx\@pytexTokeniser@buffer\@pytexTMP@keyword
			\@pytexTokenList@push{\@pytexToken@while}
		\else
		\def\@pytexTMP@keyword{with}
		\ifx\@pytexTokeniser@buffer\@pytexTMP@keyword
			\@pytexTokenList@push{\@pytexToken@with}
		\else
		\def\@pytexTMP@keyword{yield}
		\ifx\@pytexTokeniser@buffer\@pytexTMP@keyword
			\@pytexTokenList@push{\@pytexToken@yield}
		\else
			\e\@pytexTokenList@push\e{\e\@pytexToken@Identifier\e{\@pytexTokeniser@buffer}}
		\fi\fi\fi\fi\fi\fi\fi\fi\fi\fi\fi\fi\fi\fi\fi\fi\fi\fi\fi\fi\fi\fi\fi\fi\fi\fi\fi\fi\fi\fi\fi\fi\fi\fi\fi
		\state=0
		\gdef\@pytexTokeniser@buffer{}
		\let\@pytexTMP@keyword\undefined
	\fi
}

\def\@pytexTokeniser@endsymboltoken{
	\ifnum\state=3
		\def\@pytexTMP@symbol{Plus}
		\ifx\@pytexTokeniser@buffer\@pytexTMP@symbol
			\@pytexTokenList@push{\@pytexToken@Addition}
		\else
		\def\@pytexTMP@symbol{Minus}
		\ifx\@pytexTokeniser@buffer\@pytexTMP@symbol
			\@pytexTokenList@push{\@pytexToken@Subtraction}
		\else
		\def\@pytexTMP@symbol{Asterisk}
		\ifx\@pytexTokeniser@buffer\@pytexTMP@symbol
			\@pytexTokenList@push{\@pytexToken@Multiplication}
		\else
		\def\@pytexTMP@symbol{ForwardSlash}
		\ifx\@pytexTokeniser@buffer\@pytexTMP@symbol
			\@pytexTokenList@push{\@pytexToken@Division}
		\else
		\def\@pytexTMP@symbol{ForwardSlashForwardSlash}
		\ifx\@pytexTokeniser@buffer\@pytexTMP@symbol
			\@pytexTokenList@push{\@pytexToken@FloorDivision}
		\else
		\def\@pytexTMP@symbol{Percent}
		\ifx\@pytexTokeniser@buffer\@pytexTMP@symbol
			\@pytexTokenList@push{\@pytexToken@Modulo}
		\else
		\def\@pytexTMP@symbol{AsteriskAsterisk}
		\ifx\@pytexTokeniser@buffer\@pytexTMP@symbol
			\@pytexTokenList@push{\@pytexToken@Exponentiation}
		\else
		\def\@pytexTMP@symbol{EqualsEquals}
		\ifx\@pytexTokeniser@buffer\@pytexTMP@symbol
			\@pytexTokenList@push{\@pytexToken@EqualTo}
		\else
		\def\@pytexTMP@symbol{ExclamationMarkEquals}
		\ifx\@pytexTokeniser@buffer\@pytexTMP@symbol
			\@pytexTokenList@push{\@pytexToken@NotEqualTo}
		\else
		\def\@pytexTMP@symbol{GreaterThan}
		\ifx\@pytexTokeniser@buffer\@pytexTMP@symbol
			\@pytexTokenList@push{\@pytexToken@GreaterThan}
		\else
		\def\@pytexTMP@symbol{LessThan}
		\ifx\@pytexTokeniser@buffer\@pytexTMP@symbol
			\@pytexTokenList@push{\@pytexToken@LessThan}
		\else
		\def\@pytexTMP@symbol{GreaterThanEquals}
		\ifx\@pytexTokeniser@buffer\@pytexTMP@symbol
			\@pytexTokenList@push{\@pytexToken@GreaterThanOrEqualTo}
		\else
		\def\@pytexTMP@symbol{LessThanEquals}
		\ifx\@pytexTokeniser@buffer\@pytexTMP@symbol
			\@pytexTokenList@push{\@pytexToken@LessThanOrEqualTo}
		\else
		\def\@pytexTMP@symbol{Ampersand}
		\ifx\@pytexTokeniser@buffer\@pytexTMP@symbol
			\@pytexTokenList@push{\@pytexToken@BitwiseAnd}
		\else
		\def\@pytexTMP@symbol{Pipe}
		\ifx\@pytexTokeniser@buffer\@pytexTMP@symbol
			\@pytexTokenList@push{\@pytexToken@BitwiseOr}
		\else
		\def\@pytexTMP@symbol{Caret}
		\ifx\@pytexTokeniser@buffer\@pytexTMP@symbol
			\@pytexTokenList@push{\@pytexToken@BitwiseXor}
		\else
		\def\@pytexTMP@symbol{Tilde}
		\ifx\@pytexTokeniser@buffer\@pytexTMP@symbol
			\@pytexTokenList@push{\@pytexToken@BitwiseNot}
		\else
		\def\@pytexTMP@symbol{LessThanLessThan}
		\ifx\@pytexTokeniser@buffer\@pytexTMP@symbol
			\@pytexTokenList@push{\@pytexToken@LeftShift}
		\else
		\def\@pytexTMP@symbol{GreaterThanGreaterThan}
		\ifx\@pytexTokeniser@buffer\@pytexTMP@symbol
			\@pytexTokenList@push{\@pytexToken@RightShift}
		\else
		\def\@pytexTMP@symbol{Equals}
		\ifx\@pytexTokeniser@buffer\@pytexTMP@symbol
			\@pytexTokenList@push{\@pytexToken@Assignment}
		\else
		\def\@pytexTMP@symbol{PlusEquals}
		\ifx\@pytexTokeniser@buffer\@pytexTMP@symbol
			\@pytexTokenList@push{\@pytexToken@AddAssign}
		\else
		\def\@pytexTMP@symbol{MinusEquals}
		\ifx\@pytexTokeniser@buffer\@pytexTMP@symbol
			\@pytexTokenList@push{\@pytexToken@SubAssign}
		\else
		\def\@pytexTMP@symbol{AsteriskEquals}
		\ifx\@pytexTokeniser@buffer\@pytexTMP@symbol
			\@pytexTokenList@push{\@pytexToken@MultAssign}
		\else
		\def\@pytexTMP@symbol{ForwardSlashEquals}
		\ifx\@pytexTokeniser@buffer\@pytexTMP@symbol
			\@pytexTokenList@push{\@pytexToken@DivAssign}
		\else
		\def\@pytexTMP@symbol{ForwardSlashForwardSlashEquals}
		\ifx\@pytexTokeniser@buffer\@pytexTMP@symbol
			\@pytexTokenList@push{\@pytexToken@FloorDivAssign}
		\else
		\def\@pytexTMP@symbol{PercentEquals}
		\ifx\@pytexTokeniser@buffer\@pytexTMP@symbol
			\@pytexTokenList@push{\@pytexToken@ModuloAssign}
		\else
		\def\@pytexTMP@symbol{AsteriskAsteriskEquals}
		\ifx\@pytexTokeniser@buffer\@pytexTMP@symbol
			\@pytexTokenList@push{\@pytexToken@ExpoAssign}
		\else
		\def\@pytexTMP@symbol{AmpersandEquals}
		\ifx\@pytexTokeniser@buffer\@pytexTMP@symbol
			\@pytexTokenList@push{\@pytexToken@BitwiseAndAssign}
		\else
		\def\@pytexTMP@symbol{PipeEquals}
		\ifx\@pytexTokeniser@buffer\@pytexTMP@symbol
			\@pytexTokenList@push{\@pytexToken@BitwiseOrAssign}
		\else
		\def\@pytexTMP@symbol{CaretEquals}
		\ifx\@pytexTokeniser@buffer\@pytexTMP@symbol
			\@pytexTokenList@push{\@pytexToken@BitwiseXorAssign}
		\else
		\def\@pytexTMP@symbol{LessThanLessThanEquals}
		\ifx\@pytexTokeniser@buffer\@pytexTMP@symbol
			\@pytexTokenList@push{\@pytexToken@LeftShiftAssign}
		\else
		\def\@pytexTMP@symbol{GreaterThanGreaterThanEquals}
		\ifx\@pytexTokeniser@buffer\@pytexTMP@symbol
			\@pytexTokenList@push{\@pytexToken@RightShiftAssign}
		\else
		\def\@pytexTMP@symbol{Colon}
		\ifx\@pytexTokeniser@buffer\@pytexTMP@symbol
			\@pytexTokenList@push{\@pytexToken@Colon}
		\else
		\def\@pytexTMP@symbol{Comma}
		\ifx\@pytexTokeniser@buffer\@pytexTMP@symbol
			\@pytexTokenList@push{\@pytexToken@Comma}
		\else
		\def\@pytexTMP@symbol{Dot}
		\ifx\@pytexTokeniser@buffer\@pytexTMP@symbol
			\@pytexTokenList@push{\@pytexToken@Dot}
		\else
		\def\@pytexTMP@symbol{OpenParen}
		\ifx\@pytexTokeniser@buffer\@pytexTMP@symbol
			\@pytexTokenList@push{\@pytexToken@OpeningParenthesis}
		\else
		\def\@pytexTMP@symbol{CloseParen}
		\ifx\@pytexTokeniser@buffer\@pytexTMP@symbol
			\@pytexTokenList@push{\@pytexToken@ClosingParenthesis}
		\else
		\def\@pytexTMP@symbol{OpenBracket}
		\ifx\@pytexTokeniser@buffer\@pytexTMP@symbol
			\@pytexTokenList@push{\@pytexToken@OpeningBracket}
		\else
		\def\@pytexTMP@symbol{CloseBracket}
		\ifx\@pytexTokeniser@buffer\@pytexTMP@symbol
			\@pytexTokenList@push{\@pytexToken@ClosingBracket}
		\else
		\def\@pytexTMP@symbol{OpenBrace}
		\ifx\@pytexTokeniser@buffer\@pytexTMP@symbol
			\@pytexTokenList@push{\@pytexToken@OpeningBrace}
		\else
		\def\@pytexTMP@symbol{CloseBrace}
		\ifx\@pytexTokeniser@buffer\@pytexTMP@symbol
			\@pytexTokenList@push{\@pytexToken@ClosingBrace}
		\else
		\def\@pytexTMP@symbol{SingleQuote}
		\ifx\@pytexTokeniser@buffer\@pytexTMP@symbol
			\@pytexTokenList@push{\@pytexToken@SingleQuote}
		\else
		\def\@pytexTMP@symbol{DoubleQuote}
		\ifx\@pytexTokeniser@buffer\@pytexTMP@symbol
			\@pytexTokenList@push{\@pytexToken@DoubleQuote}
		\else
		\def\@pytexTMP@symbol{Hash}
		\ifx\@pytexTokeniser@buffer\@pytexTMP@symbol
			\@pytexTokenList@push{\@pytexToken@Comment}
		\else
		\def\@pytexTMP@symbol{Semicolon}
		\ifx\@pytexTokeniser@buffer\@pytexTMP@symbol
			\@pytexTokenList@push{\@pytexToken@Semicolon}
		\else
		\def\@pytexTMP@symbol{Backslash}
		\ifx\@pytexTokeniser@buffer\@pytexTMP@symbol
			\@pytexTokenList@push{\@pytexToken@Backslash}
		\else
		\@pytexError{Tokeniser error: unknown symbol: \@pytexTokeniser@buffer}
		\fi\fi\fi\fi\fi\fi\fi\fi\fi\fi\fi\fi\fi\fi\fi\fi\fi\fi\fi\fi\fi\fi\fi\fi\fi\fi\fi\fi\fi\fi\fi\fi\fi\fi\fi\fi\fi\fi\fi\fi\fi\fi\fi\fi\fi\fi
		\state=0
		\gdef\@pytexTokeniser@buffer{}
		\let\@pytexTMP@symbol\undefined
	\fi
}

\def\@pytexChar@a{\@pytexTokeniser@letter{a}}
\def\@pytexChar@b{\@pytexTokeniser@letter{b}}
\def\@pytexChar@c{\@pytexTokeniser@letter{c}}
\def\@pytexChar@d{\@pytexTokeniser@letter{d}}
\def\@pytexChar@e{\@pytexTokeniser@letter{e}}
\def\@pytexChar@f{\@pytexTokeniser@letter{f}}
\def\@pytexChar@g{\@pytexTokeniser@letter{g}}
\def\@pytexChar@h{\@pytexTokeniser@letter{h}}
\def\@pytexChar@i{\@pytexTokeniser@letter{i}}
\def\@pytexChar@j{\@pytexTokeniser@letter{j}}
\def\@pytexChar@k{\@pytexTokeniser@letter{k}}
\def\@pytexChar@l{\@pytexTokeniser@letter{l}}
\def\@pytexChar@m{\@pytexTokeniser@letter{m}}
\def\@pytexChar@n{\@pytexTokeniser@letter{n}}
\def\@pytexChar@o{\@pytexTokeniser@letter{o}}
\def\@pytexChar@p{\@pytexTokeniser@letter{p}}
\def\@pytexChar@q{\@pytexTokeniser@letter{q}}
\def\@pytexChar@r{\@pytexTokeniser@letter{r}}
\def\@pytexChar@s{\@pytexTokeniser@letter{s}}
\def\@pytexChar@t{\@pytexTokeniser@letter{t}}
\def\@pytexChar@u{\@pytexTokeniser@letter{u}}
\def\@pytexChar@v{\@pytexTokeniser@letter{v}}
\def\@pytexChar@w{\@pytexTokeniser@letter{w}}
\def\@pytexChar@x{\@pytexTokeniser@letter{x}}
\def\@pytexChar@y{\@pytexTokeniser@letter{y}}
\def\@pytexChar@z{\@pytexTokeniser@letter{z}}
\def\@pytexChar@A{\@pytexTokeniser@letter{A}}
\def\@pytexChar@B{\@pytexTokeniser@letter{B}}
\def\@pytexChar@C{\@pytexTokeniser@letter{C}}
\def\@pytexChar@D{\@pytexTokeniser@letter{D}}
\def\@pytexChar@E{\@pytexTokeniser@letter{E}}
\def\@pytexChar@F{\@pytexTokeniser@letter{F}}
\def\@pytexChar@G{\@pytexTokeniser@letter{G}}
\def\@pytexChar@H{\@pytexTokeniser@letter{H}}
\def\@pytexChar@I{\@pytexTokeniser@letter{I}}
\def\@pytexChar@J{\@pytexTokeniser@letter{J}}
\def\@pytexChar@K{\@pytexTokeniser@letter{K}}
\def\@pytexChar@L{\@pytexTokeniser@letter{L}}
\def\@pytexChar@M{\@pytexTokeniser@letter{M}}
\def\@pytexChar@N{\@pytexTokeniser@letter{N}}
\def\@pytexChar@O{\@pytexTokeniser@letter{O}}
\def\@pytexChar@P{\@pytexTokeniser@letter{P}}
\def\@pytexChar@Q{\@pytexTokeniser@letter{Q}}
\def\@pytexChar@R{\@pytexTokeniser@letter{R}}
\def\@pytexChar@S{\@pytexTokeniser@letter{S}}
\def\@pytexChar@T{\@pytexTokeniser@letter{T}}
\def\@pytexChar@U{\@pytexTokeniser@letter{U}}
\def\@pytexChar@V{\@pytexTokeniser@letter{V}}
\def\@pytexChar@W{\@pytexTokeniser@letter{W}}
\def\@pytexChar@X{\@pytexTokeniser@letter{X}}
\def\@pytexChar@Y{\@pytexTokeniser@letter{Y}}
\def\@pytexChar@Z{\@pytexTokeniser@letter{Z}}
\def\@pytexChar@Underscore{\@pytexTokeniser@letter{_}}
\def\@pytexChar@Zero{\@pytexTokeniser@number{0}}
\def\@pytexChar@One{\@pytexTokeniser@number{1}}
\def\@pytexChar@Two{\@pytexTokeniser@number{2}}
\def\@pytexChar@Three{\@pytexTokeniser@number{3}}
\def\@pytexChar@Four{\@pytexTokeniser@number{4}}
\def\@pytexChar@Five{\@pytexTokeniser@number{5}}
\def\@pytexChar@Six{\@pytexTokeniser@number{6}}
\def\@pytexChar@Seven{\@pytexTokeniser@number{7}}
\def\@pytexChar@Eight{\@pytexTokeniser@number{8}}
\def\@pytexChar@Nine{\@pytexTokeniser@number{9}}
\def\@pytexChar@ExclamationMark{\@pytexTokeniser@symbol{ExclamationMark}}
\def\@pytexChar@Dollar{\@pytexTokeniser@symbol{Dollar}}
\def\@pytexChar@Percent{\@pytexTokeniser@symbol{Percent}}
\def\@pytexChar@Ampersand{\@pytexTokeniser@symbol{Ampersand}}
\def\@pytexChar@OpenParen{\@pytexTokeniser@symbol{OpenParen}}
\def\@pytexChar@CloseParen{\@pytexTokeniser@symbol{CloseParen}}
\def\@pytexChar@Asterisk{\@pytexTokeniser@symbol{Asterisk}}
\def\@pytexChar@Plus{\@pytexTokeniser@symbol{Plus}}
\def\@pytexChar@Comma{\@pytexTokeniser@symbol{Comma}}
\def\@pytexChar@Minus{\@pytexTokeniser@symbol{Minus}}
\def\@pytexChar@Dot{\@pytexTokeniser@symbol{Dot}}
\def\@pytexChar@ForwardSlash{\@pytexTokeniser@symbol{ForwardSlash}}
\def\@pytexChar@Colon{\@pytexTokeniser@symbol{Colon}}
\def\@pytexChar@Semicolon{\@pytexTokeniser@symbol{Semicolon}}
\def\@pytexChar@LessThan{\@pytexTokeniser@symbol{LessThan}}
\def\@pytexChar@Equals{\@pytexTokeniser@symbol{Equals}}
\def\@pytexChar@GreaterThan{\@pytexTokeniser@symbol{GreaterThan}}
\def\@pytexChar@QuestionMark{\@pytexTokeniser@symbol{QuestionMark}}
\def\@pytexChar@AtSymbol{\@pytexTokeniser@symbol{AtSymbol}}
\def\@pytexChar@OpenBracket{\@pytexTokeniser@symbol{OpenBracket}}
\def\@pytexChar@CloseBracket{\@pytexTokeniser@symbol{CloseBracket}}
\def\@pytexChar@Caret{\@pytexTokeniser@symbol{Caret}}
\def\@pytexChar@Underscore{\@pytexTokeniser@symbol{Underscore}}
\def\@pytexChar@Backtick{\@pytexTokeniser@symbol{Backtick}}
\def\@pytexChar@OpenBrace{\@pytexTokeniser@symbol{OpenBrace}}
\def\@pytexChar@Pipe{\@pytexTokeniser@symbol{Pipe}}
\def\@pytexChar@Tilde{\@pytexTokeniser@symbol{Tilde}}
\def\@pytexChar@Backslash{\@pytexTokeniser@symbol{Backslash}}
\def\@pytexChar@CloseBrace{\@pytexTokeniser@symbol{CloseBrace}\@pytexTokeniser@checkEnd}

% https://www.programiz.com/python-programming/precedence-associativity
% https://www.wscubetech.com/resources/python/precedence-associativity-operators
% https://docs.python.org/3/reference/expressions.html#grammar-token-python-grammar-primary

% try<thing> looks ahead and tries to parse <thing>.
% if parsing was succesful:
	% sets \@pytexTMP@parserReturnValue to the parsed construct
	% consumes the parsed tokens
	% ends with \iftrue
% else,
	% leaves \@pytexTMP@parserReturnValue untouched
	% leaves tokens untouched
	% ends with \iffalse
% use \wrapif{\try<thing>}{<iftrue>}{<iffalse>} to avoid problems with ifs

\def\@pytexParser@NotImplemented{\iffalse}%


% token type enum
\def\tokentypeEND{0}
\def\enumnext#1{%
	\e\edef\csname tokentype#1\endcsname{\tokentypeEND}%
	\edef\tokentypeEND{\the\numexpr\tokentypeEND+1}%
}

\enumnext{NEWLINE}

\enumnext{IDENTIFIER}
\enumnext{NUMBER}

\enumnext{AND}
\enumnext{OR}
\enumnext{NOT}
\enumnext{IS}
\enumnext{ISNOT}
\enumnext{IN}
\enumnext{NOTIN}

\enumnext{EXPONENTIATION}
\enumnext{INVERT}
\enumnext{MULTIPLY}
\enumnext{MATMULT}
\enumnext{FLOORDIVIDE}
\enumnext{DIVIDE}
\enumnext{MODULO}

\enumnext{ADD}
\enumnext{SUBTRACT}
\enumnext{SHIFTLEFT}
\enumnext{SHIFTRIGHT}

\enumnext{BITWISEAND}
\enumnext{BITWISEOR}
\enumnext{BITWISEXOR}

\enumnext{EQUAL}
\enumnext{NOTEQUAL}
\enumnext{LESSTHAN}
\enumnext{LESSEQUAL}
\enumnext{GREATER}
\enumnext{GREATEREQUAL}

\enumnext{ASSIGN}

% keywords
\enumnext{AWAIT}
\enumnext{IF}
\enumnext{ELSE}
\enumnext{ELIF}
\enumnext{FOR}
\enumnext{PASS}
\enumnext{BREAK}
\enumnext{CONTINUE}
\enumnext{RETURN}
\enumnext{TRUE}
\enumnext{FALSE}

% other
\enumnext{COMMA}
\enumnext{SEMICOLON}
\enumnext{COLON}
\enumnext{INDENT}
\enumnext{DEDENT}




% token macros - sets \toketype and, optionally, \tokenvalue
\def\@pytexToken@Newline{\edef\tokentype{\tokentypeNEWLINE}}
\def\@pytexToken@Identifier#1{\edef\tokentype{\tokentypeIDENTIFIER}\def\tokenvalue{#1}}
\def\@pytexToken@Number#1{\edef\tokentype{\tokentypeNUMBER}\def\tokenvalue{#1}}

\def\@pytexToken@and{\edef\tokentype{\tokentypeAND}}
\def\@pytexToken@or{\edef\tokentype{\tokentypeOR}}
\def\@pytexToken@in{\edef\tokentype{\tokentypeIN}}
\def\@pytexToken@is{\edef\tokentype{\tokentypeIS}}
\def\@pytexToken@not{\edef\tokentype{\tokentypeNOT}}

\def\@pytexToken@Addition{\edef\tokentype{\tokentypeADD}}
\def\@pytexToken@Subtraction{\edef\tokentype{\tokentypeSUBTRACT}}
\def\@pytexToken@Multiplication{\edef\tokentype{\tokentypeMULTIPLY}}
\def\@pytexToken@Division{\edef\tokentype{\tokentypeDIVIDE}}
\def\@pytexToken@FloorDivision{\edef\tokentype{\tokentypeFLOORDIVIDE}}
\def\@pytexToken@EqualTo{\edef\tokentype{\tokentypeEQUAL}}
\def\@pytexToken@NotEqualTo{\edef\tokentype{\tokentypeNOTEQUAL}}
\def\@pytexToken@Exponentiation{\edef\tokentype{\tokentypeEXPONENTIATION}}
\def\@pytexToken@Modulo{\edef\tokentype{\tokentypeMODULO}}

\def\@pytexToken@Assignment{\edef\tokentype{\tokentypeASSIGN}}
\def\@pytexToken@Indent{\edef\tokentype{\tokentypeINDENT}}
\def\@pytexToken@Dedent{\edef\tokentype{\tokentypeDEDENT}}
\def\@pytexToken@BitwiseOr{\edef\tokentype{\tokentypeBITWISEOR}}
\def\@pytexToken@BitwiseXor{\edef\tokentype{\tokentypeBITWISEXOR}}
\def\@pytexToken@BitwiseAnd{\edef\tokentype{\tokentypeBITWISEAND}}

\def\@pytexToken@if{\edef\tokentype{\tokentypeIF}}
\def\@pytexToken@else{\edef\tokentype{\tokentypeELSE}}
\def\@pytexToken@elif{\edef\tokentype{\tokentypeELIF}}
\def\@pytexToken@Colon{\edef\tokentype{\tokentypeCOLON}}
\def\@pytexToken@True{\edef\tokentype{\tokentypeTRUE}}
\def\@pytexToken@False{\edef\tokentype{\tokentypeFALSE}}




\def\ifmatch#1{%
	\e\@pytexTokeniser@TokenList@getim\e{\currentidx}%
	\ifnum \tokentype = #1 %
		\edef\currentidx{\the\numexpr\currentidx+1}%
}
\def\ifpeek#1{%
	\e\@pytexTokeniser@TokenList@getim\e{\currentidx}%
	\ifnum \tokentype = #1 %
}



\def\currentidx{0}
\def\@pytexTMP@parserReturnValue{\undefined}
\@pytexStack@new{idxsavestack}
\@pytexStack@new{resultsavestack}

\def\savestate{%
	\e\idxsavestack@push\e{\currentidx}%
	\e\resultsavestack@push\e{\@pytexTMP@parserReturnValue}%
}
\def\restorestate{%
	\@pytexStack@pop{idxsavestack}{\currentidx}%
	\@pytexStack@pop{resultsavestack}{\@pytexTMP@parserReturnValue}%
}
\def\commitstate{%
	\@pytexStack@popd{idxsavestack}%
	\@pytexStack@popd{resultsavestack}%
}



% #1 = if, #2 = iftrue, #3 = iffalse
\def\wrapif#1#2#3{%
	#1 %
		#2 %
	\else %
		#3 %
	\fi %
}

% #1 = iftrue, #2 = if, #3 = iffalse
\def\wrapifsw#1#2#3{%
	#2 %
		#1 %
	\else %
		#3 %
	\fi %
}

% result ::= #1, parserReturnValue is left untouched
\def\@pytexParser@tryToken#1{%
	\ifmatch{#1}%
}

% result ::= #1
\def\@pytexParser@tryTokenValued#1{%
	\ifmatch{#1}%
		\e\def\e\@pytexTMP@parserReturnValue\e{\tokenvalue}%
}

% result ::= #1 | #2
\def\@pytexParser@choice#1#2{%
	\wrapif{#1}{%
		\def\ifresult{1}%
	}{%
		\wrapif{#2}{%
			\def\ifresult{1}%
		}{%
			\def\ifresult{0}%
		}%
	}%
	\ifnum\ifresult=1 %
}

% result ::= #1 | #2 | #3
\def\@pytexParser@choiceThree#1#2#3{%
	\@pytexParser@choice{#1}{\@pytexParser@choice{#2}{#3}}%
}

% result ::= #1 | #2 | #3 | #4
\def\@pytexParser@choiceFour#1#2#3#4{%
	\@pytexParser@choice{#1}{\@pytexParser@choiceThree{#2}{#3}{#4}}%
}

% result ::= #1 | #2 | #3 | #4 | #5
\def\@pytexParser@choiceFive#1#2#3#4#5{%
	\@pytexParser@choice{#1}{\@pytexParser@choiceFour{#2}{#3}{#4}{#5}}%
}

% result ::= #1 | #2 | #3 | #4 | #5 | #6
\def\@pytexParser@choiceSix#1#2#3#4#5#6{%
	\@pytexParser@choice{#1}{\@pytexParser@choiceFive{#2}{#3}{#4}{#5}{#6}}%
}


\def\@pytexParser@tryIdentifier{%
	\@pytexParser@tryToken{\tokentypeIDENTIFIER}%
		\e\def\e\@pytexTMP@parserReturnValue\e{\tokenvalue}%
}



% #1 = first thing, #2 = second thing, #3 = joining operation (results of #1 and #2 are in \first and \second)
% result ::= #1 #2
\def\@pytexParser@join#1#2#3{%
	\savestate%
	\wrapif{#1}{%
		\e\wrapifsw\e{%
			\e\def\e\first\e{\@pytexTMP@parserReturnValue}%
			\let\second\@pytexTMP@parserReturnValue%
			#3%
			\def\ifresult{1}%
			\commitstate%
		}{#2}{%
			\def\ifresult{0}%
			\restorestate%
		}%
	}{%
		\def\ifresult{0}%
		\restorestate%
	}%
	\ifnum\ifresult=1 %
}

% similar to \@pytexParser@join, but \@pytexTMP@parserReturnValue is automatically set to the return value of #1
\def\@pytexParser@joinFirst#1#2{%
	\savestate%
	\wrapif{#1}{%
		\e\wrapifsw\e{\e\def\e\@pytexTMP@parserReturnValue\e{\@pytexTMP@parserReturnValue}%
			\commitstate%
			\def\ifresult{1}%
		}{#2}{%
			\restorestate%
			\def\ifresult{0}%
		}%
	}{%
		\restorestate%
		\def\ifresult{0}%
	}%
	\ifnum\ifresult=1 %
}

% similar to \@pytexParser@join, but \@pytexTMP@parserReturnValue is automatically set to the return value of #2
\def\@pytexParser@joinSecond#1#2{%
	\savestate%
	\wrapif{#1}{% TODO: this could mangle parserReturnValue
		\wrapif{#2}{%
			\commitstate%
			\def\ifresult{1}%
		}{%
			\restorestate%
			\def\ifresult{0}%
		}%
	}{%
		\restorestate%
		\def\ifresult{0}%
	}%
	\ifnum\ifresult=1 %
}

% similar to \@pytexParser@join, but \@pytexTMP@parserReturnValue is set to #3 (given \@pytexTMP@parserReturnValue as result of #1)
\def\@pytexParser@joinFirstUnary#1#2#3{%
	\savestate%
	\wrapif{#1}{%
		\e\wrapifsw\e{%
			\e\def\e\@pytexTMP@parserReturnValue\e{\@pytexTMP@parserReturnValue}%
			#3%
			\commitstate%
			\def\ifresult{1}%
		}{#2}{%
			\restorestate%
			\def\ifresult{0}%
		}%
	}{%
		\restorestate%
		\def\ifresult{0}%
	}%
	\ifnum\ifresult=1 %
}

% similar to \@pytexParser@join, but \@pytexTMP@parserReturnValue is set to #3 (given \@pytexTMP@parserReturnValue as result of #2)
\def\@pytexParser@joinSecondUnary#1#2#3{%
	\savestate%
	\wrapif{#1}{%
		\wrapif{#2}{%
			#3%
			\commitstate%
			\def\ifresult{1}%
		}{%
			\restorestate%
			\def\ifresult{0}%
		}%
	}{%
		\restorestate%
		\def\ifresult{0}%
	}%
	\ifnum\ifresult=1 %
}

% #1 = first thing, #2 = second thing, #3 = joining operation (provided with \first and \second), only executed when #2 is present
% result ::= #1 [#2]	-> {if(#2): #3}
\def\@pytexParser@joinOptional#1#2#3{%
	\savestate%
	\wrapif{#1}{%
		\e\wrapifsw\e{%
			\e\def\e\first\e{\@pytexTMP@parserReturnValue}%
			\let\second\@pytexTMP@parserReturnValue%
			#3%
			\def\ifresult{1}%
			\commitstate%
		}{#2}{%
			\def\ifresult{1}%
			\commitstate%
		}%
	}{%
		\def\ifresult{0}%
		\restorestate%
	}%
	\ifnum\ifresult=1 %
}


% #1 = first thing, #2 = second thing, #3 = single operation, #4 = double joining operation
% result ::= #1 [#2]	-> {if(#2): #4 else: #3}
\def\@pytexParser@joinOptionalDouble#1#2#3#4{%
	\savestate%
	\wrapif{#1}{%
		\@pytexLocal@begin{current}%
		\e\def\e\current\e{\@pytexTMP@parserReturnValue}%
		\wrapif{#2}{%
			\let\first\current%
			\let\second\@pytexTMP@parserReturnValue%
			#4%
			\def\ifresult{1}%
			\commitstate%
		}{%
			\let\first\current%
			#3%
			\def\ifresult{1}%
			\commitstate%
		}%
		\@pytexLocal@end{current}%
	}{%
		\def\ifresult{0}%
		\restorestate%
	}%
	\ifnum\ifresult=1 %
}





% await_primary:
%     | 'await' primary 
%     | primary
\def\@pytexParser@tryAwaitPrimary{%
    \@pytexParser@choice%
    {%
        \@pytexParser@join%
            {\@pytexParser@tryToken{\tokentypeAWAIT}}%
            {\@pytexParser@tryPrimary}%
            {\e\def\e\@pytexParserReturn\e{\e\@pytexOperator@await\e{\second}}}%
    }%
    {\@pytexParser@tryPrimary}%
}

% primary:
% TODO | primary '.' NAME 
% TODO | primary genexp 
% TODO | primary '(' [arguments] ')' 
% TODO | primary '[' slices ']' 
%      | atom
\def\@pytexParser@tryPrimary{%
	\@pytexParser@tryAtom%
}


% slices:
%     | slice !',' 
%     | ','.(slice | starred_expression)+ [','] 
% TODO


% slice:
%     | [expression] ':' [expression] [':' [expression] ] 
%     | named_expression 
% TODO


\def\@pytexParser@tryInt{%
	\@pytexParser@tryTokenValued{\tokentypeNUMBER}%
		\e\def\e\@pytexParserReturn\e{\e\@pytexRuntime@int\e{\tokenvalue}}%
}

\def\@pytexParser@tryIdentifierExpr{%
	\@pytexParser@tryIdentifier%
		\e\def\e\@pytexParserReturn\e{\e\@pytexRuntime@GetIdentifier\e{\tokenvalue}}%
}

\def\@pytexParser@tryBool{%
	\@pytexParser@choice%
	{\@pytexParser@tryToken{\tokentypeTRUE}%
		\def\@pytexParserReturn{\@pytexRuntime@boolTrue}}%
	{\@pytexParser@tryToken{\tokentypeFALSE}%
		\def\@pytexParserReturn{\@pytexRuntime@boolFalse}}%
}

% atom:
%     | NAME
%     | 'True'
%     | 'False'
%TODO | 'None'
%TODO | strings
%     | NUMBER
%TODO | (tuple | group | genexp)
%TODO | (list | listcomp)
%TODO | (dict | set | dictcomp | setcomp)
%TODO | '...'
\def\@pytexParser@tryAtom{%
	\@pytexParser@choiceFour%
		{\@pytexParser@tryIdentifierExpr}%
		{\@pytexParser@tryInt}%
		{\@pytexParser@tryBool}%
		{\@pytexParser@tryGroup}%
}


% group:
%     | '(' (yield_expr | named_expression) ')' 
\def\@pytexParser@tryGroup{%
	\@pytexParser@joinSecond%
		{\@pytexParser@tryToken{\tokentypeLPAREN}}%
		{\@pytexParser@joinFirst%
			{\@pytexParser@choice{\@pytexParser@tryYieldExpr}{\@pytexParser@tryNamedExpression}}%
			{\@pytexParser@tryToken{\tokentypeRPAREN}}%
		}%
}








% sum:
%     | sum '+' term 
%     | sum '-' term 
%     | term
\def\@pytexParser@trySum{%
	\@pytexParser@tryTerm%
		\@pytexLocal@begin{current}%
		\let\current\@pytexParserReturn%
		\@pytexParser@trySumNext%
		\let\@pytexParserReturn\current%
		\@pytexLocal@end{current}%
}
\def\@pytexParser@trySumNext{%
	\@pytexParser@choice%
		{%
		\@pytexParser@join{\@pytexParser@tryToken{\tokentypeADD}}{\@pytexParser@tryTerm}%
			{\e\e\e\def\e\e\e\current\e\e\e{\e\e\e\@pytexOperator@add\e\e\e{\e\current\e}\e{\@pytexParserReturn}}}%
		}{%
		\@pytexParser@join{\@pytexParser@tryToken{\tokentypeSUBTRACT}}{\@pytexParser@tryTerm}%
			{\e\e\e\def\e\e\e\current\e\e\e{\e\e\e\@pytexOperator@subtract\e\e\e{\e\current\e}\e{\@pytexParserReturn}}}%
		}%
		\def\next{\@pytexParser@trySumNext}%
	\else%
		\def\next{\relax}%
	\fi%
	\next%
}




% term:
%     | term '*' factor 
%     | term '/' factor 
%     | term '//' factor 
%     | term '%' factor 
%     | term '@' factor 
%     | factor
\def\@pytexParser@tryTerm{%
	\@pytexParser@tryFactor%
		\@pytexLocal@begin{current}%
		\let\current\@pytexParserReturn%
		\@pytexParser@tryTermNext%
		\let\@pytexParserReturn\current%
		\@pytexLocal@end{current}%
}
\def\@pytexParser@tryTermNext{%
	\@pytexParser@choiceFive%
		{%
		\@pytexParser@join{\@pytexParser@tryToken{\tokentypeMULTIPLY}}{\@pytexParser@tryFactor}%
			{\e\e\e\def\e\e\e\current\e\e\e{\e\e\e\@pytexOperator@multiply\e\e\e{\e\current\e}\e{\@pytexParserReturn}}}%
		}{%
		\@pytexParser@join{\@pytexParser@tryToken{\tokentypeMATMULT}}{\@pytexParser@tryFactor}%
			{\e\e\e\def\e\e\e\current\e\e\e{\e\e\e\@pytexOperator@matmult\e\e\e{\e\current\e}\e{\@pytexParserReturn}}}%
		}{%
		\@pytexParser@join{\@pytexParser@tryToken{\tokentypeFLOORDIVIDE}}{\@pytexParser@tryFactor}%
			{\e\e\e\def\e\e\e\current\e\e\e{\e\e\e\@pytexOperator@floordivide\e\e\e{\e\current\e}\e{\@pytexParserReturn}}}%
		}{%
		\@pytexParser@join{\@pytexParser@tryToken{\tokentypeDIVIDE}}{\@pytexParser@tryFactor}%
			{\e\e\e\def\e\e\e\current\e\e\e{\e\e\e\@pytexOperator@divide\e\e\e{\e\current\e}\e{\@pytexParserReturn}}}%
		}{%
		\@pytexParser@join{\@pytexParser@tryToken{\tokentypeMODULO}}{\@pytexParser@tryFactor}%
			{\e\e\e\def\e\e\e\current\e\e\e{\e\e\e\@pytexOperator@modulo\e\e\e{\e\current\e}\e{\@pytexParserReturn}}}%
		}%
		\def\next{\@pytexParser@tryTermNext}%
	\else%
		\def\next{\relax}%
	\fi%
	\next%
}




% factor:
%     | '+' factor
%     | '-' factor
%     | '~' factor
%     | power
\def\@pytexParser@tryFactor{%
	\@pytexParser@choiceFour%
		{\@pytexParser@tryPower}% power
		{%
		\@pytexParser@join{\@pytexParser@tryToken{\tokentypeSUBTRACT}}{\@pytexParser@tryFactor}% "-" u_expr
			{\e\def\e\@pytexParserReturn\e{\e\@pytexOperator@negate\e{\second}}}%
		}{%
		\@pytexParser@join{\@pytexParser@tryToken{\tokentypeADD}}{\@pytexParser@tryFactor}% "+" u_expr
			{\e\def\e\@pytexParserReturn\e{\e\@pytexOperator@pos\e{\second}}}%
		}{%
		\@pytexParser@join{\@pytexParser@tryToken{\tokentypeINVERT}}{\@pytexParser@tryFactor}% "~" u_expr
			{\e\def\e\@pytexParserReturn\e{\e\@pytexOperator@invert\e{\second}}}%
		}%
}



% power:
%     | await_primary '**' factor 
%     | await_primary
\def\@pytexParser@tryPower{%
	\@pytexParser@joinOptional%
		{\@pytexParser@tryAwaitPrimary}
		{\@pytexParser@join{\@pytexParser@tryToken{\tokentypeEXPONENTIATION}}{\@pytexParser@tryFactor}{\let\@pytexParserReturn\second}}%
		{\e\e\e\def\e\e\e\@pytexParserReturn\e\e\e{\e\e\e\@pytexOperator@pow\e\e\e{\e\first\e}\e{\second}}}%
}




% bitwise_or:
%     | bitwise_or '|' bitwise_xor 
%     | bitwise_xor
\def\@pytexParser@tryBitwiseOr{%
	\@pytexParser@tryBitwiseXor%
		\@pytexLocal@begin{current}%
		\let\current\@pytexParserReturn%
		\@pytexParser@tryBitwiseOrNext%
		\let\@pytexParserReturn\current%
		\@pytexLocal@end{current}%
}
\def\@pytexParser@tryBitwiseOrNext{%
	\@pytexParser@join{\@pytexParser@tryToken{\tokentypeBITWISEOR}}{\@pytexParser@tryBitwiseXor}%
		{\e\e\e\def\e\e\e\current\e\e\e{\e\e\e\@pytexOperator@bitwiseor\e\e\e{\e\current\e}\e{\@pytexParserReturn}}}%
		\def\next{\@pytexParser@tryBitwiseOrNext}%
	\else%
		\def\next{\relax}%
	\fi%
	\next%
}



% bitwise_xor:
%     | bitwise_xor '^' bitwise_and 
%     | bitwise_and
\def\@pytexParser@tryBitwiseXor{%
	\@pytexParser@tryBitwiseAnd%
		\@pytexLocal@begin{current}%
		\let\current\@pytexParserReturn%
		\@pytexParser@tryBitwiseXorNext%
		\let\@pytexParserReturn\current%
		\@pytexLocal@end{current}%
}
\def\@pytexParser@tryBitwiseXorNext{%
	\@pytexParser@join{\@pytexParser@tryToken{\tokentypeBITWISEXOR}}{\@pytexParser@tryBitwiseAnd}%
		{\e\e\e\def\e\e\e\current\e\e\e{\e\e\e\@pytexOperator@bitwisexor\e\e\e{\e\current\e}\e{\@pytexParserReturn}}}%
		\def\next{\@pytexParser@tryBitwiseXorNext}%
	\else%
		\def\next{\relax}%
	\fi%
	\next%
}



% bitwise_and:
%     | bitwise_and '&' shift_expr 
%     | shift_expr
\def\@pytexParser@tryBitwiseAnd{%
	\@pytexParser@tryShiftExpr%
		\@pytexLocal@begin{current}%
		\let\current\@pytexParserReturn%
		\@pytexParser@tryBitwiseAndNext%
		\let\@pytexParserReturn\current%
		\@pytexLocal@end{current}%
}
\def\@pytexParser@tryBitwiseAndNext{%
	\@pytexParser@join{\@pytexParser@tryToken{\tokentypeBITWISEAND}}{\@pytexParser@tryShiftExpr}%
		{\e\e\e\def\e\e\e\current\e\e\e{\e\e\e\@pytexOperator@bitwiseand\e\e\e{\e\current\e}\e{\@pytexParserReturn}}}%
		\def\next{\@pytexParser@tryBitwiseAndNext}%
	\else%
		\def\next{\relax}%
	\fi%
	\next%
}



% shift_expr:
%     | shift_expr '<<' sum 
%     | shift_expr '>>' sum 
%     | sum
\def\@pytexParser@tryShiftExpr{%
	\@pytexParser@trySum%
		\@pytexLocal@begin{current}%
		\let\current\@pytexParserReturn%
		\@pytexParser@tryShiftExprNext%
		\let\@pytexParserReturn\current%
		\@pytexLocal@end{current}%
}
\def\@pytexParser@tryShiftExprNext{%
	\@pytexParser@choice%
		{%
		\@pytexParser@join{\@pytexParser@tryToken{\tokentypeSHIFTLEFT}}{\@pytexParser@trySum}%
			{\e\e\e\def\e\e\e\current\e\e\e{\e\e\e\@pytexOperator@shiftleft\e\e\e{\e\current\e}\e{\@pytexParserReturn}}}%
		}{%
		\@pytexParser@join{\@pytexParser@tryToken{\tokentypeSHIFTRIGHT}}{\@pytexParser@trySum}%
			{\e\e\e\def\e\e\e\current\e\e\e{\e\e\e\@pytexOperator@shiftright\e\e\e{\e\current\e}\e{\@pytexParserReturn}}}%
		}%
		\def\next{\@pytexParser@tryShiftExprNext}%
	\else%
		\def\next{\relax}%
	\fi%
	\next%
}





% comparison:
%     | bitwise_or compare_op_bitwise_or_pair+ 
%     | bitwise_or
\def\@pytexParser@tryComparison{% TODO
	\@pytexParser@tryBitwiseOr%
}


% compare_op_bitwise_or_pair:
%     | eq_bitwise_or
%     | noteq_bitwise_or
%     | lte_bitwise_or
%     | lt_bitwise_or
%     | gte_bitwise_or
%     | gt_bitwise_or
%     | notin_bitwise_or
%     | in_bitwise_or
%     | isnot_bitwise_or
%     | is_bitwise_or



% eq_bitwise_or:    '==' bitwise_or 
% noteq_bitwise_or: '!=' bitwise_or 
% lte_bitwise_or:   '<=' bitwise_or 
% lt_bitwise_or:    '<'  bitwise_or 
% gte_bitwise_or:   '>=' bitwise_or 
% gt_bitwise_or:    '>'  bitwise_or 

% in_bitwise_or:    'in'       bitwise_or 
% notin_bitwise_or: 'not' 'in' bitwise_or 
% is_bitwise_or:    'is'       bitwise_or 
% isnot_bitwise_or: 'is' 'not' bitwise_or 




% expression:
%     | disjunction 'if' disjunction 'else' expression 
%     | disjunction
%     | lambdef
\def\@pytexParser@tryExpression{%
	\@pytexParser@choiceThree%
		{\@pytexParser@NotImplemented}% TODO
		{\@pytexParser@tryDisjunction}%
		{\@pytexParser@NotImplemented}% TODO
}

% yield_expr:
%     | 'yield' 'from' expression 
%     | 'yield' [star_expressions] 
% TODO

% star_expressions:
%     | star_expression (',' star_expression )+ [','] 
%     | star_expression ',' 
%     | star_expression
\def\@pytexParser@tryStarExpressions{%
	\@pytexParser@joinOptional%
		{\@pytexParser@tryStarExpressionsBare}%
		{\@pytexParser@tryToken{\tokentypeCOMMA}}%
		{\e\def\e\@pytexTMP@parserReturnValue\e{\first}}%
}
\def\@pytexParser@tryStarExpressionsBare{%
	\@pytexParser@joinOptional{\@pytexParser@tryStarExpression}%
		{\@pytexParser@joinSecond{\@pytexParser@tryToken{\tokentypeCOMMA}}{\@pytexParser@tryStarExpressionsBare}}%
		{\e\e\e\def\e\e\e\@pytexTMP@parserReturnValue\e\e\e{\e\e\e\@pytexOperator@starExpressionList\e\e\et{\e\first\e}\e{\second}}}%
}


% star_expression:
%     | '*' bitwise_or 
%     | expression
\def\@pytexParser@tryStarExpression{%
	\@pytexParser@choice%
		{\@pytexParser@join%
			{\@pytexParser@tryToken{\tokentypeMULTIPLY}}%
			{\@pytexParser@tryBitwiseOr}%
			{\e\def\e\@pytexTMP@parserReturnValue\e{\e\@pytexOperator@unpack\e{\second}}}%
		}%
		{\@pytexParser@tryExpression}%
}


% star_named_expressions: ','.star_named_expression+ [','] 
% TODO

% star_named_expression:
%     | '*' bitwise_or 
%     | named_expression
% TODO

% assignment_expression:
%     | NAME ':=' ~ expression 
% TODO
\def\@pytexParser@tryAssignmentExpression{\iffalse }%

% named_expression:
%     | assignment_expression
%     | expression !':='
\def\@pytexParser@tryNamedExpression{%
	\@pytexParser@choice%
		{\@pytexParser@tryAssignmentExpression}%
		{\@pytexParser@tryExpression}% TODO: do we need !':=' ?
}


% disjunction:
%     | conjunction ('or' conjunction )+ 
%     | conjunction
\def\@pytexParser@tryDisjunction{%
	\@pytexParser@tryConjunction%
		\@pytexLocal@begin{current}%
		\e\def\e\current\e{\@pytexTMP@parserReturnValue}%
		%
		\@pytexParser@tryDisjunctionNext%
		%
		\e\def\e\@pytexTMP@parserReturnValue\e{\current}%
		\@pytexLocal@end{current}%
}
\def\@pytexParser@tryDisjunctionNext{%
	\@pytexParser@join{\@pytexParser@tryToken{\tokentypeOR}}{\@pytexParser@tryConjunction}%
		{\e\e\e\def\e\e\e\current\e\e\e{\e\e\e\@pytexOperator@or\e\e\e{\e\current\e}\e{\@pytexTMP@parserReturnValue}}}%
		\def\next{\@pytexParser@tryDisjunctionNext}%
	\else%
		\def\next{\relax}%
	\fi%
	\next%
}


% conjunction:
%     | inversion ('and' inversion )+ 
%     | inversion
\def\@pytexParser@tryConjunction{%
	\@pytexParser@tryInversion
		\@pytexLocal@begin{current}%
		\e\def\e\current\e{\@pytexTMP@parserReturnValue}%
		\@pytexParser@tryConjunctionNext%
		\e\def\e\@pytexTMP@parserReturnValue\e{\current}%
		\@pytexLocal@end{current}%
}
\def\@pytexParser@tryConjunctionNext{%
	\@pytexParser@join{\@pytexParser@tryToken{\tokentypeAND}}{\@pytexParser@tryInversion}%
		{\e\e\e\def\e\e\e\current\e\e\e{\e\e\e\@pytexOperator@and\e\e\e{\e\current\e}\e{\@pytexTMP@parserReturnValue}}}%
		\def\next{\@pytexParser@tryConjunctionNext}%
	\else%
		\def\next{\relax}%
	\fi%
	\next%
}



% inversion:
%     | 'not' inversion 
%     | comparison
\def\@pytexParser@tryInversion{%
	\@pytexParser@choice{\@pytexParser@tryComparison}%
		{%
			\@pytexParser@join{\@pytexParser@tryToken{\tokentypeNOT}} {\@pytexParser@tryInversion}%
				{\e\def\e\@pytexTMP@parserReturnValue\e{\e\@pytexOperator@not\e{\second}}}%
		}%
}

% https://docs.python.org/3/reference/simple_stmts.html



% expression_stmt ::= starred_expression | expression
\def\@pytexParser@tryExpressionStmt{% TODO
	\@pytexParser@tryExpression%
}

% assert_stmt ::= "assert" expression ["," expression]
\def\@pytexParser@tryAssertStmt{% TODO
	\iffalse%
}

% target ::= identifier
%            | "(" [target_list] ")" TODO
%            | "[" [target_list] "]" TODO
%            | attributeref 	TODO
%            | subscription 	TODO
%            | slicing 			TODO
%            | "*" target 		TODO
\def\@pytexParser@tryTarget{%
	\@pytexParser@tryIdentifier%
}

% target_list ::= target ("," target)* [","]
\def\@pytexParser@tryTargetList{%
	\@pytexParser@tryTargetListBare% target ("," target)*
		\wrapif{\@pytexParser@tryToken{\tokentypeCOMMA}}{}{}% [","] % TODO: technically could mangle parserReturnValue, should use \joinOptional instead
}
\def\@pytexParser@tryTargetListBare{%
	\@pytexParser@joinOptional{\@pytexParser@tryTarget}%
		{\@pytexParser@joinSecond{\@pytexParser@tryToken{\tokentypeCOMMA}}{\@pytexParser@tryTargetListBare}}%
		{\e\e\e\def\e\e\e\@pytexTMP@parserReturnValue\e\e\e{\e\e\e\@pytexOperator@targetList\e\e\et{\e\first\e}\e{\second}}}%
}

\def\@pytexParser@tryYieldExpr{\iffalse}% TODO

% assignment_stmt ::= (target_list "=")+ (starred_expression | yield_expression)
\def\@pytexParser@tryAssignmentStmt{% TODO
	\@pytexParser@join%
		{\@pytexParser@tryAsignmentList}%
		{\@pytexParser@choice{\@pytexParser@tryStarExpression}{\@pytexParser@tryYieldExpr}}%
		{\e\e\e\def\e\e\e\@pytexTMP@parserReturnValue\e\e\e{\e\e\e\@pytexOperator@assignment\e\e\e{\e\first\e}\e{\second}}}%
}
\def\@pytexParser@tryAsignmentList{% (target_list "=")+
	\@pytexParser@joinOptional%
		{\@pytexParser@joinFirst{\@pytexParser@tryTargetList}{\@pytexParser@tryToken{\tokentypeASSIGN}}}% target_list "="
		{\@pytexParser@tryAsignmentList}% repeat
		{\e\e\e\def\e\e\e\@pytexTMP@parserReturnValue\e\e\e{\e\e\e\@pytexOperator@assignmentList\e\e\et{\e\first\e}\e{\second}}}%
}


% augmented_assignment_stmt ::= augtarget augop (expression_list | yield_expression)
% augtarget                 ::= identifier | attributeref | subscription | slicing
% augop                     ::= "+=" | "-=" | "*=" | "@=" | "/=" | "//=" | "%=" | "**="
%                               | ">>=" | "<<=" | "&=" | "^=" | "|="
\def\@pytexParser@tryAugmentedAssignmentStmt{% TODO
	\iffalse%
}


% annotated_assignment_stmt ::= augtarget ":" expression
%                               ["=" (starred_expression | yield_expression)]
\def\@pytexParser@tryAnnotatedAssignmentStmt{% TODO
	\iffalse%
}

% pass_stmt ::= "pass"
\def\@pytexParser@tryPassStmt{%
	\@pytexParser@tryToken{\tokentypePASS}%
		\def\@pytexTMP@parserReturnValue{\@pytexOperator@pass}% TODO: does this make sense?
}

% del_stmt ::= "del" target_list
\def\@pytexParser@tryDelStmt{% TODO
	\iffalse%
}



% yield_stmt ::= yield_expression
\def\@pytexParser@tryYieldStmt{% TODO
	\iffalse%\@pytexParser@tryYieldExpr%
}

% raise_stmt ::= "raise" [expression ["from" expression]]
\def\@pytexParser@tryRaiseStmt{% TODO
	\iffalse%
}

% break_stmt ::= "break"
\def\@pytexParser@tryBreakStmt{%
	\@pytexParser@tryToken{\tokentypeBREAK}%
		\def\@pytexTMP@parserReturnValue{\@pytexOperator@break}%
}

% continue_stmt ::= "continue"
\def\@pytexParser@tryContinueStmt{%
	\@pytexParser@tryToken{\tokentypeCONTINUE}%
		\def\@pytexTMP@parserReturnValue{\@pytexOperator@continue}%
}

%
\def\@pytexParser@tryImportStmt{% WONT BE IMPLEMENTED
	\iffalse%
}

%
\def\@pytexParser@tryFutureStmt{% WONT BE IMPLEMENTED
	\iffalse%
}

% global_stmt ::= "global" identifier ("," identifier)*
\def\@pytexParser@tryGlobalStmt{% TODO
	\iffalse%
}

% nonlocal_stmt ::= "nonlocal" identifier ("," identifier)*
\def\@pytexParser@tryNonlocalStmt{% TODO
	\iffalse%
}

% type_stmt ::= 'type' identifier [type_params] "=" expression
\def\@pytexParser@tryTypeStmt{% TODO
	\iffalse%
}






% simple_stmt ::= expression_stmt
%               | assert_stmt
%               | assignment_stmt
%               | augmented_assignment_stmt
%               | annotated_assignment_stmt
%               | pass_stmt
%               | del_stmt
%               | return_stmt
%               | yield_stmt
%               | raise_stmt
%               | break_stmt
%               | continue_stmt
%               | import_stmt
%               | future_stmt
%               | global_stmt
%               | nonlocal_stmt
%               | type_stmt
\def\@pytexParser@trySimpleStmt{%
	\@pytexParser@choiceFour%
	{\@pytexParser@choiceFive%
		{\@pytexParser@tryAssertStmt}%
		{\@pytexParser@tryAssignmentStmt}%
		{\@pytexParser@tryAugmentedAssignmentStmt}%
		{\@pytexParser@tryAnnotatedAssignmentStmt}%
		{\@pytexParser@tryExpressionStmt}%
	}{\@pytexParser@choiceFive%
		{\@pytexParser@tryPassStmt}%
		{\@pytexParser@tryDelStmt}%
		{\@pytexParser@tryReturnStmt}%
		{\@pytexParser@tryYieldStmt}%
		{\@pytexParser@tryRaiseStmt}%
	}{\@pytexParser@choiceFive%
		{\@pytexParser@tryBreakStmt}%
		{\@pytexParser@tryContinueStmt}%
		{\@pytexParser@tryImportStmt}%
		{\@pytexParser@tryFutureStmt}%
		{\@pytexParser@tryGlobalStmt}%
	}{\@pytexParser@choice%
		{\@pytexParser@tryNonlocalStmt}%
		{\@pytexParser@tryTypeStmt}%
	}%
}

%%%%%%%%%%%%%%%%%%%%%%%%%%%%%%%%%%%%%%%%%%%%%%%%%%
%%%%%%%%%%%%%%%%%%%%%%%%%%%%%%%%%%%%%%%%%%%%%%%%%%
%%%%%%%%%%%%%%%%%%%%%%%%%%%%%%%%%%%%%%%%%%%%%%%%%%
%%%%%%%%%%%%%%%%% NEW STUFF %%%%%%%%%%%%%%%%%%%%%%
%%%%%%%%%%%%%%%%%%%%%%%%%%%%%%%%%%%%%%%%%%%%%%%%%%
%%%%%%%%%%%%%%%%%%%%%%%%%%%%%%%%%%%%%%%%%%%%%%%%%%
%%%%%%%%%%%%%%%%%%%%%%%%%%%%%%%%%%%%%%%%%%%%%%%%%%


% simple_stmts:
%     | simple_stmt !';' NEWLINE  # Not needed, there for speedup
%     | ';'.simple_stmt+ [';'] NEWLINE 
% TODO




% # NOTE: assignment MUST precede expression, else parsing a simple assignment
% # will throw a SyntaxError.
% simple_stmt:
%     | assignment
%     | type_alias
%     | star_expressions 
%     | return_stmt
%     | import_stmt
%     | raise_stmt
%     | 'pass' 
%     | del_stmt
%     | yield_stmt
%     | assert_stmt
%     | 'break' 
%     | 'continue' 
%     | global_stmt
%     | nonlocal_stmt
% TODO




% compound_stmt:
%     | function_def
%     | if_stmt
%     | class_def
%     | with_stmt
%     | for_stmt
%     | try_stmt
%     | while_stmt
%     | match_stmt
% TODO




% # NOTE: annotated_rhs may start with 'yield'; yield_expr must start with 'yield'
% assignment:
%     | NAME ':' expression ['=' annotated_rhs ] 
%     | ('(' single_target ')' 
%          | single_subscript_attribute_target) ':' expression ['=' annotated_rhs ] 
%     | (star_targets '=' )+ (yield_expr | star_expressions) !'=' [TYPE_COMMENT] 
%     | single_target augassign ~ (yield_expr | star_expressions) 
% TODO




% annotated_rhs: yield_expr | star_expressions
% TODO




% augassign:
%     | '+=' 
%     | '-=' 
%     | '*=' 
%     | '@=' 
%     | '/=' 
%     | '%=' 
%     | '&=' 
%     | '|=' 
%     | '^=' 
%     | '<<=' 
%     | '>>=' 
%     | '**=' 
%     | '//=' 
% TODO




%return_stmt:
%    | 'return' [star_expressions] 
\def\@pytexParser@tryReturnStmt{%
	\@pytexParser@joinOptionalDouble%
		{\@pytexParser@tryToken{\tokentypeRETURN}}%
		{\@pytexParser@tryStarExpressions}%
		{\def\@pytexTMP@parserReturnValue{\@pytexOperator@return{None}}} % TODO: actually return None
		{\e\def\e\@pytexTMP@parserReturnValue\e{\e\@pytexOperator@return\e{\second}}}%
}


% raise_stmt:
%     | 'raise' expression ['from' expression ] 
%     | 'raise' 
% TODO




% global_stmt: 'global' ','.NAME+ 
% TODO




% nonlocal_stmt: 'nonlocal' ','.NAME+ 
% TODO




% del_stmt:
%     | 'del' del_targets &(';' | NEWLINE) 
% TODO




% yield_stmt: yield_expr 
% TODO




% assert_stmt: 'assert' expression [',' expression ] 
% TODO




% import_stmt:
%     | import_name
%     | import_from
% TODO - wont implement (probably)





% statements: statement+ 
\def\@pytexParser@tryStatements{%
	\@pytexParser@joinOptional%
		{\@pytexParser@tryStatement}%
		{\@pytexParser@tryStatements}%
		{\e\e\e\def\e\e\e\@pytexTMP@parserReturnValue\e\e\e{\e\first\second}}%
}


% compound_stmt:
%     | function_def
%     | if_stmt 		TODO
%     | class_def 		TODO
%     | with_stmt 		TODO
%     | for_stmt 		TODO
%     | try_stmt 		TODO
%     | while_stmt 		TODO
%     | match_stmt 		TODO
\def\@pytexParser@tryCompoundStmt{%
	\@pytexParser@choice%
		{\@pytexParser@tryFunctionDef}%
		{\@pytexParser@tryIfStmt}%
}



% block:
%     | NEWLINE INDENT statements DEDENT 
%     | simple_stmts
\def\@pytexParser@tryBlock{%
	\@pytexParser@choice%
		{%
			\@pytexParser@joinSecond{\@pytexParser@tryToken{\tokentypeNEWLINE}}%
				{%
					\@pytexParser@joinSecond{\@pytexParser@tryToken{\tokentypeINDENT}}%
						{%
							\@pytexParser@joinFirst{\@pytexParser@tryStatements}{\@pytexParser@tryToken{\tokentypeDEDENT}}	%
						}%
				}%
		}%
		{\@pytexParser@trySimpleStmts}%
}



% decorators: ('@' named_expression NEWLINE )+ 
% TODO




% class_def:
%     | decorators class_def_raw 
%     | class_def_raw
% TODO




% class_def_raw:
%     | 'class' NAME [type_params] ['(' [arguments] ')' ] ':' block 
% TODO



 
% function_def:
%     | decorators function_def_raw TODO
%     | function_def_raw
\def\@pytexParser@tryFunctionDef{%
	\@pytexParser@tryFunctionDefRaw%
}




% function_def_raw:
%     | 'def' NAME [type_params] '(' [params] ')' ['->' expression ] ':' [func_type_comment] block 
%     | 'async' 'def' NAME [type_params] '(' [params] ')' ['->' expression ] ':' [func_type_comment] block 
\def\@pytexParser@tryFunctionDefRaw{%
	\iffalse % TODO
}




% if_stmt:
%     | 'if' named_expression ':' block elif_stmt
%     | 'if' named_expression ':' block [else_block]
\def\@pytexParser@tryIfStmt{%
	\@pytexParser@joinSecond%
		{\@pytexParser@tryToken{\tokentypeIF}}%
		{\@pytexParser@tryIfStmtBare}%
}

% if_stmt_bare:
%     | named_expression ':' block elif_stmt
%     | named_expression ':' block [else_block]
\def\@pytexParser@tryIfStmtBare{%
	\@pytexParser@join%
		{\@pytexParser@tryNamedExpression}% named_expression
		{\@pytexParser@joinSecond%
			{\@pytexParser@tryToken{\tokentypeCOLON}}% ':'
			{\@pytexParser@tryBlock}%
		}%
		{\e\e\e\def\e\e\e\@pytexTMP@parserReturnValue\e\e\e{\e\e\e{\e\first\e}\e{\second}}} % format as macro arguments, to add \operatorIf or \operatorIfElse later
		%
		% test whether we have an if-else (or if-elif), or only if
		\e\wrapifsw\e{\e\def\e\first\e{\@pytexTMP@parserReturnValue}%
			% we have an if-else
			\e\e\e\def\e\e\e\@pytexTMP@parserReturnValue\e\e\e{\e\e\e\@pytexOperator@ifelse\e\first\e{\@pytexTMP@parserReturnValue}}%
		}%
		{\@pytexParser@choice{\@pytexParser@tryElseBlock}{\@pytexParser@tryElifStmt}}%
		{%
			% we only have an if
			\e\def\e\@pytexTMP@parserReturnValue\e{\e\@pytexOperator@if\@pytexTMP@parserReturnValue}%
		}%
}

% elif_stmt:
%     | 'elif' named_expression ':' block elif_stmt
%     | 'elif' named_expression ':' block [else_block]
\def\@pytexParser@tryElifStmt{%
	\@pytexParser@joinSecond%
		{\@pytexParser@tryToken{\tokentypeELIF}}%
		{\@pytexParser@tryIfStmtBare}%
}

% else_block:
%     | 'else' ':' block
\def\@pytexParser@tryElseBlock{%
	\@pytexParser@joinSecond%
		{\@pytexParser@tryToken{\tokentypeELSE}}%
		{\@pytexParser@joinSecond%
			{\@pytexParser@tryToken{\tokentypeCOLON}}%
			{\@pytexParser@tryBlock}%
		}%
}






















\def\@pytexParser@skipEmptyLines{%
	\ifmatch{\tokentypeNEWLINE}%
		\def\next{\@pytexParser@skipEmptyLines}%
	\else%
		\def\next{\relax}%
	\fi%
	\next%
}

% statement: compound_stmt  | simple_stmts
\def\@pytexParser@tryStatement{%
	\@pytexParser@choice%
		{\@pytexParser@trySimpleStmts}%
		{\@pytexParser@tryCompoundStmt}%
}

\def\@pytexParser@parseNext{%
	\@pytexParser@skipEmptyLines%
	\ifmatch{\tokentypeEND}%
		\def\next{\relax}%
	\else %
		\wrapif{\@pytexParser@tryStatement}{}%
		{%
			\let\@pytexTMP@parserReturnValue\undefined%
			EXPECTED SOMETHING!
		}%
		%
		%\show\@pytexTMP@parserReturnValue%
		\@pytexTMP@parserReturnValue%
		%
		\def\next{\@pytexParser@parseNext}%
	\fi %
	\next%
}

\def\@pytexParser@parse{
	\@pytexLocal@new{current}%
	\@pytexTokeniser@TokenList@peekim%
	\e\@pytexTokeniser@TokenList@push\e{\edef\tokentype{\tokentypeEND}}%
	
	\@pytexParser@parseNext%
	
	\let\@pytexTMP@parserReturnValue\undefined%
}




\def\STR_Int{Int}%

% #1 = Int object ptr, #2 = object ptr
\def\@pytexRuntime@Int@__add__#1#2{
	\edef\secondType{\@pytexRuntime@Typeof{#2}}%
	\ifx\STR_Int\secondType%
		\edef\result{\the\numexpr\csname#1@value\endcsname+\csname#2@value\endcsname}%
		\e\@pytexCreateInt\e{\result}%
		\show\result%
	\else%
		% TODO: throw NotImplemented error
	\fi%
}




\def\Swap#1#2{%
	#2%
	#1%
}

\let\@pytexRuntime@return\undefined

\newcount\objnum
\objnum=0

\def\@pytexRuntime@alloc{% -> empty object ptr TODO: refcounting
	\advance\objnum by 1\relax%
	\edef\@pytexRuntime@return{@pytexObject@\the\objnum}%
}

\def\@pytexRuntime@allocType#1{% -> empty object ptr with type #1
	\@pytexRuntime@alloc%
	\e\e\e\def\e\csname\@pytexRuntime@return @type\endcsname{#1}%
}

\def\@pytexRuntime@Typeof#1{%
	\csname#1@type\endcsname%
}

\def\@pytexCreateInt#1{% -> int object ptr
	\@pytexRuntime@allocType{Int}%
	\e\e\e\def\e\csname\@pytexRuntime@return @value\endcsname{#1}%
}

% #1 = name, \@pytexRuntime@return = value
\def\@pytexRuntime@assign#1{%
	\e\let\csname @pytexRuntime@currentScope@#1\endcsname\@pytexRuntime@return%
}

% #1 = name
\def\@pytexRuntime@GetIdentifier#1{% -> object ptr
	\e\let\e\@pytexRuntime@return\csname @pytexRuntime@currentScope@#1\endcsname%
}


% #1 = name (for now), #2 = expression
\def\@pytexOperator@assignment#1#2{%
	#2%
	\@pytexRuntime@assign{#1}%
}

% #1, #2 = expression
\def\@pytexOperator@add#1#2{%
	#1%
	\e\Swap\e{\e\def\e\first\e{\@pytexRuntime@return}}{#2}%
	\let\second\@pytexRuntime@return%
	%
	\edef\firsttype{\e\@pytexRuntime@Typeof\e{\first}}%
	%
	\edef\func{@pytexRuntime@\firsttype @__add__}%
	\begingroup\e\endgroup\e\def\e\func\e{\csname\e\func\endcsname}%
	\e\e\e\func\e\e\e{\e\first\e}\e{\second}%
}




\def\@pytexChar@Backslash{\loggingall\@pytexParser@parse}
\let\@pytexChar@Dollar\bye

\def\@pytexChar@DoubleQuote{\relax}
\def\@pytexChar@SingleQuote{\relax}


%BAD MACRO: tokentypeMODULO

\ifx\@pytexOperator@assignment\undefined\else DEFINED MACRO: @pytexOperator@assignment
\fi

BAD MACRO: resultsavestack@push

\ifx\@pytexParser@choice\undefined\else DEFINED MACRO: @pytexParser@choice
\fi

BAD MACRO: tokentype

\ifx\@pytexRuntime@GetIdentifier\undefined\else DEFINED MACRO: @pytexRuntime@GetIdentifier
\fi

BAD MACRO: tokentypeNOTEQUAL

BAD MACRO: tokentypeSUBTRACT

BAD MACRO: restorestate

BAD MACRO: tokentypeNEWLINE

BAD MACRO: swap

\ifx\@pytexParser@choiceFour\undefined\else DEFINED MACRO: @pytexParser@choiceFour
\fi

\ifx\@pytexTMP@prevStarCatcode\undefined\else TMP MACRO: @pytexTMP@prevStarCatcode was not undefined
\fi

BAD MACRO: currentindent

BAD MACRO: indentStack@push

\ifx\@pytexTMP@keyword\undefined\else TMP MACRO: @pytexTMP@keyword was not undefined
\fi

BAD MACRO: tokentypeEND

BAD MACRO: i

\ifx\@pytexTMP@prevTildaCatcode\undefined\else TMP MACRO: @pytexTMP@prevTildaCatcode was not undefined
\fi

\ifx\@pytexError\undefined\else DEFINED MACRO: @pytexError
\fi

BAD MACRO: tokentypeEQUAL

BAD MACRO: iffalse

\ifx\@pytexTMP@resetcatcode\undefined\else TMP MACRO: @pytexTMP@resetcatcode was not undefined
\fi

\ifx\@pytexTMP@macro\undefined\else TMP MACRO: @pytexTMP@macro was not undefined
\fi

\ifx\@pytexParser@tryCompoundStmt\undefined\else DEFINED MACRO: @pytexParser@tryCompoundStmt
\fi

BAD MACRO: tokentypeASSIGN

BAD MACRO: try

BAD MACRO: tokentypeMULTIPLY

\ifx\@pytexTMP@parserReturnValue\undefined\else TMP MACRO: @pytexTMP@parserReturnValue was not undefined
\fi

BAD MACRO: x

BAD MACRO: indentStack@peek

BAD MACRO: t

\ifx\@pytexParser@choiceFive\undefined\else DEFINED MACRO: @pytexParser@choiceFive
\fi

\ifx\@pytexParser@tryStatement\undefined\else DEFINED MACRO: @pytexParser@tryStatement
\fi

BAD MACRO: ifpeek

BAD MACRO: tokentypeCOLON

BAD MACRO: wrapif

\ifx\@pytexParser@tryIdentifier\undefined\else DEFINED MACRO: @pytexParser@tryIdentifier
\fi

\ifx\@pytexParser@joinFirstUnary\undefined\else DEFINED MACRO: @pytexParser@joinFirstUnary
\fi

BAD MACRO: wrapifsw

BAD MACRO: @pytexToken@Dedent

BAD MACRO: Swap

\ifx\@pytexTMP@stackElement\undefined\else TMP MACRO: @pytexTMP@stackElement was not undefined
\fi

\ifx\@pytexTMP@symbol\undefined\else TMP MACRO: @pytexTMP@symbol was not undefined
\fi

\ifx\@pytexLocal@begin\undefined\else DEFINED MACRO: @pytexLocal@begin
\fi

\ifx\@pytexParser@joinSecond\undefined\else DEFINED MACRO: @pytexParser@joinSecond
\fi

BAD MACRO: ExpandAfter

\ifx\@pytexParser@NotImplemented\undefined\else DEFINED MACRO: @pytexParser@NotImplemented
\fi

BAD MACRO: tokentypeDEDENT

BAD MACRO: first

BAD MACRO: tokentypeBITWISEOR

BAD MACRO: tokentypeIF

BAD MACRO: ifmatch

BAD MACRO: tokentypeNOT

BAD MACRO: tokentypeELIF

\ifx\@pytexParser@tryTokenValued\undefined\else DEFINED MACRO: @pytexParser@tryTokenValued
\fi

BAD MACRO: tokentypeBITWISEXOR

BAD MACRO: firsttype

BAD MACRO: tokentypeFLOORDIVIDE

BAD MACRO: tokentypeINDENT

\ifx\@pytexOperator@add\undefined\else DEFINED MACRO: @pytexOperator@add
\fi

BAD MACRO: repeat

BAD MACRO: tokenvalue

BAD MACRO: message

BAD MACRO: tokentypeIDENTIFIER

BAD MACRO: currentidx

BAD MACRO: tokentypeELSE

\ifx\@pytexParser@joinOptional\undefined\else DEFINED MACRO: @pytexParser@joinOptional
\fi

\ifx\@pytexParser@join\undefined\else DEFINED MACRO: @pytexParser@join
\fi

\ifx\@pytexRuntime@allocType\undefined\else DEFINED MACRO: @pytexRuntime@allocType
\fi

\ifx\@pytexRuntime@alloc\undefined\else DEFINED MACRO: @pytexRuntime@alloc
\fi

\ifx\@pytexTMP@prevCaretCatcode\undefined\else TMP MACRO: @pytexTMP@prevCaretCatcode was not undefined
\fi

\ifx\@pytexRuntime@assign\undefined\else DEFINED MACRO: @pytexRuntime@assign
\fi

\ifx\@pytexRuntime@Typeof\undefined\else DEFINED MACRO: @pytexRuntime@Typeof
\fi

BAD MACRO: current

\ifx\@pytexParser@parse\undefined\else DEFINED MACRO: @pytexParser@parse
\fi

\ifx\@pytexTMP@prevcatcode\undefined\else TMP MACRO: @pytexTMP@prevcatcode was not undefined
\fi

BAD MACRO: tokentypeDIVIDE

\ifx\@pytexParser@trySimpleStmts\undefined\else DEFINED MACRO: @pytexParser@trySimpleStmts
\fi

BAD MACRO: next

\ifx\@pytexTMP@localvar\undefined\else TMP MACRO: @pytexTMP@localvar was not undefined
\fi

BAD MACRO: lastindent

\ifx\@pytexParser@choiceThree\undefined\else DEFINED MACRO: @pytexParser@choiceThree
\fi

\ifx\@pytexLocal@new\undefined\else DEFINED MACRO: @pytexLocal@new
\fi

\ifx\@pytexParser@tryToken\undefined\else DEFINED MACRO: @pytexParser@tryToken
\fi

BAD MACRO: state

\ifx\@pytexParser@joinFirst\undefined\else DEFINED MACRO: @pytexParser@joinFirst
\fi

BAD MACRO: tokentypeIS

BAD MACRO: advance

\ifx\@pytexParser@joinOptionalDouble\undefined\else DEFINED MACRO: @pytexParser@joinOptionalDouble
\fi

BAD MACRO: tokentypeNUMBER

\ifx\@pytexCreateInt\undefined\else DEFINED MACRO: @pytexCreateInt
\fi

BAD MACRO: loop

\ifx\@pytexParser@parseNext\undefined\else DEFINED MACRO: @pytexParser@parseNext
\fi

BAD MACRO: iftrue

BAD MACRO: commitstate

BAD MACRO: tokentypeBITWISEAND

BAD MACRO: idxsavestack@push

\ifx\@pytexRuntime@return\undefined\else DEFINED MACRO: @pytexRuntime@return
\fi

BAD MACRO: tokentypeIN

BAD MACRO: tokentypeOR

BAD MACRO: tokentypeADD

BAD MACRO: enumnext

BAD MACRO: tokentypeAND

BAD MACRO: objnum

BAD MACRO: savestate

BAD MACRO: indentStack@popd

BAD MACRO: ifresult

BAD MACRO: e

BAD MACRO: toketype

BAD MACRO: typeout

BAD MACRO: tokentypeEXPONENTIATION

\ifx\@pytexLocal@end\undefined\else DEFINED MACRO: @pytexLocal@end
\fi

\ifx\@pytexParser@choiceSix\undefined\else DEFINED MACRO: @pytexParser@choiceSix
\fi

\ifx\@pytexParser@skipEmptyLines\undefined\else DEFINED MACRO: @pytexParser@skipEmptyLines
\fi

BAD MACRO: @pytexToken@Indent

\ifx\@pytexParser@joinSecondUnary\undefined\else DEFINED MACRO: @pytexParser@joinSecondUnary
\fi

BAD MACRO: second

\ifx\@pytexTMP@queueElement\undefined\else TMP MACRO: @pytexTMP@queueElement was not undefined
\fi

BAD MACRO: func



PROGRAM RESULT:

\@pytexMakeAllActive

# todo: function definitions
# todo: class definitions
# todo: in-place assignments (+=, -=, etc.)
# todo: f-strings and other format strings


# Arithmetic operators
a = 10
b = 5
c = a + b  # Addition
d = a - b  # Subtraction
e = a * b  # Multiplication
# f = a / b  # Division TODO
g = a // b # Floor Division
# h = a % b  # Modulo TODO
i = -a     # Negation
i2 = +a    # Positive
# j = a ** b # Exponentiation TODO


# Parentheses for order of operations
k = (a + b) * c

# # Comparison operators TODO
# print(a == b)  # Equal to
# print(a != b)  # Not equal to
# print(a > b)   # Greater than
# print(a < b)   # Less than
# print(a >= b)  # Greater than or equal to
# print(a <= b)  # Less than or equal to


# # Logical operators
x = True and False  # AND
y = False or 5   # OR
z = not True        # NOT


# Conditional statements
if x:
    pass
elif y:
    pass
else:
    pass


if z:
    pass
else:
    pass

y = 15
if y-15:
    pass
elif not y:
	pass
elif True and y:
    pass
else:
    pass


# # Bitwise operators TODO
# m = 10  # 1010 in binary
# n = 4   # 0100 in binary
# print(m & n)  # Bitwise AND
# print(m | n)  # Bitwise OR
# print(m ^ n)  # Bitwise XOR
# print(~m)   # Bitwise NOT
# print(m << 2) # Left shift
# print(m >> 1) # Right shift
# 
# 
# # Assignment operators TODO
# x = 10
# x += 5
# x -= 2

# Boolean values
isTrue = True
isFalse = False

# # Multiple assignments TODO
# p, q, r = 1, 2, 3
# print(p,q,r)

# TODO
# '''This is a
# multiline comment'''
# 
# 
# String operations TODO
# str1 = "Hello"
# str2 = " World"
# str3 = str1 + str2
# print(str3)
# len(str3)
# 

\$
\bye
