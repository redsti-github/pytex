% NOTE: this is not the entire tokeniser. Parts are generated by 'gen_token.py' and added to this file

\newcount\state
\state=5

\def\@pytexTokeniser@buffer{}
\@pytexList@new{@pytexTokenList}

\newcount\currentindent
\currentindent=0
\@pytexStack@new{indentStack}
\indentStack@push{0}


\def\STR_python{python}
\def\STR_CloseBrace{CloseBrace}
\def\STR_end{end}

\def\@pytexTokeniser@checkEnd{%
	\@pytexTokenList@getendim{0}%
	\ifnum\tokentype=\tokentypeIDENTIFIER% python
		\ifx\tokenvalue\STR_python%
		
			\@pytexTokenList@getendim{1}%
			\ifnum\tokentype=\tokentypeLBRACE% {
			
				\@pytexTokenList@getendim{2}%
				\ifnum\tokentype=\tokentypeIDENTIFIER% end
					\ifx\tokenvalue\STR_end%
					
						\@pytexTokenList@getendim{3}%
						\ifnum\tokentype=\tokentypeBACKSLASH% \
						
							\ifx\@pytexTokeniser@buffer\STR_CloseBrace% }
								\@pytexTokeniser@endsymboltoken%
								\@pytexTokenList@popd% }
								\@pytexTokenList@popd% python
								\@pytexTokenList@popd% {
								\@pytexTokenList@popd% end
								\@pytexTokenList@popd% \

								\end{python}%
							\fi%
						\fi%
					\fi%
				\fi%
			\fi%
		\fi%
	\fi%
}


\def\@pytexTokeniser@endnumbertoken{
	\ifnum\state=2
		\e\@pytexTokenList@push\e{\e\@pytexToken@Number\e{\@pytexTokeniser@buffer}}
		\gdef\@pytexTokeniser@buffer{}
		\state=0
	\fi
}

\def\@pytexTokeniser@dedent{
	\ifnum\currentindent<\lastindent
		\@pytexTokenList@push{\@pytexToken@Dedent}
		\indentStack@popd
		\indentStack@peek{\lastindent}
		\ifnum\currentindent>\lastindent
			\@pytexError{IndentationError: unindent does not match any outer indentation level}
		\fi
		\@pytexTokeniser@dedent
	\fi
}

\def\@pytexTokeniser@endindent{
	\ifnum\state=5
		\indentStack@peek{\lastindent}
		\ifnum\currentindent>\lastindent
			\e\indentStack@push\e{\the\currentindent}
			\edef\lastindent{\the\currentindent}
			\@pytexTokenList@push{\@pytexToken@Indent}
		\fi
		\@pytexTokeniser@dedent
		\currentindent=0
		\state=0
	\fi
}



\def\@pytexTokeniser@letter#1{
	\@pytexTokeniser@endindent
	\@pytexTokeniser@endsymboltoken
	\ifnum\state=4\else	\ifnum\state=0\else\ifnum\state=1\else \@pytexError{Internal error: starting to tokenise name when not in 0 state.}\fi\fi
		\xdef\@pytexTokeniser@buffer{\@pytexTokeniser@buffer #1}
		\state=1
	\fi
}

\def\@pytexTokeniser@number#1{
	\@pytexTokeniser@endindent
	\@pytexTokeniser@endsymboltoken
	\ifnum\state=4\else	\ifnum\state=1
		\@pytexTokeniser@letter{#1}
	\else
		\ifnum\state=0\else\ifnum\state=2\else \@pytexError{Internal error: starting to tokenise number when not in 0 state.}\fi\fi
		\xdef\@pytexTokeniser@buffer{\@pytexTokeniser@buffer #1}
		\state=2
	\fi
\fi}

\def\@pytexTokeniser@symbol#1{
	\@pytexTokeniser@endnametoken
	\@pytexTokeniser@endnumbertoken
	\@pytexTokeniser@endindent
	\ifnum\state=4\else	\ifnum\state=0\else\ifnum\state=3\else \@pytexError{Internal error: starting to symbols name when not in 0 state.}\fi\fi
	\xdef\@pytexTokeniser@buffer{\@pytexTokeniser@buffer #1}
	\state=3
	\fi
}




\def\@pytexChar@Newline{
    \@pytexTokeniser@endnametoken
    \@pytexTokeniser@endnumbertoken
    \@pytexTokeniser@endindent
    \@pytexTokeniser@endsymboltoken
    \ifnum\state=0\else\ifnum\state=4\else \@pytexError{Internal error: starting to tokenise comment when not in 0 state.}\fi\fi
    \state=5
    \@pytexTokenList@push{\@pytexToken@Newline}
}

\def\@pytexChar@Space{
    \@pytexTokeniser@endnametoken
    \@pytexTokeniser@endnumbertoken
    \@pytexTokeniser@endsymboltoken
    \ifnum\state=5
        \advance\currentindent by 1
    \fi
}

\def\@pytexChar@Tab{
    \@pytexTokeniser@endnametoken
    \@pytexTokeniser@endnumbertoken
    \@pytexTokeniser@endsymboltoken
    \ifnum\state=5
        \advance\currentindent by 1
	\fi
}

\def\@pytexChar@Hash{
\ifnum\state=4\else
    \@pytexTokeniser@endnametoken
    \@pytexTokeniser@endnumbertoken
    \@pytexTokeniser@endsymboltoken
    \@pytexTokeniser@endindent
    \ifnum\state=0\else \@pytexError{Internal error: starting comment when not in 0 state.}\fi
    \state=4
\fi
}
