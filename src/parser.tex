% https://www.programiz.com/python-programming/precedence-associativity
% https://www.wscubetech.com/resources/python/precedence-associativity-operators
% https://docs.python.org/3/reference/expressions.html#grammar-token-python-grammar-primary

% try<thing> looks ahead and tries to parse <thing>.
% if parsing was succesful:
	% sets \@pytexParserReturn to the parsed construct
	% consumes the parsed tokens
	% ends with \iftrue
% else,
	% leaves \@pytexParserReturn untouched
	% leaves tokens untouched
	% ends with \iffalse
% use \wrapif{\try<thing>}{<iftrue>}{<iffalse>} to avoid problems with ifs

\def\@pytexParser@NotImplemented{\iffalse}%


% token type enum
\def\tokentypeEND{0}
\def\enumnext#1{%
	\e\edef\csname tokentype#1\endcsname{\tokentypeEND}%
	\edef\tokentypeEND{\the\numexpr\tokentypeEND+1}%
}

\enumnext{NEWLINE}

\enumnext{IDENTIFIER}
\enumnext{NUMBER}

\enumnext{AND}
\enumnext{OR}
\enumnext{NOT}
\enumnext{IS}
\enumnext{ISNOT}
\enumnext{IN}
\enumnext{NOTIN}

\enumnext{EXPONENTIATION}
\enumnext{INVERT}
\enumnext{MULTIPLY}
\enumnext{MATMULT}
\enumnext{FLOORDIVIDE}
\enumnext{DIVIDE}
\enumnext{MODULO}

\enumnext{ADD}
\enumnext{SUBTRACT}
\enumnext{SHIFTLEFT}
\enumnext{SHIFTRIGHT}

\enumnext{BITWISEAND}
\enumnext{BITWISEOR}
\enumnext{BITWISEXOR}

\enumnext{EQUAL}
\enumnext{NOTEQUAL}
\enumnext{LESSTHAN}
\enumnext{LESSEQUAL}
\enumnext{GREATER}
\enumnext{GREATEREQUAL}

\enumnext{ASSIGN}

% keywords
\enumnext{AWAIT}
\enumnext{IF}
\enumnext{ELSE}
\enumnext{ELIF}
\enumnext{FOR}
\enumnext{PASS}
\enumnext{BREAK}
\enumnext{CONTINUE}
\enumnext{RETURN}
\enumnext{TRUE}
\enumnext{FALSE}

% other
\enumnext{COMMA}
\enumnext{SEMICOLON}
\enumnext{COLON}
\enumnext{INDENT}
\enumnext{DEDENT}
\enumnext{RPAREN}
\enumnext{LPAREN}
\enumnext{RBRACE}
\enumnext{LBRACE}
\enumnext{BACKSLASH}

\enumnext{PRINT}



% token macros - sets \toketype and, optionally, \tokenvalue
\def\@pytexToken@Newline{\edef\tokentype{\tokentypeNEWLINE}}
\def\@pytexToken@Identifier#1{\edef\tokentype{\tokentypeIDENTIFIER}\def\tokenvalue{#1}}
\def\@pytexToken@Number#1{\edef\tokentype{\tokentypeNUMBER}\def\tokenvalue{#1}}

\def\@pytexToken@and{\edef\tokentype{\tokentypeAND}}
\def\@pytexToken@or{\edef\tokentype{\tokentypeOR}}
\def\@pytexToken@in{\edef\tokentype{\tokentypeIN}}
\def\@pytexToken@is{\edef\tokentype{\tokentypeIS}}
\def\@pytexToken@not{\edef\tokentype{\tokentypeNOT}}

\def\@pytexToken@Addition{\edef\tokentype{\tokentypeADD}}
\def\@pytexToken@Subtraction{\edef\tokentype{\tokentypeSUBTRACT}}
\def\@pytexToken@Multiplication{\edef\tokentype{\tokentypeMULTIPLY}}
\def\@pytexToken@Division{\edef\tokentype{\tokentypeDIVIDE}}
\def\@pytexToken@FloorDivision{\edef\tokentype{\tokentypeFLOORDIVIDE}}
\def\@pytexToken@EqualTo{\edef\tokentype{\tokentypeEQUAL}}
\def\@pytexToken@NotEqualTo{\edef\tokentype{\tokentypeNOTEQUAL}}
\def\@pytexToken@Exponentiation{\edef\tokentype{\tokentypeEXPONENTIATION}}
\def\@pytexToken@Modulo{\edef\tokentype{\tokentypeMODULO}}

\def\@pytexToken@Assignment{\edef\tokentype{\tokentypeASSIGN}}
\def\@pytexToken@Indent{\edef\tokentype{\tokentypeINDENT}}
\def\@pytexToken@Dedent{\edef\tokentype{\tokentypeDEDENT}}
\def\@pytexToken@BitwiseOr{\edef\tokentype{\tokentypeBITWISEOR}}
\def\@pytexToken@BitwiseXor{\edef\tokentype{\tokentypeBITWISEXOR}}
\def\@pytexToken@BitwiseAnd{\edef\tokentype{\tokentypeBITWISEAND}}

\def\@pytexToken@if{\edef\tokentype{\tokentypeIF}}
\def\@pytexToken@else{\edef\tokentype{\tokentypeELSE}}
\def\@pytexToken@elif{\edef\tokentype{\tokentypeELIF}}
\def\@pytexToken@Colon{\edef\tokentype{\tokentypeCOLON}}
\def\@pytexToken@True{\edef\tokentype{\tokentypeTRUE}}
\def\@pytexToken@False{\edef\tokentype{\tokentypeFALSE}}
\def\@pytexToken@pass{\edef\tokentype{\tokentypePASS}}

\def\@pytexToken@OpeningParenthesis{\edef\tokentype{\tokentypeLPAREN}}
\def\@pytexToken@ClosingParenthesis{\edef\tokentype{\tokentypeRPAREN}}
\def\@pytexToken@OpeningBrace{\edef\tokentype{\tokentypeLBRACE}}
\def\@pytexToken@ClosingBrace{\edef\tokentype{\tokentypeRBRACE}}
\def\@pytexToken@Backslash{\edef\tokentype{\tokentypeBACKSLASH}}

\def\@pytexToken@print{\edef\tokentype{\tokentypePRINT}}




\def\ifmatch#1{%
	\e\@pytexTokenList@getim\e{\currentidx}%
	\ifnum\tokentype=#1%
		\edef\currentidx{\the\numexpr\currentidx+1}%
}
\def\ifpeek#1{%
	\e\@pytexTokenList@getim\e{\currentidx}%
	\ifnum\tokentype=#1%
}

\def\savestate{%
	\e\idxsavestack@push\e{\currentidx}%
	\e\resultsavestack@push\e{\@pytexParserReturn}%
}
\def\restorestate{%
	\@pytexStack@pop{idxsavestack}{\currentidx}%
	\@pytexStack@pop{resultsavestack}{\@pytexParserReturn}%
}
\def\commitstate{%
	\@pytexStack@popd{idxsavestack}%
	\@pytexStack@popd{resultsavestack}%
}



% #1 = if, #2 = iftrue, #3 = iffalse
\def\wrapif#1#2#3{%
	#1%
		#2%
	\else%
		#3%
	\fi%
}

% #1 = iftrue, #2 = if, #3 = iffalse
\def\wrapifsw#1#2#3{%
	#2%
		#1%
	\else%
		#3%
	\fi%
}

% result ::= #1, ParserReturn is left untouched
\def\@pytexParser@tryToken#1{%
	\ifmatch{#1}%
}

% result ::= #1
\def\@pytexParser@tryTokenValued#1{%
	\ifmatch{#1}%
		\e\def\e\@pytexParserReturn\e{\tokenvalue}%
}

% result ::= #1 | #2
\def\@pytexParser@choice#1#2{%
	#1%
		\def\ifresult{1}%
	\else%
		\wrapif{#2}{%
			\def\ifresult{1}%
		}{%
			\def\ifresult{0}%
		}%
	\fi%
	\ifnum\ifresult=1%
}

% result ::= #1 | #2 | #3
\def\@pytexParser@choiceThree#1#2#3{%
	\@pytexParser@choice{#1}{\@pytexParser@choice{#2}{#3}}%
}

% result ::= #1 | #2 | #3 | #4
\def\@pytexParser@choiceFour#1#2#3#4{%
	\@pytexParser@choice{#1}{\@pytexParser@choiceThree{#2}{#3}{#4}}%
}

% result ::= #1 | #2 | #3 | #4 | #5
\def\@pytexParser@choiceFive#1#2#3#4#5{%
	\@pytexParser@choice{#1}{\@pytexParser@choiceFour{#2}{#3}{#4}{#5}}%
}

% result ::= #1 | #2 | #3 | #4 | #5 | #6
\def\@pytexParser@choiceSix#1#2#3#4#5#6{%
	\@pytexParser@choice{#1}{\@pytexParser@choiceFive{#2}{#3}{#4}{#5}{#6}}%
}


\def\@pytexParser@tryIdentifier{%
	\@pytexParser@tryToken{\tokentypeIDENTIFIER}%
		\e\def\e\@pytexParserReturn\e{\tokenvalue}%
}



% #1 = first thing, #2 = second thing, #3 = joining operation (results of #1 and #2 are in \first and \second)
% result ::= #1 #2
\def\@pytexParser@join#1#2#3{%
	\savestate%
	\wrapif{#1}{%
		\e\wrapifsw\e{%
			\e\def\e\first\e{\@pytexParserReturn}%
			\let\second\@pytexParserReturn%
			#3%
			\def\ifresult{1}%
			\commitstate%
		}{#2}{%
			\def\ifresult{0}%
			\restorestate%
		}%
	}{%
		\def\ifresult{0}%
		\restorestate%
	}%
	\ifnum\ifresult=1%
}

% similar to \@pytexParser@join, but \@pytexParserReturn is automatically set to the return value of #1
\def\@pytexParser@joinFirst#1#2{%
	\savestate%
	\wrapif{#1}{%
		\e\wrapifsw\e{\e\def\e\pytexParserReturn\e{\@pytexParserReturn}%
			\commitstate%
			\def\ifresult{1}%
		}{#2}{%
			\restorestate%
			\def\ifresult{0}%
		}%
	}{%
		\restorestate%
		\def\ifresult{0}%
	}%
	\ifnum\ifresult=1%
}

% similar to \@pytexParser@join, but \@pytexParserReturn is automatically set to the return value of #2
\def\@pytexParser@joinSecond#1#2{%
	\savestate%
	\wrapif{#1}{% TODO: this could mangle ParserReturn
		\wrapif{#2}{%
			\commitstate%
			\def\ifresult{1}%
		}{%
			\restorestate%
			\def\ifresult{0}%
		}%
	}{%
		\restorestate%
		\def\ifresult{0}%
	}%
	\ifnum\ifresult=1%
}

% similar to \@pytexParser@join, but \@pytexParserReturn is set to #3 (given \@pytexParserReturn as result of #1)
\def\@pytexParser@joinFirstUnary#1#2#3{%
	\savestate%
	\wrapif{#1}{%
		\e\wrapifsw\e{%
			\e\def\e\@pytexParserReturn\e{\@pytexParserReturn}%
			#3%
			\commitstate%
			\def\ifresult{1}%
		}{#2}{%
			\restorestate%
			\def\ifresult{0}%
		}%
	}{%
		\restorestate%
		\def\ifresult{0}%
	}%
	\ifnum\ifresult=1%
}

% similar to \@pytexParser@join, but \@pytexParserReturn is set to #3 (given \@pytexParserReturn as result of #2)
\def\@pytexParser@joinSecondUnary#1#2#3{%
	\savestate%
	\wrapif{#1}{%
		\wrapif{#2}{%
			#3%
			\commitstate%
			\def\ifresult{1}%
		}{%
			\restorestate%
			\def\ifresult{0}%
		}%
	}{%
		\restorestate%
		\def\ifresult{0}%
	}%
	\ifnum\ifresult=1%
}

% #1 = first thing, #2 = second thing, #3 = joining operation (provided with \first and \second), only executed when #2 is present
% result ::= #1 [#2]	-> {if(#2): #3}
\def\@pytexParser@joinOptional#1#2#3{%
	\savestate%
	\wrapif{#1}{%
		\e\wrapifsw\e{%
			\e\def\e\first\e{\@pytexParserReturn}%
			\let\second\@pytexParserReturn%
			#3%
			\def\ifresult{1}%
			\commitstate%
		}{#2}{%
			\def\ifresult{1}%
			\commitstate%
		}%
	}{%
		\def\ifresult{0}%
		\restorestate%
	}%
	\ifnum\ifresult=1%
}


% #1 = first thing, #2 = second thing, #3 = single operation, #4 = double joining operation
% result ::= #1 [#2]	-> {if(#2): #4 else: #3}
\def\@pytexParser@joinOptionalDouble#1#2#3#4{%
	\savestate%
	\wrapif{#1}{%
		\@pytexLocal@begin{current}%
		\let\current\@pytexParserReturn%
		\wrapif{#2}{%
			\let\first\current%
			\let\second\@pytexParserReturn%
			#4%
			\def\ifresult{1}%
			\commitstate%
		}{%
			\let\first\current%
			#3%
			\def\ifresult{1}%
			\commitstate%
		}%
		\@pytexLocal@end{current}%
	}{%
		\def\ifresult{0}%
		\restorestate%
	}%
	\ifnum\ifresult=1%
}





% await_primary:
%     | 'await' primary 
%     | primary
\def\@pytexParser@tryAwaitPrimary{%
    \@pytexParser@choice%
    {%
        \@pytexParser@join%
            {\@pytexParser@tryToken{\tokentypeAWAIT}}%
            {\@pytexParser@tryPrimary}%
            {\e\def\e\@pytexParserReturn\e{\e\@pytexOperator@await\e{\second}}}%
    }%
    {\@pytexParser@tryPrimary}%
}

% primary:
% TODO | primary '.' NAME 
% TODO | primary genexp 
% TODO | primary '(' [arguments] ')' 
% TODO | primary '[' slices ']' 
%      | atom
\def\@pytexParser@tryPrimary{%
	\@pytexParser@tryAtom%
}


% slices:
%     | slice !',' 
%     | ','.(slice | starred_expression)+ [','] 
% TODO


% slice:
%     | [expression] ':' [expression] [':' [expression] ] 
%     | named_expression 
% TODO


\def\@pytexParser@tryInt{%
	\@pytexParser@tryTokenValued{\tokentypeNUMBER}%
		\e\def\e\@pytexParserReturn\e{\e\@pytexRuntime@int\e{\tokenvalue}}%
}

\def\@pytexParser@tryIdentifierExpr{%
	\@pytexParser@tryIdentifier%
		\e\def\e\@pytexParserReturn\e{\e\@pytexRuntime@GetIdentifier\e{\tokenvalue}}%
}

\def\@pytexParser@tryBool{%
	\@pytexParser@choice%
	{\@pytexParser@tryToken{\tokentypeTRUE}%
		\def\@pytexParserReturn{\@pytexRuntime@boolTrue}}%
	{\@pytexParser@tryToken{\tokentypeFALSE}%
		\def\@pytexParserReturn{\@pytexRuntime@boolFalse}}%
}

% atom:
%     | NAME
%     | 'True'
%     | 'False'
%TODO | 'None'
%TODO | strings
%     | NUMBER
%TODO | (tuple | group | genexp)
%TODO | (list | listcomp)
%TODO | (dict | set | dictcomp | setcomp)
%TODO | '...'
\def\@pytexParser@tryAtom{%
	\@pytexParser@choiceFour%
		{\@pytexParser@tryIdentifierExpr}%
		{\@pytexParser@tryInt}%
		{\@pytexParser@tryBool}%
		{\@pytexParser@tryGroup}%
}


% group:
%     | '(' (yield_expr | named_expression) ')' 
\def\@pytexParser@tryGroup{%
	\@pytexParser@joinSecond%
		{\@pytexParser@tryToken{\tokentypeLPAREN}}%
		{\@pytexParser@joinFirst%
			{\@pytexParser@choice{\@pytexParser@tryYieldExpr}{\@pytexParser@tryNamedExpression}}%
			{\@pytexParser@tryToken{\tokentypeRPAREN}}%
		}%
}








% sum:
%     | sum '+' term 
%     | sum '-' term 
%     | term
\def\@pytexParser@trySum{%
	\@pytexParser@tryTerm%
		\@pytexLocal@begin{current}%
		\let\current\@pytexParserReturn%
		\@pytexParser@trySumNext%
		\let\@pytexParserReturn\current%
		\@pytexLocal@end{current}%
}
\def\@pytexParser@trySumNext{%
	\@pytexParser@choice%
		{%
		\@pytexParser@join{\@pytexParser@tryToken{\tokentypeADD}}{\@pytexParser@tryTerm}%
			{\e\e\e\def\e\e\e\current\e\e\e{\e\e\e\@pytexOperator@add\e\e\e{\e\current\e}\e{\@pytexParserReturn}}}%
		}{%
		\@pytexParser@join{\@pytexParser@tryToken{\tokentypeSUBTRACT}}{\@pytexParser@tryTerm}%
			{\e\e\e\def\e\e\e\current\e\e\e{\e\e\e\@pytexOperator@subtract\e\e\e{\e\current\e}\e{\@pytexParserReturn}}}%
		}%
		\def\next{\@pytexParser@trySumNext}%
	\else%
		\def\next{\relax}%
	\fi%
	\next%
}




% term:
%     | term '*' factor 
%     | term '/' factor 
%     | term '//' factor 
%     | term '%' factor 
%     | term '@' factor 
%     | factor
\def\@pytexParser@tryTerm{%
	\@pytexParser@tryFactor%
		\@pytexLocal@begin{current}%
		\let\current\@pytexParserReturn%
		\@pytexParser@tryTermNext%
		\let\@pytexParserReturn\current%
		\@pytexLocal@end{current}%
}
\def\@pytexParser@tryTermNext{%
	\@pytexParser@choiceFive%
		{%
		\@pytexParser@join{\@pytexParser@tryToken{\tokentypeMULTIPLY}}{\@pytexParser@tryFactor}%
			{\e\e\e\def\e\e\e\current\e\e\e{\e\e\e\@pytexOperator@multiply\e\e\e{\e\current\e}\e{\@pytexParserReturn}}}%
		}{%
		\@pytexParser@join{\@pytexParser@tryToken{\tokentypeMATMULT}}{\@pytexParser@tryFactor}%
			{\e\e\e\def\e\e\e\current\e\e\e{\e\e\e\@pytexOperator@matmult\e\e\e{\e\current\e}\e{\@pytexParserReturn}}}%
		}{%
		\@pytexParser@join{\@pytexParser@tryToken{\tokentypeFLOORDIVIDE}}{\@pytexParser@tryFactor}%
			{\e\e\e\def\e\e\e\current\e\e\e{\e\e\e\@pytexOperator@floordivide\e\e\e{\e\current\e}\e{\@pytexParserReturn}}}%
		}{%
		\@pytexParser@join{\@pytexParser@tryToken{\tokentypeDIVIDE}}{\@pytexParser@tryFactor}%
			{\e\e\e\def\e\e\e\current\e\e\e{\e\e\e\@pytexOperator@divide\e\e\e{\e\current\e}\e{\@pytexParserReturn}}}%
		}{%
		\@pytexParser@join{\@pytexParser@tryToken{\tokentypeMODULO}}{\@pytexParser@tryFactor}%
			{\e\e\e\def\e\e\e\current\e\e\e{\e\e\e\@pytexOperator@modulo\e\e\e{\e\current\e}\e{\@pytexParserReturn}}}%
		}%
		\def\next{\@pytexParser@tryTermNext}%
	\else%
		\def\next{\relax}%
	\fi%
	\next%
}




% factor:
%     | '+' factor
%     | '-' factor
%     | '~' factor
%     | power
\def\@pytexParser@tryFactor{%
	\@pytexParser@choiceFour%
		{\@pytexParser@tryPower}% power
		{%
		\@pytexParser@join{\@pytexParser@tryToken{\tokentypeSUBTRACT}}{\@pytexParser@tryFactor}% "-" u_expr
			{\e\def\e\@pytexParserReturn\e{\e\@pytexOperator@negate\e{\second}}}%
		}{%
		\@pytexParser@join{\@pytexParser@tryToken{\tokentypeADD}}{\@pytexParser@tryFactor}% "+" u_expr
			{\e\def\e\@pytexParserReturn\e{\e\@pytexOperator@pos\e{\second}}}%
		}{%
		\@pytexParser@join{\@pytexParser@tryToken{\tokentypeINVERT}}{\@pytexParser@tryFactor}% "~" u_expr
			{\e\def\e\@pytexParserReturn\e{\e\@pytexOperator@invert\e{\second}}}%
		}%
}



% power:
%     | await_primary '**' factor 
%     | await_primary
\def\@pytexParser@tryPower{%
	\@pytexParser@joinOptional%
		{\@pytexParser@tryAwaitPrimary}
		{\@pytexParser@join{\@pytexParser@tryToken{\tokentypeEXPONENTIATION}}{\@pytexParser@tryFactor}{\let\@pytexParserReturn\second}}%
		{\e\e\e\def\e\e\e\@pytexParserReturn\e\e\e{\e\e\e\@pytexOperator@pow\e\e\e{\e\first\e}\e{\second}}}%
}




% bitwise_or:
%     | bitwise_or '|' bitwise_xor 
%     | bitwise_xor
\def\@pytexParser@tryBitwiseOr{%
	\@pytexParser@tryBitwiseXor%
		\@pytexLocal@begin{current}%
		\let\current\@pytexParserReturn%
		\@pytexParser@tryBitwiseOrNext%
		\let\@pytexParserReturn\current%
		\@pytexLocal@end{current}%
}
\def\@pytexParser@tryBitwiseOrNext{%
	\@pytexParser@join{\@pytexParser@tryToken{\tokentypeBITWISEOR}}{\@pytexParser@tryBitwiseXor}%
		{\e\e\e\def\e\e\e\current\e\e\e{\e\e\e\@pytexOperator@bitwiseor\e\e\e{\e\current\e}\e{\@pytexParserReturn}}}%
		\def\next{\@pytexParser@tryBitwiseOrNext}%
	\else%
		\def\next{\relax}%
	\fi%
	\next%
}



% bitwise_xor:
%     | bitwise_xor '^' bitwise_and 
%     | bitwise_and
\def\@pytexParser@tryBitwiseXor{%
	\@pytexParser@tryBitwiseAnd%
		\@pytexLocal@begin{current}%
		\let\current\@pytexParserReturn%
		\@pytexParser@tryBitwiseXorNext%
		\let\@pytexParserReturn\current%
		\@pytexLocal@end{current}%
}
\def\@pytexParser@tryBitwiseXorNext{%
	\@pytexParser@join{\@pytexParser@tryToken{\tokentypeBITWISEXOR}}{\@pytexParser@tryBitwiseAnd}%
		{\e\e\e\def\e\e\e\current\e\e\e{\e\e\e\@pytexOperator@bitwisexor\e\e\e{\e\current\e}\e{\@pytexParserReturn}}}%
		\def\next{\@pytexParser@tryBitwiseXorNext}%
	\else%
		\def\next{\relax}%
	\fi%
	\next%
}



% bitwise_and:
%     | bitwise_and '&' shift_expr 
%     | shift_expr
\def\@pytexParser@tryBitwiseAnd{%
	\@pytexParser@tryShiftExpr%
		\@pytexLocal@begin{current}%
		\let\current\@pytexParserReturn%
		\@pytexParser@tryBitwiseAndNext%
		\let\@pytexParserReturn\current%
		\@pytexLocal@end{current}%
}
\def\@pytexParser@tryBitwiseAndNext{%
	\@pytexParser@join{\@pytexParser@tryToken{\tokentypeBITWISEAND}}{\@pytexParser@tryShiftExpr}%
		{\e\e\e\def\e\e\e\current\e\e\e{\e\e\e\@pytexOperator@bitwiseand\e\e\e{\e\current\e}\e{\@pytexParserReturn}}}%
		\def\next{\@pytexParser@tryBitwiseAndNext}%
	\else%
		\def\next{\relax}%
	\fi%
	\next%
}



% shift_expr:
%     | shift_expr '<<' sum 
%     | shift_expr '>>' sum 
%     | sum
\def\@pytexParser@tryShiftExpr{%
	\@pytexParser@trySum%
		\@pytexLocal@begin{current}%
		\let\current\@pytexParserReturn%
		\@pytexParser@tryShiftExprNext%
		\let\@pytexParserReturn\current%
		\@pytexLocal@end{current}%
}
\def\@pytexParser@tryShiftExprNext{%
	\@pytexParser@choice%
		{%
		\@pytexParser@join{\@pytexParser@tryToken{\tokentypeSHIFTLEFT}}{\@pytexParser@trySum}%
			{\e\e\e\def\e\e\e\current\e\e\e{\e\e\e\@pytexOperator@shiftleft\e\e\e{\e\current\e}\e{\@pytexParserReturn}}}%
		}{%
		\@pytexParser@join{\@pytexParser@tryToken{\tokentypeSHIFTRIGHT}}{\@pytexParser@trySum}%
			{\e\e\e\def\e\e\e\current\e\e\e{\e\e\e\@pytexOperator@shiftright\e\e\e{\e\current\e}\e{\@pytexParserReturn}}}%
		}%
		\def\next{\@pytexParser@tryShiftExprNext}%
	\else%
		\def\next{\relax}%
	\fi%
	\next%
}





% comparison:
%     | bitwise_or compare_op_bitwise_or_pair+ 
%     | bitwise_or
\def\@pytexParser@tryComparison{% TODO
	\@pytexParser@tryBitwiseOr%
}


% compare_op_bitwise_or_pair:
%     | eq_bitwise_or
%     | noteq_bitwise_or
%     | lte_bitwise_or
%     | lt_bitwise_or
%     | gte_bitwise_or
%     | gt_bitwise_or
%     | notin_bitwise_or
%     | in_bitwise_or
%     | isnot_bitwise_or
%     | is_bitwise_or



% eq_bitwise_or:    '==' bitwise_or 
% noteq_bitwise_or: '!=' bitwise_or 
% lte_bitwise_or:   '<=' bitwise_or 
% lt_bitwise_or:    '<'  bitwise_or 
% gte_bitwise_or:   '>=' bitwise_or 
% gt_bitwise_or:    '>'  bitwise_or 

% in_bitwise_or:    'in'       bitwise_or 
% notin_bitwise_or: 'not' 'in' bitwise_or 
% is_bitwise_or:    'is'       bitwise_or 
% isnot_bitwise_or: 'is' 'not' bitwise_or 




% expression:
%     | disjunction 'if' disjunction 'else' expression 
%     | disjunction
%     | lambdef
\def\@pytexParser@tryExpression{%
	\@pytexParser@choiceThree%
		{\@pytexParser@NotImplemented}% TODO
		{\@pytexParser@tryDisjunction}%
		{\@pytexParser@NotImplemented}% TODO
}

% yield_expr:
%     | 'yield' 'from' expression 
%     | 'yield' [star_expressions] 
% TODO

% star_expressions:
%     | star_expression (',' star_expression )+ [','] 
%     | star_expression ',' 
%     | star_expression
\def\@pytexParser@tryStarExpressions{%
	\@pytexParser@joinOptional%
		{\@pytexParser@tryStarExpressionsBare}%
		{\@pytexParser@tryToken{\tokentypeCOMMA}}%
		{\e\def\e\@pytexTMP@parserReturnValue\e{\first}}%
}
\def\@pytexParser@tryStarExpressionsBare{%
	\@pytexParser@joinOptional{\@pytexParser@tryStarExpression}%
		{\@pytexParser@joinSecond{\@pytexParser@tryToken{\tokentypeCOMMA}}{\@pytexParser@tryStarExpressionsBare}}%
		{\e\e\e\def\e\e\e\@pytexTMP@parserReturnValue\e\e\e{\e\e\e\@pytexOperator@starExpressionList\e\e\et{\e\first\e}\e{\second}}}%
}


% star_expression:
%     | '*' bitwise_or 
%     | expression
\def\@pytexParser@tryStarExpression{%
	\@pytexParser@choice%
		{\@pytexParser@join%
			{\@pytexParser@tryToken{\tokentypeMULTIPLY}}%
			{\@pytexParser@tryBitwiseOr}%
			{\e\def\e\@pytexTMP@parserReturnValue\e{\e\@pytexOperator@unpack\e{\second}}}%
		}%
		{\@pytexParser@tryExpression}%
}


% star_named_expressions: ','.star_named_expression+ [','] 
% TODO

% star_named_expression:
%     | '*' bitwise_or 
%     | named_expression
% TODO

% assignment_expression:
%     | NAME ':=' ~ expression 
% TODO
\def\@pytexParser@tryAssignmentExpression{\iffalse }%

% named_expression:
%     | assignment_expression
%     | expression !':='
\def\@pytexParser@tryNamedExpression{%
	\@pytexParser@choice%
		{\@pytexParser@tryAssignmentExpression}%
		{\@pytexParser@tryExpression}% TODO: do we need !':=' ?
}


% disjunction:
%     | conjunction ('or' conjunction )+ 
%     | conjunction
\def\@pytexParser@tryDisjunction{%
	\@pytexParser@tryConjunction%
		\@pytexLocal@begin{current}%
		\e\def\e\current\e{\@pytexTMP@parserReturnValue}%
		%
		\@pytexParser@tryDisjunctionNext%
		%
		\e\def\e\@pytexTMP@parserReturnValue\e{\current}%
		\@pytexLocal@end{current}%
}
\def\@pytexParser@tryDisjunctionNext{%
	\@pytexParser@join{\@pytexParser@tryToken{\tokentypeOR}}{\@pytexParser@tryConjunction}%
		{\e\e\e\def\e\e\e\current\e\e\e{\e\e\e\@pytexOperator@or\e\e\e{\e\current\e}\e{\@pytexTMP@parserReturnValue}}}%
		\def\next{\@pytexParser@tryDisjunctionNext}%
	\else%
		\def\next{\relax}%
	\fi%
	\next%
}


% conjunction:
%     | inversion ('and' inversion )+ 
%     | inversion
\def\@pytexParser@tryConjunction{%
	\@pytexParser@tryInversion
		\@pytexLocal@begin{current}%
		\e\def\e\current\e{\@pytexTMP@parserReturnValue}%
		\@pytexParser@tryConjunctionNext%
		\e\def\e\@pytexTMP@parserReturnValue\e{\current}%
		\@pytexLocal@end{current}%
}
\def\@pytexParser@tryConjunctionNext{%
	\@pytexParser@join{\@pytexParser@tryToken{\tokentypeAND}}{\@pytexParser@tryInversion}%
		{\e\e\e\def\e\e\e\current\e\e\e{\e\e\e\@pytexOperator@and\e\e\e{\e\current\e}\e{\@pytexTMP@parserReturnValue}}}%
		\def\next{\@pytexParser@tryConjunctionNext}%
	\else%
		\def\next{\relax}%
	\fi%
	\next%
}



% inversion:
%     | 'not' inversion 
%     | comparison
\def\@pytexParser@tryInversion{%
	\@pytexParser@choice{\@pytexParser@tryComparison}%
		{%
			\@pytexParser@join{\@pytexParser@tryToken{\tokentypeNOT}} {\@pytexParser@tryInversion}%
				{\e\def\e\@pytexTMP@parserReturnValue\e{\e\@pytexOperator@not\e{\second}}}%
		}%
}

% https://docs.python.org/3/reference/simple_stmts.html



% expression_stmt ::= starred_expression | expression
\def\@pytexParser@tryExpressionStmt{% TODO
	\@pytexParser@tryExpression%
}

% assert_stmt ::= "assert" expression ["," expression]
\def\@pytexParser@tryAssertStmt{% TODO
	\iffalse%
}

% target ::= identifier
%            | "(" [target_list] ")" TODO
%            | "[" [target_list] "]" TODO
%            | attributeref 	TODO
%            | subscription 	TODO
%            | slicing 			TODO
%            | "*" target 		TODO
\def\@pytexParser@tryTarget{%
	\@pytexParser@tryIdentifier%
}

% target_list ::= target ("," target)* [","]
\def\@pytexParser@tryTargetList{%
	\@pytexParser@tryTargetListBare% target ("," target)*
		\wrapif{\@pytexParser@tryToken{\tokentypeCOMMA}}{}{}% [","] % TODO: technically could mangle parserReturnValue, should use \joinOptional instead
}
\def\@pytexParser@tryTargetListBare{%
	\@pytexParser@joinOptional{\@pytexParser@tryTarget}%
		{\@pytexParser@joinSecond{\@pytexParser@tryToken{\tokentypeCOMMA}}{\@pytexParser@tryTargetListBare}}%
		{\e\e\e\def\e\e\e\@pytexTMP@parserReturnValue\e\e\e{\e\e\e\@pytexOperator@targetList\e\e\et{\e\first\e}\e{\second}}}%
}

\def\@pytexParser@tryYieldExpr{\iffalse}% TODO

% assignment_stmt ::= (target_list "=")+ (starred_expression | yield_expression)
\def\@pytexParser@tryAssignmentStmt{% TODO
	\@pytexParser@join%
		{\@pytexParser@tryAsignmentList}%
		{\@pytexParser@choice{\@pytexParser@tryStarExpression}{\@pytexParser@tryYieldExpr}}%
		{\e\e\e\def\e\e\e\@pytexTMP@parserReturnValue\e\e\e{\e\e\e\@pytexOperator@assignment\e\e\e{\e\first\e}\e{\second}}}%
}
\def\@pytexParser@tryAsignmentList{% (target_list "=")+
	\@pytexParser@joinOptional%
		{\@pytexParser@joinFirst{\@pytexParser@tryTargetList}{\@pytexParser@tryToken{\tokentypeASSIGN}}}% target_list "="
		{\@pytexParser@tryAsignmentList}% repeat
		{\e\e\e\def\e\e\e\@pytexTMP@parserReturnValue\e\e\e{\e\e\e\@pytexOperator@assignmentList\e\e\et{\e\first\e}\e{\second}}}%
}


% augmented_assignment_stmt ::= augtarget augop (expression_list | yield_expression)
% augtarget                 ::= identifier | attributeref | subscription | slicing
% augop                     ::= "+=" | "-=" | "*=" | "@=" | "/=" | "//=" | "%=" | "**="
%                               | ">>=" | "<<=" | "&=" | "^=" | "|="
\def\@pytexParser@tryAugmentedAssignmentStmt{% TODO
	\iffalse%
}


% annotated_assignment_stmt ::= augtarget ":" expression
%                               ["=" (starred_expression | yield_expression)]
\def\@pytexParser@tryAnnotatedAssignmentStmt{% TODO
	\iffalse%
}

% pass_stmt ::= "pass"
\def\@pytexParser@tryPassStmt{%
	\@pytexParser@tryToken{\tokentypePASS}%
		\def\@pytexTMP@parserReturnValue{\@pytexOperator@pass}% TODO: does this make sense?
}

% del_stmt ::= "del" target_list
\def\@pytexParser@tryDelStmt{% TODO
	\iffalse%
}



% yield_stmt ::= yield_expression
\def\@pytexParser@tryYieldStmt{% TODO
	\iffalse%\@pytexParser@tryYieldExpr%
}

% raise_stmt ::= "raise" [expression ["from" expression]]
\def\@pytexParser@tryRaiseStmt{% TODO
	\iffalse%
}

% break_stmt ::= "break"
\def\@pytexParser@tryBreakStmt{%
	\@pytexParser@tryToken{\tokentypeBREAK}%
		\def\@pytexTMP@parserReturnValue{\@pytexOperator@break}%
}

% continue_stmt ::= "continue"
\def\@pytexParser@tryContinueStmt{%
	\@pytexParser@tryToken{\tokentypeCONTINUE}%
		\def\@pytexTMP@parserReturnValue{\@pytexOperator@continue}%
}

%
\def\@pytexParser@tryImportStmt{% WONT BE IMPLEMENTED
	\iffalse%
}

%
\def\@pytexParser@tryFutureStmt{% WONT BE IMPLEMENTED
	\iffalse%
}

% global_stmt ::= "global" identifier ("," identifier)*
\def\@pytexParser@tryGlobalStmt{% TODO
	\iffalse%
}

% nonlocal_stmt ::= "nonlocal" identifier ("," identifier)*
\def\@pytexParser@tryNonlocalStmt{% TODO
	\iffalse%
}

% type_stmt ::= 'type' identifier [type_params] "=" expression
\def\@pytexParser@tryTypeStmt{% TODO
	\iffalse%
}






% simple_stmt ::= expression_stmt
%               | assert_stmt
%               | assignment_stmt
%               | augmented_assignment_stmt
%               | annotated_assignment_stmt
%               | pass_stmt
%               | del_stmt
%               | return_stmt
%               | yield_stmt
%               | raise_stmt
%               | break_stmt
%               | continue_stmt
%               | import_stmt
%               | future_stmt
%               | global_stmt
%               | nonlocal_stmt
%               | type_stmt
\def\@pytexParser@trySimpleStmt{%
	\@pytexParser@choiceFour%
	{\@pytexParser@choiceFive%
		{\@pytexParser@tryAssertStmt}%
		{\@pytexParser@tryAssignmentStmt}%
		{\@pytexParser@tryAugmentedAssignmentStmt}%
		{\@pytexParser@tryAnnotatedAssignmentStmt}%
		{\@pytexParser@tryExpressionStmt}%
	}{\@pytexParser@choiceFive%
		{\@pytexParser@tryPassStmt}%
		{\@pytexParser@tryDelStmt}%
		{\@pytexParser@tryReturnStmt}%
		{\@pytexParser@tryYieldStmt}%
		{\@pytexParser@tryRaiseStmt}%
	}{\@pytexParser@choiceFive%
		{\@pytexParser@tryBreakStmt}%
		{\@pytexParser@tryContinueStmt}%
		{\@pytexParser@tryImportStmt}%
		{\@pytexParser@tryFutureStmt}%
		{\@pytexParser@tryGlobalStmt}%
	}{\@pytexParser@choice%
		{\@pytexParser@tryNonlocalStmt}%
		{\@pytexParser@tryTypeStmt}%
	}%
}

%%%%%%%%%%%%%%%%%%%%%%%%%%%%%%%%%%%%%%%%%%%%%%%%%%
%%%%%%%%%%%%%%%%%%%%%%%%%%%%%%%%%%%%%%%%%%%%%%%%%%
%%%%%%%%%%%%%%%%%%%%%%%%%%%%%%%%%%%%%%%%%%%%%%%%%%
%%%%%%%%%%%%%%%%% NEW STUFF %%%%%%%%%%%%%%%%%%%%%%
%%%%%%%%%%%%%%%%%%%%%%%%%%%%%%%%%%%%%%%%%%%%%%%%%%
%%%%%%%%%%%%%%%%%%%%%%%%%%%%%%%%%%%%%%%%%%%%%%%%%%
%%%%%%%%%%%%%%%%%%%%%%%%%%%%%%%%%%%%%%%%%%%%%%%%%%


% simple_stmts:
%     | simple_stmt !';' NEWLINE  # Not needed, there for speedup
%     | ';'.simple_stmt+ [';'] NEWLINE 
% TODO




% # NOTE: assignment MUST precede expression, else parsing a simple assignment
% # will throw a SyntaxError.
% simple_stmt:
%     | assignment
%     | type_alias
%     | star_expressions 
%     | return_stmt
%     | import_stmt
%     | raise_stmt
%     | 'pass' 
%     | del_stmt
%     | yield_stmt
%     | assert_stmt
%     | 'break' 
%     | 'continue' 
%     | global_stmt
%     | nonlocal_stmt
% TODO




% compound_stmt:
%     | function_def
%     | if_stmt
%     | class_def
%     | with_stmt
%     | for_stmt
%     | try_stmt
%     | while_stmt
%     | match_stmt
% TODO




% # NOTE: annotated_rhs may start with 'yield'; yield_expr must start with 'yield'
% assignment:
%     | NAME ':' expression ['=' annotated_rhs ] 
%     | ('(' single_target ')' 
%          | single_subscript_attribute_target) ':' expression ['=' annotated_rhs ] 
%     | (star_targets '=' )+ (yield_expr | star_expressions) !'=' [TYPE_COMMENT] 
%     | single_target augassign ~ (yield_expr | star_expressions) 
% TODO




% annotated_rhs: yield_expr | star_expressions
% TODO




% augassign:
%     | '+=' 
%     | '-=' 
%     | '*=' 
%     | '@=' 
%     | '/=' 
%     | '%=' 
%     | '&=' 
%     | '|=' 
%     | '^=' 
%     | '<<=' 
%     | '>>=' 
%     | '**=' 
%     | '//=' 
% TODO




%return_stmt:
%    | 'return' [star_expressions] 
\def\@pytexParser@tryReturnStmt{%
	\@pytexParser@joinOptionalDouble%
		{\@pytexParser@tryToken{\tokentypeRETURN}}%
		{\@pytexParser@tryStarExpressions}%
		{\def\@pytexTMP@parserReturnValue{\@pytexOperator@return{None}}} % TODO: actually return None
		{\e\def\e\@pytexTMP@parserReturnValue\e{\e\@pytexOperator@return\e{\second}}}%
}


% raise_stmt:
%     | 'raise' expression ['from' expression ] 
%     | 'raise' 
% TODO




% global_stmt: 'global' ','.NAME+ 
% TODO




% nonlocal_stmt: 'nonlocal' ','.NAME+ 
% TODO




% del_stmt:
%     | 'del' del_targets &(';' | NEWLINE) 
% TODO




% yield_stmt: yield_expr 
% TODO




% assert_stmt: 'assert' expression [',' expression ] 
% TODO




% import_stmt:
%     | import_name
%     | import_from
% TODO - wont implement (probably)





% statements: statement+ 
\def\@pytexParser@tryStatements{%
	\@pytexParser@joinOptional%
		{\@pytexParser@tryStatement}%
		{\@pytexParser@tryStatements}%
		{\e\e\e\def\e\e\e\@pytexTMP@parserReturnValue\e\e\e{\e\first\second}}%
}


% compound_stmt:
%     | function_def
%     | if_stmt 		TODO
%     | class_def 		TODO
%     | with_stmt 		TODO
%     | for_stmt 		TODO
%     | try_stmt 		TODO
%     | while_stmt 		TODO
%     | match_stmt 		TODO
\def\@pytexParser@tryCompoundStmt{%
	\@pytexParser@choice%
		{\@pytexParser@tryFunctionDef}%
		{\@pytexParser@tryIfStmt}%
}



% block:
%     | NEWLINE INDENT statements DEDENT 
%     | simple_stmts
\def\@pytexParser@tryBlock{%
	\@pytexParser@choice%
		{%
			\@pytexParser@joinSecond{\@pytexParser@tryToken{\tokentypeNEWLINE}}%
				{%
					\@pytexParser@joinSecond{\@pytexParser@tryToken{\tokentypeINDENT}}%
						{%
							\@pytexParser@joinFirst{\@pytexParser@tryStatements}{\@pytexParser@tryToken{\tokentypeDEDENT}}	%
						}%
				}%
		}%
		{\@pytexParser@trySimpleStmts}%
}



% decorators: ('@' named_expression NEWLINE )+ 
% TODO




% class_def:
%     | decorators class_def_raw 
%     | class_def_raw
% TODO




% class_def_raw:
%     | 'class' NAME [type_params] ['(' [arguments] ')' ] ':' block 
% TODO



 
% function_def:
%     | decorators function_def_raw TODO
%     | function_def_raw
\def\@pytexParser@tryFunctionDef{%
	\@pytexParser@tryFunctionDefRaw%
}




% function_def_raw:
%     | 'def' NAME [type_params] '(' [params] ')' ['->' expression ] ':' [func_type_comment] block 
%     | 'async' 'def' NAME [type_params] '(' [params] ')' ['->' expression ] ':' [func_type_comment] block 
\def\@pytexParser@tryFunctionDefRaw{%
	\iffalse % TODO
}




% if_stmt:
%     | 'if' named_expression ':' block elif_stmt
%     | 'if' named_expression ':' block [else_block]
\def\@pytexParser@tryIfStmt{%
	\@pytexParser@joinSecond%
		{\@pytexParser@tryToken{\tokentypeIF}}%
		{\@pytexParser@tryIfStmtBare}%
}

% if_stmt_bare:
%     | named_expression ':' block elif_stmt
%     | named_expression ':' block [else_block]
\def\@pytexParser@tryIfStmtBare{%
	\@pytexParser@join%
		{\@pytexParser@tryNamedExpression}% named_expression
		{\@pytexParser@joinSecond%
			{\@pytexParser@tryToken{\tokentypeCOLON}}% ':'
			{\@pytexParser@tryBlock}%
		}%
		{\e\e\e\def\e\e\e\@pytexTMP@parserReturnValue\e\e\e{\e\e\e{\e\first\e}\e{\second}}} % format as macro arguments, to add \operatorIf or \operatorIfElse later
		%
		% test whether we have an if-else (or if-elif), or only if
		\e\wrapifsw\e{\e\def\e\first\e{\@pytexTMP@parserReturnValue}%
			% we have an if-else
			\e\e\e\def\e\e\e\@pytexTMP@parserReturnValue\e\e\e{\e\e\e\@pytexOperator@ifelse\e\first\e{\@pytexTMP@parserReturnValue}}%
		}%
		{\@pytexParser@choice{\@pytexParser@tryElseBlock}{\@pytexParser@tryElifStmt}}%
		{%
			% we only have an if
			\e\def\e\@pytexTMP@parserReturnValue\e{\e\@pytexOperator@if\@pytexTMP@parserReturnValue}%
		}%
}

% elif_stmt:
%     | 'elif' named_expression ':' block elif_stmt
%     | 'elif' named_expression ':' block [else_block]
\def\@pytexParser@tryElifStmt{%
	\@pytexParser@joinSecond%
		{\@pytexParser@tryToken{\tokentypeELIF}}%
		{\@pytexParser@tryIfStmtBare}%
}

% else_block:
%     | 'else' ':' block
\def\@pytexParser@tryElseBlock{%
	\@pytexParser@joinSecond%
		{\@pytexParser@tryToken{\tokentypeELSE}}%
		{\@pytexParser@joinSecond%
			{\@pytexParser@tryToken{\tokentypeCOLON}}%
			{\@pytexParser@tryBlock}%
		}%
}






















\def\@pytexParser@skipEmptyLines{%
	\ifmatch{\tokentypeNEWLINE}%
		\def\next{\@pytexParser@skipEmptyLines}%
	\else%
		\def\next{\relax}%
	\fi%
	\next%
}

% statement: compound_stmt  | simple_stmts
\def\@pytexParser@tryStatement{%
	\@pytexParser@choice%
		{\@pytexParser@trySimpleStmts}%
		{\@pytexParser@tryCompoundStmt}%
}

\def\@pytexParser@parseNext{%
	\@pytexParser@skipEmptyLines%
	\ifmatch{\tokentypeEND}%
		\def\next{\relax}%
	\else%
		\wrapif{\@pytexParser@tryStatement}{}%
		{%
			\let\@pytexParserReturn\undefined%
			EXPECTED SOMETHING!
		}%
		%
		%\show\@pytexParserReturn%
		\@pytexParserReturn%
		%
		\def\next{\@pytexParser@parseNext}%
	\fi%
	\next%
}

\def\@pytexParser@parse{%
	\@pytexLocal@new{current}%
	\@pytexStack@new{idxsavestack}%
	\@pytexStack@new{resultsavestack}%
	\def\currentidx{0}%
	\def\@pytexParserReturn{\relax}%
	%
	\@pytexParser@parseNext%
	%
	\let\@pytexParserReturn\undefined%
	\@pytexLocal@delete{current}%
	\@pytexStack@delete{idxsavestack}%
	\@pytexStack@delete{resultsavestack}%
}

