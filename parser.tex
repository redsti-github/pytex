
% token type enum
\def\tokentypeEND{0}
\def\enumnext#1{%
	\e\edef\csname tokentype#1\endcsname{\tokentypeEND}%
	\edef\tokentypeEND{\the\numexpr\tokentypeEND+1}%
}

\enumnext{NEWLINE}

\enumnext{IDENTIFIER}
\enumnext{NUMBER}

\enumnext{AND}
\enumnext{OR}
\enumnext{NOT}
\enumnext{IS}
\enumnext{ISNOT}
\enumnext{IN}
\enumnext{NOTIN}

\enumnext{EXPONENTIATION}
\enumnext{INVERT}
\enumnext{MULTIPLY}
\enumnext{MATMULT}
\enumnext{FLOORDIVIDE}
\enumnext{DIVIDE}
\enumnext{MODULO}

\enumnext{ADD}
\enumnext{SUBTRACT}

\enumnext{EQUAL}
\enumnext{NOTEQUAL}
\enumnext{LESSTHAN}
\enumnext{LESSEQUAL}
\enumnext{GREATER}
\enumnext{GREATEREQUAL}

\enumnext{ASSIGN}

% keywords
\enumnext{AWAIT}
\enumnext{IF}
\enumnext{ELSE}
\enumnext{FOR}



% token macros - sets \toketype and, optionally, \tokenvalue
\def\@pytexToken@Newline{\edef\tokentype{\tokentypeNEWLINE}}
\def\@pytexToken@Identifier#1{\def\tokentype{\tokentypeIDENTIFIER}\def\tokenvalue{#1}}
\def\@pytexToken@Number#1{\def\tokentype{\tokentypeNUMBER}\def\tokenvalue{#1}}

\def\@pytexToken@and{\def\tokentype{\tokentypeAND}}
\def\@pytexToken@or{\def\tokentype{\tokentypeOR}}
\def\@pytexToken@in{\def\tokentype{\tokentypeIN}}
\def\@pytexToken@is{\def\tokentype{\tokentypeIS}}
\def\@pytexToken@not{\def\tokentype{\tokentypeNOT}}

\def\@pytexToken@Addition{\def\tokentype{\tokentypeADD}}
\def\@pytexToken@Subtraction{\def\tokentype{\tokentypeSUBTRACT}}
\def\@pytexToken@Multiplication{\def\tokentype{\tokentypeMULTIPLY}}
\def\@pytexToken@Division{\def\tokentype{\tokentypeDIVIDE}}
\def\@pytexToken@FloorDivision{\def\tokentype{\tokentypeFLOORDIVIDE}}
\def\@pytexToken@EqualTo{\def\tokentype{\tokentypeEQUAL}}
\def\@pytexToken@NotEqualTo{\def\tokentype{\tokentypeNOTEQUAL}}
\def\@pytexToken@Exponentiation{\def\tokentype{\tokentypeEXPONENTIATION}}
\def\@pytexToken@Modulo{\def\tokentype{\tokentypeMODULO}}

\def\@pytexToken@Assignment{\def\tokentype{\tokentypeASSIGN}}

\def\ifmatch#1{%
	\e\@pytexTokeniser@TokenList@getim\e{\currentidx}%
	\ifnum \tokentype = #1 %
		\edef\currentidx{\the\numexpr\currentidx+1}%
}
\def\ifpeek#1{%
	\e\@pytexTokeniser@TokenList@getim\e{\currentidx}%
	\ifnum \tokentype = #1 %
}



\def\currentidx{0}
\@pytexStack@new{idxsavestack}

\def\savestate{%
	\e\idxsavestack@push\e{\currentidx}%
}
\def\restorestate{%
	\@pytexStack@pop{idxsavestack}{\currentidx}%
}
\def\commitstate{%
	\@pytexStack@popd{idxsavestack}%
}



% #1 = if, #2 = iftrue, #3 = iffalse
\def\wrapif#1#2#3{
	#1 %
		#2 %
	\else %
		#3 %
	\fi %
}

% #1 = iftrue, #2 = if, #3 = iffalse
\def\wrapifsw#1#2#3{
	#2 %
		#1 %
	\else %
		#3 %
	\fi %
}

% result ::= #1
\def\@pytexParser@tryToken#1{%
	\ifmatch{#1}%
		\e\def\e\@pytexTMP@parserReturnValue\e{\tokenvalue}%
}

% result ::= #1 | #2
\def\@pytexParser@choice#1#2{%
	\wrapif{#1}{%
		\def\ifresult{1}%
	}{%
		\wrapif{#2}{
			\def\ifresult{1}%
		}{%
			\def\ifresult{0}%
		}%
	}%
	\ifnum\ifresult=1 %
}

% result ::= #1 | #2 | #3
\def\@pytexParser@choiceThree#1#2#3{%
	\@pytexParser@choice{#1}{\@pytexParser@choice{#2}{#3}}%
}

% result ::= #1 | #2 | #3 | #4
\def\@pytexParser@choiceFour#1#2#3#4{%
	\@pytexParser@choice{#1}{\@pytexParser@choiceThree{#2}{#3}{#4}}%
}

% result ::= #1 | #2 | #3 | #4 | #5
\def\@pytexParser@choiceFive#1#2#3#4#5{%
	\@pytexParser@choice{#1}{\@pytexParser@choiceFour{#2}{#3}{#4}{#5}}%
}

% result ::= #1 | #2 | #3 | #4 | #5 | #6
\def\@pytexParser@choiceSix#1#2#3#4#5#6{%
	\@pytexParser@choice{#1}{\@pytexParser@choiceFive{#2}{#3}{#4}{#5}{#6}}%
}



% #1 = first thing, #2 = second thing, #3 = joining operation (results of #1 and #2 are in \first and \second)
% result ::= #1 #2
\def\@pytexParser@join#1#2#3{%
	\savestate%
	\wrapif{#1}{%
		\e\wrapifsw\e{%
			\e\def\e\first\e{\@pytexTMP@parserReturnValue}%
			\let\second\@pytexTMP@parserReturnValue%
			#3%
			\def\ifresult{1}%
			\commitstate%
		}{#2}{
			\def\ifresult{0}%
			\restorestate%
		}
	}{
		\def\ifresult{0}%
		\restorestate%
	}
	\ifnum\ifresult=1 %
}

% #1 = first thing, #2 = second thing, #3 = joining operation (provided with \first and \second), only executed when #2 is present
% result ::= #1 [#2]	-> {if(#2): #3}
\def\@pytexParser@joinOptional#1#2#3{%
	\savestate%
	\wrapif{#1}{%
		\e\wrapifsw\e{%
			\e\def\e\first\e{\@pytexTMP@parserReturnValue}%
			\let\second\@pytexTMP@parserReturnValue%
			#3%
			\def\ifresult{1}%
			\commitstate%
		}{#2}{
			\def\ifresult{1}%
			\commitstate%
		}
	}{
		\def\ifresult{0}%
		\restorestate%
	}
	\ifnum\ifresult=1 %
}




% expression:
%     | disjunction 'if' disjunction 'else' expression 
%     | disjunction
%     | lambdef
\def\@pytexParser@tryExpression{%
	\@pytexParser@choiceThree%
		{\@pytexParser@NotImplemented}% TODO
		{\@pytexParser@tryDisjunction}%
		{\@pytexParser@NotImplemented}% TODO
}

% yield_expr:
%     | 'yield' 'from' expression 
%     | 'yield' [star_expressions] 
% TODO

% star_expressions:
%     | star_expression (',' star_expression )+ [','] 
%     | star_expression ',' 
%     | star_expression
\def\@pytexParser@tryStarExpressions{%
	\@pytexParser@joinOptional%
		{\@pytexParser@tryStarExpressionsBare}%
		{\@pytexParser@tryToken{\tokentypeCOMMA}}%
		{\e\def\e\@pytexTMP@parserReturnValue\e{\first}}%
}
\def\@pytexParser@tryStarExpressionsBare{%
	\@pytexParser@joinOptional{\@pytexParser@tryStarExpression}%
		{\@pytexParser@joinSecond{\@pytexParser@tryToken{\tokentypeCOMMA}}{\@pytexParser@tryStarExpressionsBare}}%
		{\e\e\e\def\e\e\e\@pytexTMP@parserReturnValue\e\e\e{\e\e\e\@pytexOperator@starExpressionList\e\e\et{\e\first\e}\e{\second}}}%
}


% star_expression:
%     | '*' bitwise_or 
%     | expression
\def\@pytexParser@tryStarExpression{%
	\@pytexParser@choice%
		{\@pytexParser@join%
			{\@pytexParser@tryToken{\tokentypeMULTIPLY}}%
			{\@pytexParser@tryBitwiseOr}%
			{\e\def\e\@pytexTMP@parserReturnValue\e{\e\@pytexOperator@unpack\e{\second}}}%
		}%
		{\@pytexParser@tryExpression}%
}


% star_named_expressions: ','.star_named_expression+ [','] 
% TODO

% star_named_expression:
%     | '*' bitwise_or 
%     | named_expression
% TODO

% assignment_expression:
%     | NAME ':=' ~ expression 
% TODO
\def\@pytexParser@tryAssignmentExpression{\iffalse }%

% named_expression:
%     | assignment_expression
%     | expression !':='
\def\@pytexParser@tryNamedExpression{%
	\@pytexParser@choice%
		{\@pytexParser@tryAssignmentExpression}%
		{\@pytexParser@tryExpression}% TODO: do we need !':=' ?
}


% disjunction:
%     | conjunction ('or' conjunction )+ 
%     | conjunction
\def\@pytexParser@tryDisjunction{%
	\@pytexParser@tryConjunction%
		\@pytexLocal@begin{current}%
		\e\def\e\current\e{\@pytexTMP@parserReturnValue}%
		%
		\@pytexParser@tryDisjunctionNext%
		%
		\e\def\e\@pytexTMP@parserReturnValue\e{\current}%
		\@pytexLocal@end{current}%
}
\def\@pytexParser@tryDisjunctionNext{%
	\@pytexParser@join{\@pytexParser@tryToken{\tokentypeOR}}{\@pytexParser@tryConjunction}%
		{\e\e\e\def\e\e\e\current\e\e\e{\e\e\e\@pytexOperator@or\e\e\e{\e\current\e}\e{\@pytexTMP@parserReturnValue}}}%
		\def\next{\@pytexParser@tryDisjunctionNext}%
	\else%
		\def\next{\relax}%
	\fi%
	\next%
}


% conjunction:
%     | inversion ('and' inversion )+ 
%     | inversion
\def\@pytexParser@tryConjunction{%
	\@pytexParser@tryInversion
		\@pytexLocal@begin{current}%
		\e\def\e\current\e{\@pytexTMP@parserReturnValue}%
		\@pytexParser@tryConjunctionNext%
		\e\def\e\@pytexTMP@parserReturnValue\e{\current}%
		\@pytexLocal@end{current}%
}
\def\@pytexParser@tryConjunctionNext{%
	\@pytexParser@join{\@pytexParser@tryToken{\tokentypeAND}}{\@pytexParser@tryInversion}%
		{\e\e\e\def\e\e\e\current\e\e\e{\e\e\e\@pytexOperator@and\e\e\e{\e\current\e}\e{\@pytexTMP@parserReturnValue}}}%
		\def\next{\@pytexParser@tryConjunctionNext}%
	\else%
		\def\next{\relax}%
	\fi%
	\next%
}



% inversion:
%     | 'not' inversion 
%     | comparison
\def\@pytexParser@tryInversion{%
	\@pytexParser@choice{\@pytexParser@tryComparison}%
		{%
			\@pytexParser@join{\@pytexParser@tryToken{\tokentypeNOT}} {\@pytexParser@tryInversion}%
				{\e\def\e\@pytexTMP@parserReturnValue\e{\e\@pytexOperator@not\e{\second}}}%
		}%
}













\def\@pytexParser@statementAssign{ %
	\@pytexLocal@begin{current}%
%
%	
%	
	\@pytexLocal@end{current}%
}

\def\@pytexParser@skipEmptyLines{%
	\ifmatch{0}%
		\def\next{\@pytexParser@skipEmptyLines}%
	\else%
		\def\next{\relax}%
	\fi%
	\next%
}

\def\@pytexParser@statement{%
	\@pytexParser@skipEmptyLines%
	\@pytexParser@expression% TODO: expect newline after (or ';')
}

\def\@pytexParser@parseNext{%
	\def\parseNext{\relax}%
	\ifnum \@pytexTokeniser@TokenList@size > 0 %
		\@pytexParser@statement%
		\show\@pytexTMP@parserReturnValue%
	\fi %
	\parseNext%
}

\def\@pytexParser@parse{
	\@pytexLocal@new{current}%
	\@pytexTokeniser@TokenList@peekim%
	\e\@pytexTokeniser@TokenList@push\e{\tokentypeEND}%
	
	\@pytexParser@parseNext%
	
	\let\@pytexTMP@parserReturnValue\undefined%
}




