
% token type enum
\def\tokentypeEND{0}
\def\enumnext#1{%
	\e\edef\csname tokentype#1\endcsname{\tokentypeEND}%
	\edef\tokentypeEND{\the\numexpr\tokentypeEND+1}%
}

\enumnext{NEWLINE}

\enumnext{IDENTIFIER}
\enumnext{NUMBER}

\enumnext{AND}
\enumnext{OR}
\enumnext{NOT}
\enumnext{IS}
\enumnext{ISNOT}
\enumnext{IN}
\enumnext{NOTIN}

\enumnext{ADD}
\enumnext{SUBTRACT}
\enumnext{MULTIPLY}
\enumnext{DIVIDE}
\enumnext{FLOORDIVIDE}
\enumnext{EXPONENTIATION}
\enumnext{MODULO}

\enumnext{EQUAL}
\enumnext{NOTEQUAL}
\enumnext{LESSTHAN}
\enumnext{LESSEQUAL}
\enumnext{GREATER}
\enumnext{GREATEREQUAL}

\enumnext{ASSIGN}




% token macros - sets \toketype and, optionally, \tokenvalue
\def\@pytexToken@Newline{\edef\tokentype{\tokentypeNEWLINE}}
\def\@pytexToken@Identifier#1{\def\tokentype{\tokentypeIDENTIFIER}\def\tokenvalue{#1}}
\def\@pytexToken@Number#1{\def\tokentype{\tokentypeNUMBER}\def\tokenvalue{#1}}

\def\@pytexToken@and{\def\tokentype{\tokentypeAND}}
\def\@pytexToken@or{\def\tokentype{\tokentypeOR}}
\def\@pytexToken@in{\def\tokentype{\tokentypeIN}}
\def\@pytexToken@is{\def\tokentype{\tokentypeIS}}
\def\@pytexToken@not{\def\tokentype{\tokentypeNOT}}

\def\@pytexToken@Addition{\def\tokentype{\tokentypeADD}}
\def\@pytexToken@Subtraction{\def\tokentype{\tokentypeSUBTRACT}}
\def\@pytexToken@Multiplication{\def\tokentype{\tokentypeMULTIPLY}}
\def\@pytexToken@Division{\def\tokentype{\tokentypeDIVIDE}}
\def\@pytexToken@FloorDivision{\def\tokentype{\tokentypeFLOORDIVIDE}}
\def\@pytexToken@EqualTo{\def\tokentype{\tokentypeEQUAL}}
\def\@pytexToken@NotEqualTo{\def\tokentype{\tokentypeNOTEQUAL}}
\def\@pytexToken@Exponentiation{\def\tokentype{\tokentypeEXPONENTIATION}}
\def\@pytexToken@Modulo{\def\tokentype{\tokentypeMODULO}}

\def\@pytexToken@Assignment{\def\tokentype{\tokentypeASSIGN}}

\def\ifmatch#1{%
	\e\@pytexTokeniser@TokenList@getim\e{\currentidx}%
	\ifnum \tokentype = #1 %
		\edef\currentidx{\the\numexpr\currentidx+1}%
}
\def\ifpeek#1{%
	\e\@pytexTokeniser@TokenList@getim\e{\currentidx}%
	\ifnum \tokentype = #1 %
}



\def\currentidx{0}

\def\savestate#1{%
	\e\e\e\edef\csname currentidx@#1\endcsname{\currentidx}%
}
\def\restorestate#1{%
	\edef\currentidx{\csname currentidx@#1\endcsname}
	\e\e\e\let\csname currentidx@#1\endcsname\undefined%
	%\e\@pytexTokeniser@TokenList@getim\e{\currentidx}%
}
%\def\commitstate#1{%
%	\e\e\e\let\csname currentidx@#1\endcsname\undefined%
%}

% #1 = if, #2 = iftrue, #3 = iffalse
\def\wrapif#1#2#3{
	#1 %
		#2 %
	\else %
		#3 %
	\fi %
}



% https://www.programiz.com/python-programming/precedence-associativity
% https://www.wscubetech.com/resources/python/precedence-associativity-operators
% https://docs.python.org/3/reference/expressions.html#grammar-token-python-grammar-primary

% try<thing> looks ahead and tries to parse <thing>.
% if parsing was succesful:
	% sets \@pytexTMP@parserReturnValue to the parsed construct
	% consumes the parsed tokens
	% ends with \iftrue
% else,
	% leaves \@pytexTMP@parserReturnValue untouched
	% leaves tokens untouched
	% ends with \iffalse
% use \wrapif{\try<thing>}{<iftrue>}{<iffalse>} to avoid problems with ifs

\def\@pytexParser@NotImplemented{\iffalse}%



% expression:
%     | disjunction 'if' disjunction 'else' expression 
%     | disjunction
%     | lambdef
\def\@pytexParser@tryExpression{%
	\@pytexParser@choiceThree%
		{\@pytexParser@NotImplemented}% TODO
		{\@pytexParser@tryDisjunction}%
		{\@pytexParser@NotImplemented}% TODO
}

% yield_expr:
%     | 'yield' 'from' expression 
%     | 'yield' [star_expressions] 
% TODO

% star_expressions:
%     | star_expression (',' star_expression )+ [','] 
%     | star_expression ',' 
%     | star_expression
% TODO

% star_expression:
%     | '*' bitwise_or 
%     | expression
% TODO

% star_named_expressions: ','.star_named_expression+ [','] 
% TODO

% star_named_expression:
%     | '*' bitwise_or 
%     | named_expression
% TODO

% assignment_expression:
%     | NAME ':=' ~ expression 
% TODO

% named_expression:
%     | assignment_expression
%     | expression !':='
% TODO


% disjunction:
%     | conjunction ('or' conjunction )+ 
%     | conjunction
\def\@pytexParser@tryDisjunction{%
	\@pytexParser@tryConjunction%
		\@pytexLocal@begin{current}%
		\e\def\e\current\e{\@pytexTMP@parserReturnValue}%
		%
		\@pytexParser@tryDisjunctionNext%
		%
		\e\def\e\@pytexTMP@parserReturnValue\e{\current}%
		\@pytexLocal@end{current}%
}
\def\@pytexParser@tryDisjunctionNext{%
	\@pytexParser@join{\@pytexParser@tryToken{\tokentypeOR}}{\@pytexParser@tryConjunction}%
		{\e\e\e\def\e\e\e\current\e\e\e{\e\e\e\@pytexOperator@or\e\e\e{\e\current\e}\e{\@pytexTMP@parserReturnValue}}}%
		\def\next{\@pytexParser@tryDisjunctionNext}%
	\else%
		\def\next{\relax}%
	\fi%
	\next%
}


% conjunction:
%     | inversion ('and' inversion )+ 
%     | inversion
\def\@pytexParser@tryConjunction{%
	\@pytexParser@tryInversion
		\@pytexLocal@begin{current}%
		\e\def\e\current\e{\@pytexTMP@parserReturnValue}%
		\@pytexParser@tryConjunctionNext%
		\e\def\e\@pytexTMP@parserReturnValue\e{\current}%
		\@pytexLocal@end{current}%
}
\def\@pytexParser@tryConjunctionNext{%
	\@pytexParser@join{\@pytexParser@tryToken{\tokentypeAND}}{\@pytexParser@tryInversion}%
		{\e\e\e\def\e\e\e\current\e\e\e{\e\e\e\@pytexOperator@and\e\e\e{\e\current\e}\e{\@pytexTMP@parserReturnValue}}}%
		\def\next{\@pytexParser@tryConjunctionNext}%
	\else%
		\def\next{\relax}%
	\fi%
	\next%
}



% inversion:
%     | 'not' inversion 
%     | comparison
\def\@pytexParser@tryInversion{%
	\@pytexParser@choice{\@pytexParser@tryComparison}%
		{%
			\@pytexParser@join{\@pytexParser@tryToken{\tokentypeNOT}} {\@pytexParser@tryInversion}%
				{\e\def\e\@pytexTMP@parserReturnValue\e{\e\@pytexOperator@not\e{\second}}}%
		}%
}







% TODO: move this
% https://docs.python.org/3/reference/grammar.html
% star_expression:
%     | '*' bitwise_or 
%     | expression
\def\@pytexParser@tryStarExpression{%
	\@pytexParser@choice%
		{\@pytexParser@joinSecondUnary%
			{\@pytexParser@tryToken{\tokentypeMULTIPLY}}%
			{\@pytexParser@tryBitwiseOr}%
			{\e\def\e\@pytexTMP@parserReturnValue\e{\e\@pytexOperator@starred\e{\@pytexTMP@parserReturnValue}}}%
		}%
		{\@pytexParser@tryExpression}%
}
















\def\@pytexParser@statementAssign{ %
	\@pytexLocal@begin{current}%
%
%	
%	
	\@pytexLocal@end{current}%
}

\def\@pytexParser@skipEmptyLines{%
	\ifmatch{0}%
		\def\next{\@pytexParser@skipEmptyLines}%
	\else%
		\def\next{\relax}%
	\fi%
	\next%
}

\def\@pytexParser@statement{%
	\@pytexParser@skipEmptyLines%
	\@pytexParser@expression% TODO: expect newline after (or ';')
}

\def\@pytexParser@parseNext{%
	\def\parseNext{\relax}%
	\ifnum \@pytexTokeniser@TokenList@size > 0 %
		\@pytexParser@statement%
		\show\@pytexTMP@parserReturnValue%
	\fi %
	\parseNext%
}

\def\@pytexParser@parse{
	\@pytexLocal@new{current}%
	\@pytexTokeniser@TokenList@peekim%
	
	\@pytexParser@parseNext%
	
	\let\@pytexTMP@parserReturnValue\undefined%
}




